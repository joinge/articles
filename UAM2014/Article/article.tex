
%===============%                               %~~~~~~~~~~~~~~~~~~~~~~~~~~~~~~~~~~~~~~~~~~~~~~~~~%
% DOCUMENTCLASS %                                See full option description in "mytemplate.cls"
%===============%                               %~~~~~~~~~~~~~~~~~~~~~~~~~~~~~~~~~~~~~~~~~~~~~~~~~%
 
\documentclass[
   UAM                                          % Document type (non-standard)
 , 12pt                                         % Text size
%  , draft
%  , final                                        % Quality
%  , xelatex                                      % Use the XeLaTeX compiler
%  , biblatex                                     % Use 'biblatex' for references (best by far)
 , bibtex                                       % Use 'bibtex' for references (oldschool :)
%  , movie
%  , notodos                                      % Disable todos
 , layout
%  , defaultformat                                % Attempt to use default formatting options
%  , glossary                                     % Use a glossary
]{common/mytemplate}
% \bibliography{references}
\hypersetup{
   bookmarksopen=true
 , bookmarksopenlevel=2
}


\newlength\oldparindent
\setlength\oldparindent{\parindent}

\newlength\figwidth
\setlength\figwidth{0.7\linewidth}

% \setlength\floatsep{\baselineskip}
\setcounter{topnumber}{1}
% \setlength\baselineskip{12pt}
% \selectfont

\renewcommand\vec[1]{\boldsymbol{#1}}
\newcommand\mat[1]{\boldsymbol{#1}}

\renewcommand*\P{\mat P}
\newcommand*\V{\mat V}
\newcommand*\M{\mat M}

% \usepackage{caption}
\usepackage{subcaption}
\usepackage{color}


% Tune floats
\renewcommand{\topfraction}{0.9}  % max fraction of floats at top
\renewcommand{\bottomfraction}{0.8} % max fraction of floats at bottom
%   Parameters for TEXT pages (not float pages):
\setcounter{topnumber}{2}
\setcounter{bottomnumber}{2}
\setcounter{totalnumber}{4}     % 2 may work better
\setcounter{dbltopnumber}{2}    % for 2-column pages
\renewcommand{\dbltopfraction}{0.9} % fit big float above 2-col. text
\renewcommand{\textfraction}{0.07}  % allow minimal text w. figs
%   Parameters for FLOAT pages (not text pages):
\renewcommand{\floatpagefraction}{0.85}  % require fuller float pages
% N.B.: floatpagefraction MUST be less than topfraction !!
\renewcommand{\dblfloatpagefraction}{0.85} % require fuller float pages


\begin{document}
% Load the references.bib file
% \bibliography{../../../library.bib}

\pagestyle{plain}
% \layout
% \printinunitsof{cm}
% \prntlen{\baselineskip}
% \the\textwidth
% \the\linewidth
% \prntlen{\textwidth}\\
% \prntlen\linewidth\\
% \prntlen\oddsidemargin\\
% \prntlen\oddsidemargin
% \newpage



\title{A GPU Sonar Simulator for Automatic Target Recognition}%
%
\author{J.I.Buskenes\firstAddress, C.-I.C.Nilsen\secondAddress, A.Austeng\secondAddress}%
%
\begin{contact}
  \firstAddress Dept. of Informatics, P.O.Box 1080 Blindern, N-0316 OSLO
  \secondAddress Norwegian Defence Research Establishment (FFI) P.O.Box 25, NO-2027 Kjeller
\end{contact}%
%
\begin{contact}
Contact author: Jo Inge Buskenes\\
Dept. of Informatics, Univ. of Oslo, P.Box 1080 Blindern, N-0316 OSLO\\
\href{mailto:joibu@ifi.uio.no}{joibu@ifi.uio.no}
\end{contact}%
%
\begin{abstract}
Template matching is a common technique used when classifying objects in synthetic aperture sonar (SAS) images. The principle is to isolate an image segment containing an object of interest, correlate it with a set of template images, and assign it to the class of the template yielding the highest correlation coefficient. The challenge is to come up with a suitable set of template images considering that no seabed or object is truly alike.

We target this challenge with a sonar simulator that first take as input a seabed model derived from the real sonar image. Then it places an object model on the seabed, renders the scene, and adds the resulting image to the template set. For any object position, alignment, type and material, the procedure is repeated, and a correlation coefficient computed. The faster we are able to perform these simulations, the better we can expect the classification result to be. Therefore the simulator is written in OpenGL and OpenCL and run on graphics processing units (GPUs).

The result is a fast performing, mobile and portable on-the-fly template generator which can adapt its behavior to the nature of the current scene. We believe this can prove a powerful tool for mobile sonar imaging platforms such as autonomous underwater vehicles (AUVs).
\end{abstract}%
%
\keywords{Adaptive beamforming, minimum variance, MVDR, LCA, complexity, sonar}

\section{Introduction}


A synthetic aperture sonar (SAS) utilizes a wider along-track beam to ensonify the same part of the seafloor with several sucessive pings. Later this data is combined coherently to allow the synthetization of a much larger receiving array, which drastically improves the along-track resolution. 

\begin{figure}[t]\centering%
\graphicsAI[drawing,width=\linewidth]{gfx/specs.svg}%
\caption{Simulator specifications.}\label{specs}%
\end{figure}

\begin{figure}[t]\centering%
\graphicsAI[drawing,width=\linewidth]{gfx/simulator.svg}%
\caption{Simulator.}\label{buildup}%
\end{figure}


OpenGL is a framework that 

Citations
\begin{itemize}
\item Midelfart/Midtgaard: Adaptive templates \cite{Midelfart2010}
\item Simulation of sidescan images (Bell thesis \cite{Bell1997}, pre-work by V.Blake \cite{Blake1993})
\item NURC SIGMAS+ (Coiras GPU article at OCEANS 2009 \cite{Coiras2009a}, Coiras on SIGMAS maths \cite{Coiras2009b})
\end{itemize}


\section{Methods}


\subsection{Setting up the scene}

Before we can do anything with OpenGL we first have to set up the scene. This involves loading the models and then placing them where we want them with the proper orientation. 

Our models are stored in regular 3D model files. The simulator loads these with the Open Asset Import library (\texttt{assimp}). This is a portable open source library that supports loading a wide range of 3D model formats in a uniform manner. It handles the intricacies of the various model formats and outputs the model as a node tree. Each node contains data such as vertices, facets and textures, and since they are formatted in an OpenGL friendly way we simply iterate through the nodes and load it.

The next step is to scale, orient and position the objects in the scene, and finally project this scene onto an image plane. This is achieved by applying a set of transformations to each vertex in the model. In OpenGL, the set of transformations are usually formalized like this:
\begin{align*}
\bmat{x\\y\\0\\1}_\text{image} = \P \cdot \V \cdot \M \cdot \bmat{x\\y\\z\\1}_\text{model},
\end{align*}
where is the loaded model, $\bmat{\cdot}_\text{image}$ is the output image, and $\M$, $\V$ and $\P$ are transformation matrices. The first transform we apply is the model matrix $\M$, which scales and rotates the model in its local coordinate system. Then we map this model into world coordinates using the view matrix $\V$. This involves translating the model to its respectful place in the world, and then moving it into the view of the camera. Finally the relevant part of the scene is projected onto the image plane with a projection matrix $\P$. 

\subsection{OpenGL Rendering}

To produce the sonar templates we assume a rough, isotropic surface that reflects energy equally in all directions. This permits us to use a Lambertian scattering model where the backscatter intensity depends only on the incidence angle~\cite{Zhang1999}. It does not consider observation angle or sound frequency\todo{is this unproblematic?}.   

When our models loaded into the OpenGL scene we are ready to render. 



It is also possible to compute the sonar image using the OpenGL ``blend'' feature.


\section{Results}

\newlength\imgspacing\setlength\imgspacing{.5cm}

\begin{figure}[tp]\centering%
\begin{subfigure}[b]{.5\linewidth-\imgspacing/2}
\graphicsAI[drawing,width=\linewidth]{gfx/data_cylinder_submerged.svg}%
\caption{SAS image of a cylinder with 2.6\;m length and 0.53\;m radius.}\label{data_sylinder}%
\end{subfigure}\hspace{\imgspacing}%
\begin{subfigure}[b]{.5\linewidth-\imgspacing/2}
\graphicsAI[drawing,width=\linewidth]{gfx/data_cylinder_submerged_bw.svg}%
\caption{SAS image of a cylinder with 2.6\;m length and 0.53\;m radius.}\label{data_cylinder_bw}%
\end{subfigure} 
\end{figure}

\begin{figure}[tp]\centering%
\begin{subfigure}[b]{.5\linewidth-\imgspacing/2}
\graphicsAI[drawing,width=\linewidth]{gfx/sim_cylinder_submerged.svg}%
\caption{Simulated cylinder.}\label{sim_cylinder}%
\end{subfigure}\hspace{\imgspacing}%
\begin{subfigure}[b]{.5\linewidth-\imgspacing/2}
\graphicsAI[drawing,width=\linewidth]{gfx/sim_cylinder_submerged_bw.svg}%
\caption{Simulated cylinder.}\label{sim_cylinder_bw}%
\end{subfigure} 
\end{figure}

\begin{figure}[tp]\centering%
\begin{subfigure}[b]{.5\linewidth-\imgspacing/2}
\graphicsAI[drawing,width=\linewidth]{gfx/sim_cylinder_submerged_adaptive.svg}%
\caption{Simulated cylinder (adaptive).}\label{sim_cylinder}%
\end{subfigure}\hspace{\imgspacing}%
\begin{subfigure}[b]{.5\linewidth-\imgspacing/2}
\graphicsAI[drawing,width=\linewidth]{gfx/sim_cylinder_submerged_adaptive_bw.svg}%
\caption{Simulated cylinder (adaptive).}\label{sim_cylinder_bw}%
\end{subfigure} 
\end{figure}

\newcommand\cdesc[2]{{\raggedright\setlength\fboxsep{0pt}
\fbox{\colorbox[HTML]{#1}{\vrule height8.5pt depth3.5pt width0pt\hspace{.5cm}}}\ \ #2\\}}
\begin{figure}[tp]\centering%
\begin{subfigure}[b]{.5\linewidth-\imgspacing/2}
\graphicsAI[drawing,width=\linewidth]{gfx/overlay_cylinder_submerged.svg}%
\caption{Overlay.}\label{overlay_cylinder}%
\end{subfigure}\hspace{\imgspacing}%
\begin{subfigure}[b]{.5\linewidth-\imgspacing/2}
\graphicsAI[width=\linewidth]{gfx/overlay_cylinder_submerged_adaptive.svg}%
\caption{Overlay (adaptive).}\label{overlay_cylinder_adaptive}%
\end{subfigure}
\hfill\parbox{\linewidth-30pt}{
\cdesc{800000}{Pixels in both template and image highlight segments}
\cdesc{FF1000}{Pixels only in template highlight segment}
\cdesc{FFEB00}{Pixels only in image highlight segment}
\cdesc{83FF7C}{Background pixels}
\cdesc{00EFFF}{Pixels only in the image shadow segment}
\cdesc{00A7FF}{Pixels in the image segment, but in template shadow}
\cdesc{0014FF}{Pixels only in the image template highlight segment}
\cdesc{000083}{Pixels in both the template and image shadow segments}}
\end{figure}





% Ligger nesten med kortsiden til.
% Bilde og templat segmenteres hver for seg.
% Ikke adaptiv - darlig. Feil lengde, skyggen for kort (synket). Orientering feil. Feil spacing. 
% Adaptiv - malt objekt. Passer bra. 
% 
% Videre:
% - Mer testing
% - Robustifisering
% - Tuning

% 
\newpage
\bibliography{references}
% \bibliography{../../../library.bib}
% 
% % If biblatex (file is loaded in preamble):
% % \printreferences

\end{document}

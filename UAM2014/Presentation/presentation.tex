\documentclass[
    beamer                                       % Document type (non-standard)
%  , handout
%   , xelatex                                      % Use the XeLaTeX compiler
%  , movie
  ,table,dvipsnames,svgnames
]{common/mytemplate}

\pdfpageattr {/Group << /S /Transparency /I true /CS /DeviceRGB>>}

% \mode<handout>{\setbeamercolor{background canvas}{bg=black!5}}

% \RequirePackage[table,dvipsnames,svgnames]{xcolor}
\definecolor{tabBlue}{HTML}{AACCFF}
\usepackage{template/beamerthemeUiO}
% \usepackage{common/oxygentheme}
% \setbeameroption{show only notes}
% \setbeameroption{notes on second screen=left}

% \newcommand\remotesource{http://www.joinge.net/compet}

\title{\newline\newline A GPU Sonar Simulator for Automatic Target Recognition}
\subtitle{}
\author[]{\textbf{Jo Inge Buskenes}$^{\text{a}}$, Herman Midelfart$^{\text{b}}$, \O{}ivind Midtgaard$^{\text{b}}$}
\date[Jo Inge Buskenes at UA2014, Rhodes, June\ 2014]{Rhodes, Greece, 2014}
\institute[Dept.\ of Informatics, University of Oslo]{\bf$^{\text{a}}$Department of Informatics, University of Oslo, Norway\\
\bf$^{\text{b}}$Norwegian Defence Research Establishment (FFI), Norway}
% \date{Sist revidert: 27.01.2009}

\definecolor{LightBlue}{rgb}{0.3,0.3,1}
\definecolor{DarkBlue}{rgb}{0.2,0.2,0.7}
\setbeamercolor{block title}{bg=DarkBlue,fg=white}%bg=background, fg= foreground
\setbeamercolor{block body}{bg=gray!20,fg=black}%bg=background, fg= foreground
\setbeamertemplate{blocks}[rounded]

\RequirePackage[latin1]{inputenc}%              % Set input encoding (optionally latin1) utf8
\RequirePackage[T1]{fontenc}%                 % Set font encoding

\begin{document}

\begin{frame}
\vspace*{2\baselineskip}
  \titlepage
\end{frame}
\note{\null{}}

% \AtBeginSection[]
% {
%   \frame<handout:0>
%   {
%     \frametitle{Agenda}
%     \tableofcontents %[currentsection,hideallsubsections]
%   }
% }

% \AtBeginSubsection[]
% {
%   \frame<handout:0>
%   {
%     \frametitle{Agenda}
%     \tableofcontents[sectionstyle=show/hide,subsectionstyle=show/shaded/hide]
%   }
% }

\newcommand<>{\highlighton}[1]{%
  \alt#2{\structure{#1}}{{#1}}
}

\newcommand{\icon}[1]{\pgfimage[height=1em]{#1}}


%%%%%%%%%%%%%%%%%%%%%%%%%%%%%%%%%%%%%%%%%
%%%%%%%%%% Content starts here %%%%%%%%%%
%%%%%%%%%%%%%%%%%%%%%%%%%%%%%%%%%%%%%%%%%

% {



% \section{Case study}
% \subsection{Realtime sectorcan imaging}
% 
% {
% \setbeamertemplate{background}{\graphicsAI[width=\paperwidth]{gfx/use_case_sectorscan.svg}}%
% \begin{frame}
% \vspace{-80pt}\hspace{-25pt}
% \frametitle{\color{white}Realtime requirements?}
% \framesubtitle{\vspace{-10pt}\color{white}Sectorscan imaging}
% \vspace{100pt}\ 
% % \begin{figure}[H]
% % \begin{narrow}{-1.6cm}{-1.6cm}
% % \graphicsAI<1>[drawing,width=\linewidth]{gfx/use_case_sectorscan3.svg}
% % \end{narrow}
% % \end{figure}   
% \end{frame}
% }



% \renewcommand{\frametitle}[2]{{\vspace*{10pt}\bf\Large #1\par}}
% \begin{frame}
% \frametitle{Context}
% \framesubtitle{Active Sonar}
% \vspace{-25pt}
% \begin{figure}[H]
% \begin{narrow}{-0.5cm}{0pt}
% \hspace{-10pt}\graphicsAI<1>[drawing,width=1.05\linewidth]{gfx/SonarPrinciple.svg}
% \end{narrow}
% \end{figure}
% % \begin{itemize}
% % \item Each pixel value is estimated by focusing the receiver on that point.
% % \end{itemize}
% \end{frame}

% \begin{frame}
% \frametitle{Context}
% \framesubtitle{Imaging techniques}
% \vspace{-10pt}
% \begin{figure}[H]
% \begin{narrow}{-0.5cm}{0pt}
% \hspace{-10pt}\graphicsAI<1>[drawing,width=1.05\linewidth]{gfx/imaging_concepts.png}
% \end{narrow}
% \end{figure}
% A phased array can be used in several imaging modes:
% \begin{itemize}
% \item \emph{Sector scan}: Image computed from a single ping
% \item \emph{Sidescan}: Image created by stacking images from several subsequent pings
% \item \emph{Synthetic aperture sonar}: Image computed from overlapping pings
% \end{itemize}
% % \begin{block}{Proposed method: Generic, can be used in all modes}
% % \vspace{-25pt}
% % \end{block}
% \end{frame}
% }

% \section{}
% \begin{frame}
%   \frametitle{Agenda}
%   \tableofcontents%[%section=1,
%                    %hidesubsections]
% \end{frame}

\renewcommand{\frametitle}[2]{{\vspace*{10pt}\bf\Large #1\par}}




% \begin{frame}
% \frametitle{Acknowledgements}
% \framesubtitle{}
% \vspace{-3pt}
% \begin{itemize}
% \item Kongsberg Maritime.
% \item FFI guys.
% \end{itemize}
% \end{frame}

% General system, may be applied to several different datasets

% \section{Introduction}
% 
% 
% \begin{frame}
% \frametitle{What we do}
% \framesubtitle{}
% \vspace{-10pt}
% \begin{figure}[H]
% \begin{narrow}{-0.5cm}{0pt}
% \hspace{-10pt}\graphicsAI<1>[drawing,width=1.05\linewidth]{gfx/applications.svg}
% \end{narrow}
% \end{figure}
% % A phased array can be used in several imaging modes:
% % \begin{itemize}
% % \item \emph{Sector scan}: Image computed from a single ping
% % \item \emph{Sidescan}: Image created by stacking images from several subsequent pings
% % \item \emph{Synthetic aperture sonar}: Image computed from overlapping pings
% % \end{itemize}
% % % \begin{block}{Proposed method: Generic, can be used in all modes}
% % % \vspace{-25pt}
% % % \end{block}
% \end{frame}
% 

% {
% \setbeamertemplate{background}{%
% \graphicsAI[drawing,width=\paperwidth]{gfx/layover.svg}}%
{
\setbeamertemplate{background}{\graphicsAI[drawing,width=\paperwidth]{gfx/background.svg}}%
\begin{frame}
\frametitle{Simulator features}
\framesubtitle{}
\begin{figure}[H]
\begin{narrow}{-0.6cm}{-0.6cm}
\graphicsAI<1>[drawing,width=\linewidth]{gfx/specs.svg}
\end{narrow}
\end{figure}                                                                                                                                                                                                                                                                                                   
\end{frame}
}
% }

{
\setbeamertemplate{background}{\graphicsAI[drawing,width=\paperwidth]{gfx/background.svg}}%
\begin{frame}
\frametitle{\hspace{-.5cm}Synthetic Aperture Sonar (SAS)}
\framesubtitle{}
\vspace{-10pt}
\begin{figure}[H]
\begin{narrow}{.1cm}{-0.6cm}
\flushright\hspace{-10pt}\graphicsAI<1>[drawing,width=.92\linewidth]{gfx/rock_formation.svg}
\end{narrow}
\end{figure}
\vspace{-10pt}
\begin{narrow}{-0.5cm}{-0.5cm}
\begin{itemize}\small
\item SAS image of a 25 x 25m wide by 5m high rock formation at 70m depth.
\item Theoretical resolution of 3-4cm throughout the image.
\item Data from the HUGIN AUV carrying the HISAS1030 interferometric SAS.
\end{itemize}
\end{narrow}
\end{frame}
}

{
\setbeamertemplate{background}{\graphicsAI[drawing,width=\paperwidth]{gfx/optics_sas_compared.svg}}%
\begin{frame}
\frametitle{SAS vs optics}
\framesubtitle{}
\vspace{8cm}
% \begin{figure}[H]
% \begin{narrow}{-1.6cm}{-1.6cm}
% \graphicsAI<1>[drawing,width=\linewidth]{gfx/use_case_sectorscan3.svg}
% \end{narrow}
% \end{figure}   
\end{frame}
}

{
\setbeamertemplate{background}{%
\graphicsAI[drawing,width=\paperwidth]{gfx/layover.svg}}%
\begin{frame}
\frametitle{Simulating a ranged sonar}
\framesubtitle{Simple when only considering direct scatter}
\vspace{8cm}
% \mbox{}\hfill
% \begin{figure}[H]
% \begin{narrow}{-0.9cm}{-0.9cm}
% \graphicsAI<1>[drawing,width=\linewidth]{gfx/layover.svg}
% \end{narrow}
% \end{figure}                                                                                                                                                                                                                                                                                                   
\end{frame}
}


{
\setbeamertemplate{background}{%
\graphicsAI[drawing,width=\paperwidth]{gfx/lambert.svg}}%
\begin{frame}
\frametitle{Simulating a ranged sonar}
\framesubtitle{Scattering strength given by Lambert's law}
\vspace{8cm}                                                                                                                                                                                                                                                                                               
\end{frame}
}


{
\setbeamertemplate{background}{%
\graphicsAI[drawing,width=\paperwidth]{gfx/visible_pixels.svg}}%
\begin{frame}
\frametitle{Simulating a ranged sonar}
\framesubtitle{How to deal with ``hidden pixels''?}
\vspace{8cm}                                                                                                                                                                                                                                                                                               
\end{frame}
}


{
\setbeamertemplate{background}{%
\graphicsAI[drawing,width=\paperwidth]{gfx/scene_photograph.svg}}%
\begin{frame}
\frametitle{Simulation}
\framesubtitle{OpenGL!}
\vspace{8cm}                                                                                                                                                                                                                                                                                               
\end{frame}
}




% % 
% \begin{frame}
% \makeatletter%
% % \special{pdf: put @thispage <</Group << /S /Transparency /I true /CS /DeviceRGB>> >>}%
% \makeatother%
% \frametitle{GPUs for image reconstruction}
% \framesubtitle{}
% \vspace{-3pt}
% \begin{figure}[H]
% \begin{narrow}{0cm}{0cm}
% \graphicsAI<1>[drawing,width=\linewidth]{gfx/gpus.svg}%{../Docs/GeiloSrc/Figs/das.pdf}
% \end{narrow}
% \end{figure}
% \begin{itemize}
% \item CPUs: Optimized for running \emph{a few} potentially complex threads, that may have \emph{complex} data dependencies.
% \item GPUs: Optimized for running \emph{hundreds} of light-weight threads, with \emph{simple} data dependencies. Ideal for graphics!
% \end{itemize}
% \end{frame}

% 
% \section{Implementation on a GPU}
% 
\begin{frame}
\frametitle{Implementation}
\framesubtitle{Using OpenGL and OpenCL}
\vspace{-15pt}
\begin{figure}[H]
\begin{narrow}{-0.9cm}{-0.9cm}
\graphicsAI<1>[drawing,width=\linewidth]{gfx/simulator.svg}%{../Docs/GeiloSrc/Figs/das.pdf}
\end{narrow}
\end{figure}
\end{frame}


{
\setbeamertemplate{background}{%
\graphicsAI[drawing,width=\paperwidth]{gfx/viewer_manta_size.svg}}%
\begin{frame}
\frametitle{Results}
\framesubtitle{Looking at the viewer}
\vspace{8cm}
% \begin{narrow}{-.5cm}{-.5cm}
% \vspace{-2cm}
% \graphicsAI<1>[drawing,width=1\linewidth]{gfx/viewer_manta_straight.svg}
% \end{narrow}
\end{frame}
}

{
\setbeamertemplate{background}{%
\graphicsAI[drawing,width=\paperwidth]{gfx/viewer_manta_straight.svg}}%
\begin{frame}
\frametitle{Results}
\framesubtitle{Looking at the viewer}
\vspace{8cm}
% \begin{narrow}{-.5cm}{-.5cm}
% \vspace{-2cm}
% \graphicsAI<1>[drawing,width=1\linewidth]{gfx/viewer_manta_straight.svg}
% \end{narrow}
\end{frame}
}

{
\setbeamertemplate{background}{%
\graphicsAI[drawing,width=\paperwidth]{gfx/viewer_manta_positive_angle.svg}}%
\begin{frame}
\frametitle{Results}
\framesubtitle{Looking at the viewer}
\vspace{8cm}
% \begin{narrow}{-.5cm}{-.5cm}
% \vspace{-2cm}
% \graphicsAI<1>[drawing,width=1\linewidth]{gfx/viewer_manta_straight.svg}
% \end{narrow}
\end{frame}
}

{
\setbeamertemplate{background}{%
\graphicsAI[drawing,width=\paperwidth]{gfx/viewer_manta_negative_angle.svg}}%
\begin{frame}
\frametitle{Results}
\framesubtitle{Looking at the viewer}
\vspace{8cm}
\end{frame}
}

{
\setbeamertemplate{background}{%
\graphicsAI[drawing,width=\paperwidth]{gfx/viewer_ship.svg}}%
\begin{frame}
\frametitle{Results}
\framesubtitle{A full ship!!!}
\vspace{8cm}
\end{frame}
}

{
\setbeamertemplate{background}{%
\graphicsAI[drawing,width=\paperwidth]{gfx/viewer_ship_multipath.svg}}%
\begin{frame}
\frametitle{Results}
\framesubtitle{A full ship!!!}
\vspace{8cm}
\end{frame}
}

{
\setbeamertemplate{background}{%
\graphicsAI[drawing,width=\paperwidth]{gfx/viewer_ship_rough.svg}}%
\begin{frame}
\frametitle{Results}
\framesubtitle{A full ship!!!}
\vspace{8cm}
% \begin{narrow}{-.5cm}{-.5cm}
% \vspace{-2cm}
% \graphicsAI<1>[drawing,width=1\linewidth]{gfx/viewer_manta_straight.svg}
% \end{narrow}
\end{frame}
}

{
\setbeamertemplate{background}{%
\graphicsAI[drawing,width=\paperwidth]{gfx/viewer_ship_holmengraa.svg}}%
\begin{frame}
\frametitle{Results}
\framesubtitle{Compared to SSS}
\vspace{8cm}
% \begin{narrow}{-.5cm}{-.5cm}
% \vspace{-2cm}
% \graphicsAI<1>[drawing,width=1\linewidth]{gfx/viewer_manta_straight.svg}
% \end{narrow}
\end{frame}
}

{
\setbeamertemplate{background}{%
\graphicsAI[drawing,width=\paperwidth]{gfx/cylinder_submerged_hisas.svg}}%
\begin{frame}
\frametitle{Results}
\framesubtitle{HISAS image of a cylinder}
\vspace{8cm}
% \begin{narrow}{-.5cm}{-.5cm}
% \vspace{-2cm}
% \graphicsAI<1>[drawing,width=1\linewidth]{gfx/viewer_manta_straight.svg}
% \end{narrow}
\end{frame}
}

{
\setbeamertemplate{background}{\graphicsAI[drawing,width=\paperwidth]{gfx/background.svg}}%
\begin{frame}
\frametitle{Results}
\framesubtitle{SAS and template image compared}
\begin{narrow}{-.9cm}{-.9cm}
\begin{minipage}[t]{0.499\linewidth}\centering
\graphicsAI<1>[drawing,width=1\linewidth]{gfx/data_cylinder_submerged.svg}\\%
SAS image
\end{minipage}%
\begin{minipage}[t]{0.499\linewidth}\centering
\graphicsAI<1>[drawing,width=1\linewidth]{gfx/sim_cylinder_submerged_adaptive.svg}\\%
Simulated template
\end{minipage}
\end{narrow}
\end{frame}
}

% --initial work
% --compare to other simulator
% --journal: 
% 
% Normal SAS image compared.
% 
% - Most important for ATR: Highlight and shadow.
% - Probably(?) not needed: Noise, speckle, beampattern, multipath, penetration
% - OpenGL and OpenCL explained.
%   * OpenGL to detect visible pixels. Not always trivial.
%   * Not ray tracing
%   * Second step ray tracer. Can drop first reflection.
%   * Monostatic target strength formula Kerr.
% - Slide on perspective / orthographic. Sectorscan / sidescan / SAS.
% - Lambertian scattering?
% - Fix fonts simulator.
% - High angle, low angle ship.
% % - Elastic and non-elastic materials.
% - Submarine
% - 


\newcommand\cdesc[2]{{\raggedright\setlength\fboxsep{0pt}
\fbox{\colorbox[HTML]{#1}{\vrule height8.5pt depth3.5pt width0pt\hspace{.5cm}}}\ \ \parbox[t]{\linewidth-1cm}{#2}\\[.1cm]}}
% \begin{frame}
% \frametitle{Results}
% \framesubtitle{Overlay of segmented SAS image and segmented template image}
% \begin{narrow}{-.9cm}{-.9cm}
% \begin{minipage}[c]{0.62\linewidth}\centering
% \begin{figure}[H]
% \graphicsAI<1>[drawing,width=\linewidth]{gfx/overlay_cylinder_submerged.svg}
% \end{figure}%
% \end{minipage}%
% \mbox{}\hfill%
% \begin{minipage}[c]{.35\linewidth}\footnotesize%
% \begin{figure}
% \cdesc{800000}{\bf Template highlight and image highlight}
% \cdesc{FF1000}{Template highlight only}
% \cdesc{FFEB00}{Image highlight only}
% \cdesc{83FF7C}{Background pixels}
% \cdesc{000083}{\bf Template shadow and\newline image shadow}
% \cdesc{0014FF}{Template shadow only}
% \cdesc{00EFFF}{Image shadow only}
% \cdesc{00A7FF}{Template shadow and image highlight}
% \end{figure}
% \end{minipage}
% \begin{minipage}[t]{0.62\linewidth}\centering
% Best template out of a predefined set%
% \end{minipage}%
% \mbox{}\hfill%
% \begin{minipage}[t]{.35\linewidth}\centering%
% Overlay color description
% \end{minipage}
% \end{narrow}
% \end{frame}


{
\setbeamertemplate{background}{\graphicsAI[drawing,width=\paperwidth]{gfx/background.svg}}%
\begin{frame}
\frametitle{Results}
\framesubtitle{Overlay of segmented SAS image and segmented template image}
\begin{narrow}{-.9cm}{-.9cm}
\begin{minipage}[c]{0.62\linewidth}\centering
\begin{figure}[H]
\graphicsAI<1>[drawing,width=\linewidth]{gfx/overlay_cylinder_submerged_adaptive.svg}
\end{figure}%
\end{minipage}%
\mbox{}\hfill%
\begin{minipage}[c]{.35\linewidth}\footnotesize%
\begin{figure}
\cdesc{800000}{\bf Template highlight and image highlight}
\cdesc{FF1000}{Template highlight only}
\cdesc{FFEB00}{Image highlight only}
\cdesc{83FF7C}{Background pixels}
\cdesc{000083}{\bf Template shadow and\newline image shadow}
\cdesc{0014FF}{Template shadow only}
\cdesc{00EFFF}{Image shadow only}
\cdesc{00A7FF}{Template shadow and image highlight}
\end{figure}
\end{minipage}
\begin{minipage}[t]{0.62\linewidth}\centering
Adaptive template%
\end{minipage}%
\mbox{}\hfill%
\begin{minipage}[t]{.35\linewidth}\centering%
Overlay color description
\end{minipage}
\end{narrow}
\end{frame}
}


{
\setbeamertemplate{background}{\graphicsAI[width=\paperwidth]{gfx/improvements.svg}}%
\begin{frame}
\vspace{-10pt}
\frametitle{Future work}
\framesubtitle{}
\vspace{8cm}
\end{frame}
}

{
\setbeamertemplate{background}{\graphicsAI[width=\paperwidth]{gfx/AUVenv.png}}%
\begin{frame}
\vspace{-25pt}\frametitle{\color{white}Conclusions}
\framesubtitle{}
\vspace{-5pt}\color{white}
\begin{itemize}
\item \color{white}Sonar templates generated @ 100+ FPS on a regular GPU, using
\begin{itemize}
\item \color{white}OpenGL to capture a ``camera'' version the scene as seen from the sonar, and
\item \color{white}OpenCL to convert the OpenGL image with corresponding depth information to a ranged sonar-like image.
\end{itemize}
\item \color{white}Image quality
\begin{itemize}
\item \color{white}Shape and size seems ok.
\item \color{white}Fishy stuff going on in the transition of highlight/shadow
\end{itemize}
\item \color{white}What to improve?
\end{itemize}
\vfill
\textbf{\color{black}Acknowledgements}:
\begin{itemize}
\item Thanks to Kongsberg Maritime for making great hardware and to FFI for the assignment of developing the simulator.
\end{itemize}
\end{frame}
}


\end{document}


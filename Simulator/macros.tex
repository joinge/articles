%\definecolor{cBase}{HTML}{375e97}
% \colorlet{cBase}{yellow}
\colorlet{cBase}{tabBlue}
%\definecolor{cBaseMark}{HTML}{3f681c}
\colorlet{cBaseMark}{Green}
\colorlet{cFrame}{Green}
% \definecolor{cBasis}{HTML}{ff662f}
\definecolor{cBasis}{HTML}{ff662f}
\definecolor{cPoint}{HTML}{3492c2}
% \definecolor{cPoint}{HTML}{2492c2}
% \definecolor{cDarkArrow}{HTML}{ff662f}
% \definecolor{cBrightArrow}{HTML}{3492c2}
\colorlet{cDarkArrow}{gray}
\colorlet{cBrightArrow}{gray}

\newcommand\supcarry[1]{{\color{cPoint}#1}}
\newcommand\subcarry[1]{{\color{cBasis}#1}}




%\newcommand{\tikzmark}[1]{\tikz[remember picture]\coordinate(#1){(-.35em,-.545ex)};}
\providecommand\tikzmark{}\renewcommand{\tikzmark}[2]{\tikz[remember picture,baseline=(#1.base)]\node[shape=rectangle,inner sep=0pt](#1){#2};}
%\newcommand{\tikzmark}[2][]{\tikz[remember picture,overlay]\coordinate[#1](#2);}
\providecommand*\circled{}\renewcommand*\circled[1]{\tikz[baseline=(char.base)]{
            \node[shape=circle,draw,inner sep=2.5pt] (char) {#1};}}
            
\tikzset{-{to[length=.5em,width=.5em]}}

\newcommand\step[1]{\tikz[baseline=(char.base)]{%
            \node[shape=circle,draw,color=cBaseMark,shade,top color=black!5!white,bottom color=black!15!white,inner sep=.75pt] (char) {\color{cBaseMark}#1};}}
\newcommand\figstep[2]{\step{#1}\mbox{}\hfill\parbox[t]{.97\linewidth}{#2}}

\ifPhdDoc\else
\makeatletter
\let\@oldseccntformat\@seccntformat
\DeclareRobustCommand{\@seccntformat}[1]{%
  \def\temp@@a{#1}%
  \def\temp@@b{subsubsection}%
  \ifx\temp@@a\temp@@b
  \else
  \@oldseccntformat{#1}
  \fi
}
\let\oldsubsubsection\subsubsection
\renewcommand\subsubsection{\@startsection{subsubsection}{3}{\parindent\normalfont\arabic{subsubsection}\enskip}{0ex plus 0.1ex minus 0.1ex}{0ex}{\normalfont\normalsize\itshape}}%
\makeatother
\fi

% The below line commands ensure proper fill color along with the booktabs package
% \specialrule[width]{abovespace}{belowspace}
\colorlet{tableheadcolor}{cBase}
\newlength\arulesep\setlength\arulesep{\aboverulesep}
\newlength\brulesep\setlength\brulesep{\belowrulesep}
\setlength{\aboverulesep}{0pt}
\setlength{\belowrulesep}{0pt}  
\newcommand\topline{\arrayrulecolor{black}\specialrule{\heavyrulewidth}{\arulesep}{0pt}%
            \arrayrulecolor{tableheadcolor}\specialrule{\brulesep}{0pt}{0pt}%
            \arrayrulecolor{black}}
\newcommand\midline{\arrayrulecolor{tableheadcolor}\specialrule{\arulesep}{0pt}{0pt}%
            \arrayrulecolor{black}\specialrule{\lightrulewidth}{0pt}{0pt}%
            \arrayrulecolor{white}\specialrule{\brulesep}{0pt}{0pt}%
            \arrayrulecolor{black}}
\newcommand\cmidline[2][]{\arrayrulecolor{tableheadcolor}\cmidrule[\arulesep]{#2}%
                        \arrayrulecolor{black}\cmidrule[\lightrulewidth]{#2}%
                        \arrayrulecolor{tableheadcolor}\cmidrule[\brulesep]{#2}%
                        \arrayrulecolor{black}}

\newlength\trulesep{}\setlength  \trulesep{\arulesep}
                     \addtolength\trulesep{\lightrulewidth}
                     \addtolength\trulesep{\brulesep}
\newcommand\cmidwrap[1]{%\arrayrulecolor{tableheadcolor}\specialrule{\trulesep}{0pt}{-\trulesep}%
                        \arrayrulecolor{tableheadcolor}\specialrule{\arulesep}{0pt}{0pt}%
                        \arrayrulecolor{tableheadcolor}\specialrule{\lightrulewidth}{0pt}{-\lightrulewidth}%
%                        \arrayrulecolor{black}\cmidrule[\lightrulewidth]{#2}%
%                        \arrayrulecolor{tableheadcolor}\cmidrule[\brulesep]{#2}%
                        \arrayrulecolor{black}#1
                        \arrayrulecolor{tableheadcolor}\specialrule{\brulesep}{0pt}{0pt}%
%            \setlength{\aboverulesep}{10pt}
%            \setlength{\belowrulesep}{10pt}
%            #1
            }


%%%%%%%%%%%%%%%%%%%%%%%%%%%%%%%                  ~~~~~~~~~~~~~~~~~~~~~~~~~~~~~~~~~~~~~~~~~~~~~~~~~~
%% IEEE ''APPROVED'' PACKAGES %
%%%%%%%%%%%%%%%%%%%%%%%%%%%%%%%                  ~~~~~~~~~~~~~~~~~~~~~~~~~~~~~~~~~~~~~~~~~~~~~~~~~~
%
%\usepackage{cite}
%
%\ifCLASSINFOpdf
%   \usepackage[dvips]{graphicx}                 % Might not work. Use 'latex' instead of 
%   \ifFlatArchive\else                          % 'pdflatex'
%      \graphicspath{./gfx/}
%   \fi
%\else
%   \usepackage[dvips]{graphicx}
%   \ifFlatArchive\else
%      \graphicspath{./gfx/}
%   \fi
%\fi
%
%\RequirePackage[table,dvipsnames,svgnames]{xcolor}
%
%\usepackage[cmex10]{amsmath}                    % cmex10 option to be IEEE explore compliant
%\interdisplaylinepenalty=2500                   % Allows multiline equations to be broken
%
%% \RequirePackage{amssymb}
%
%\RequirePackage{array}
%
%\ifCLASSOPTIONcompsoc
%   \usepackage[caption=false,font=normalsize,labelfont=sf,textfont=sf]{subfig}
%\else
%   \usepackage[caption=false,font=footnotesize]{subfig}
%\fi
%
%% \usepackage{caption}
%% \usepackage{subcaption}
%\usepackage{color}
%\usepackage{calc}
%\usepackage{fp}
%
%% 
%% \ifCLASSOPTIONcaptionsoff                       % IEEE promoted hack to turn off captions from the 
%%    \let\MYorigsubfloat\subfloat                 % subfloat package should the captionsoff option
%%    \renewcommand{\subfloat}[2][\relax]{\MYorigsubfloat[]{#2}} % be specified.
%% \fi
%
%\ifFloatAtEnd
%\ifCLASSOPTIONcaptionsoff                       % Places float at the end of the document when the
%  \usepackage[nomarkers]{endfloat}              % captionsoff options is specified to IEEEtrans.cls
%  \let\MYoriglatexcaption\caption               % (PeerReview mode)
%  \renewcommand{\caption}[2][\relax]{\MYoriglatexcaption[#2]{#2}}
%\fi
%\fi
%
%% \usepackage{fixltx2e}                           % Fix some twocolumn float problems
%
%% \usepackage{stfloats}                          % Allows: \begin{figure*}[!b]
%                                                % (double column figures on top/bottom)
%
%\usepackage{url}                                % Support for handling and breaking URLs
%
%% NOTE: PDF hyperlink and bookmark features are not required in IEEE
%%       papers and their use requires extra complexity and work.
%% \newcommand\MYhyperrefoptions{bookmarks=true,bookmarksnumbered=true,
%% pdfpagemode={UseOutlines},plainpages=false,pdfpagelabels=true,
%% colorlinks=true,linkcolor={black},citecolor={black},urlcolor={black},
%% pdftitle={Optimal Window Design for the Low Complexity Adaptive Beamformer in Active Sonar Imaging},
%% pdfsubject={},
%% pdfauthor={Jo Inge Buskenes},
%% pdfkeywords={adaptive beamforming, beamforming, complexity, sonar, active}}%
%\ifCLASSINFOpdf
%%    \usepackage[\MYhyperrefoptions,pdftex]{hyperref}
%   \usepackage[pdftex,
%    final,                %Keep hyperlink stuff even when in draft mode
%    breaklinks=true,      %Break links when necessary
%    linktocpage=true,     %Enable link to page?
%    linkcolor=black,  %Colour of links to labels within document
%    citecolor=black,      %Colour of links to the biliography
%    filecolor=red,        %Colour of links to local files
%    pagecolor=black,        %Colour of links to other pages withing document
%    urlcolor=Blue,        %Colour of the links to external URLs
%    colorlinks=true,      %??
%    plainpages=false,     %Store roman/arabic numbering differently to avoid
%    bookmarksnumbered ]{hyperref}
%% 
%% \else
%%    \usepackage[\MYhyperrefoptions,breaklinks=true,dvips]{hyperref}%
%\fi
%%    \usepackage{breakurl}                        % Allows 'dvips' driver to break links
%
%%%%%%%%%%%%%%%%%%%%%%%%                         ~~~~~~~~~~~~~~~~~~~~~~~~~~~~~~~~~~~~~~~~~~~~~~~~~~
%% ADDITIONAL PACKAGES %       
%%%%%%%%%%%%%%%%%%%%%%%%                         ~~~~~~~~~~~~~~~~~~~~~~~~~~~~~~~~~~~~~~~~~~~~~~~~~~
%
%% \ifPeerReview
%% \let\OldIncludegraphics{\includegraphics}
%% \usepackage{letltxmacro}
%% \LetLtxMacro{\OldIncludegrsaphics}{\includegraphics}
%% \renewcommand{\includegraphics}[2][]{\OldIncludegraphics[width=\linewidth, #1]{#2}}
%% \fi
%
%% % \makeatletter
%% % \def\maxwidth{\ifdim\Gin@nat@width>\linewidth\linewidth
%% % \else\Gin@nat@width\fi}
%% % \makeatother
%% % \let\Oldincludegraphics\includegraphics
%% % \renewcommand{\includegraphics}[1]{\Oldincludegraphics[width=\maxwidth]{#1}}
%
%% \usepackage[maxfloats=40]{morefloats}

% \setcounter{todoidx}
\ifOPTtodos\else
\ifTODO
% \usepackage{marginnote}
   \newcounter{todoidx}
%   \vbadness=99999	
   \definecolor{todobackground}{rgb}{0.95,0.95,0.95}
   \setlength\marginparsep{1pt}
   \setlength\marginparwidth{35pt}
   \newlength\marginparwidthsmall
   \setlength\marginparwidthsmall{\marginparwidth}
   \addtolength\marginparwidthsmall{-7pt}
   \newcommand\todo[1]{%
      \addtocounter{todoidx}{1}%
      {\color{Red}\bf(\thetodoidx{})}%%\fbox{\bf\thetodoidx{}}}%
      \marginpar{%
         {\vspace*{-10pt}\color{Red}\fbox{\bf\thetodoidx{}}}\\%
         \fcolorbox{red}{todobackground}{\parbox{\marginparwidthsmall}{\raggedright\scriptsize #1}}}}

   \newcommand\todopar[1]{\fcolorbox{red}{white}{\parbox{0.97\linewidth}{#1}}}
\else
%    \usepackage[disable]{./todonotes}
      \providecommand\todo[1]{}
      \renewcommand\todo[1]{}
\fi
\fi
%
%\newenvironment{narrow}[2]{%
%\begin{list}{}{%
%\setlength{\topsep}{0pt}%
%\setlength{\leftmargin}{#1}%
%\setlength{\rightmargin}{#2}%
%\setlength{\listparindent}{\parindent}%
%\setlength{\itemindent}{\parindent}%
%\setlength{\parsep}{\parskip}}%
%\item[]}{\end{list}}
%
%% \usepackage{float}
%
%\ifOnlineColor
%   \definecolor{tabBlue}{HTML}{AACCFF}
%\else
%   \definecolor{tabBlue}{HTML}{CCCCCC}
%\fi
%
%%%%%%%%%%%                                      ~~~~~~~~~~~~~~~~~~~~~~~~~~~~~~~~~~~~~~~~~~~~~~~~~~
%% MACROS %       
%%%%%%%%%%%                                      ~~~~~~~~~~~~~~~~~~~~~~~~~~~~~~~~~~~~~~~~~~~~~~~~~~
%
\ifPhdDoc\else
\ifOverLeaf\else
\let\oldincludegraphics\includegraphics
\renewcommand\includegraphics[2][]{%
  \immediate\write18{../Simulator/bin/laFigure #2 #1}%
  \input{result}}%
\fi
\fi
%  
%% \DeclareMathOperator*{\argmin}{\text{arg}\;\text{min}}
%
%\newcommand\Fig[1]{Fig.~\ref{#1}}
%\newcommand\eq[1]{(\ref{#1})}
%
%\newcommand\Grey[1]{{\color{Grey}#1}}
%\newcommand\Red[1]{{\color{Red}#1}}
%\newcommand\Blue[1]{{\color{Blue}#1}}
%\newcommand\DarkBlue[1]{{\color{DarkBlue}#1}}
%\newcommand\LightBlue[1]{{\color{LightBlue}#1}}
%\newcommand\Brown[1]{{\color{Brown}#1}}
%\newcommand\Green[1]{{\color{Green}#1}}
%\newcommand\SeaGreen[1]{{\color{SeaGreen}#1}}
%\newcommand\Yellow[1]{{\color{yellow}#1}}
%\newcommand\Orange[1]{{\color{orange}#1}}
%
%\newcommand\nn{\nonumber\\}
%
%\newcommand\nmat[1]{\begin{matrix}#1\end{matrix}}
%\newcommand\bmat[1]{\begin{bmatrix}#1\end{bmatrix}}
%\newcommand\case[1]{\begin{cases}#1\end{cases}}
%\newcommand\textbox[2]{\footnotesize\text{\parbox{#1}{\centering\emph{#2}}}}
%
%\newcommand\rand{\text{rand}}
%\newcommand\randn{\text{randn}}
%\newcommand\rect{\text{rect}}
%\newcommand\sinc{\text{sinc}}
%\newcommand\tr{\text{tr}}
%\newcommand\adj{\text{adj}}
%
%% \newcommand\max{\text{max}}
%\newcommand\argmin[1]{\text{arg}\;\underset{#1}{\text{min}}}
%
%\newcommand\qqquad{\quad\qquad}
%\newcommand\qqqquad{\qquad\qquad}
%
%% \renewcommand\l[1]{\left#1}
%% \renewcommand\r[1]{\right#1}
%
%% {\text{\parbox{1.5cm}{\centering volume hyper- sphere}}}
%
%%Keyword colouring:
%\newcommand\kw[1]{#1}
%\newcommand\parm[1]{#1}%\color{Black}#1\color{Black}}
%
%\newcommand\of[1]{\scriptstyle(\parm{#1})\displaystyle}
%\newcommand\df[1]{\scriptstyle[\parm{#1}]\displaystyle}
%\newcommand\var[3]{#1_\text{#2}\of{#3}}
%
%\newcommand\diag{\text{diag}}
%
%% \raisebox{lift}[extend-above-baseline][extend-below-baseline]{text}
%\newcommand\mt[1]{\text{\emph{#1}}} %mt = mathtext
%\newcommand\mathnorm{\textstyle}
%\newcommand\mathbig[1]{\displaystyle#1\mathnorm}
%\newcommand\mathsmall[1]{\scriptstyle#1\mathnorm}
%\newcommand\mathtiny[1]{\scriptscriptstyle#1\mathnorm}
%\newcommand\sfrac[2]{\scriptstyle\raisebox{0.25pt}[0pt][0pt]{$\frac{#1}{#2}$}\mathnorm}
%\newcommand\nfrac[2]{\textstyle\frac{#1}{#2}\displaystyle}
%
%\newcommand\sumu[1]{\sum\limits^{#1}\;}
%\newcommand\suml[1]{\sum\limits_{#1}\;}
%\newcommand\sumb[2]{\sum\limits_{#1}^{#2}\;}
%
%\newcommand\produ[1]{\prod\limits^{#1}\;}
%\newcommand\prodl[1]{\prod\limits_{#1}\;}
%\newcommand\prodb[2]{\prod\limits_{#1}^{#2}\;}
%
%\newcommand\defeq{\overset{\underset{\mathrm{def}}{}}{=}}
%
%%Math macros:
%\newcommand\T{^{\scriptscriptstyle T}}
%\renewcommand\H{^{\scriptscriptstyle H}}
%
%\renewcommand\vec[1]{\boldsymbol{#1}}
%\newcommand\mat[1]{\boldsymbol{#1}}
%
%\newcommand\Om{O_\text{m}}
%\newcommand\Oa{O_\text{a}}
%\newcommand\Nl{N_\text{l}}
%\newcommand\Nk{N_\text{k}}
%\newcommand\1{\vec 1}
%\newcommand\I{\mat I}
%\renewcommand*\a{\vec a}
%\newcommand*\f{\vec f}
%\renewcommand*\i{\vec i}
%\renewcommand*\k{\vec k}
%\newcommand*\n{\vec n}
%\newcommand*\p{\vec p}
%\newcommand*\s{\vec s}
%\newcommand*\w{\vec w}
%\newcommand*\x{\vec x}
%\newcommand*\y{\vec y}
%
%\newcommand*\A{\mat A}
%\newcommand*\B{\mat B}
%% \newcommand*\C{\mat C}
%\newcommand*\E{\mat E}
%\renewcommand*\P{\mat P}
%\newcommand*\eP{\mat{\hat P}}
%\newcommand*\R{\mat R}
%\newcommand*\Ri{\R^{-1}}
%\newcommand*\eR{\mat{\hat R}}
%\newcommand*\eRi{\hat{\mat R}\;\!^{-1}}
%\newcommand*\Navg{N_\text{avg}}
%\newcommand*\W{\mat W}
%\newcommand*\X{\mat X}
%\newcommand*\Xd{\X_{\!\Delta}}
%\newcommand*\Y{\mat Y}
%
%\renewcommand\argmin{\text{argmin}}
%
%

%
%\renewcommand*\L{\mat \Lambda}
%% \newcommand*\U{\mat U}
%% \renewcommand*\t{\mathtiny{^T}}
%% \newcommand*\h{\mathtiny{^H}}
%\renewcommand*\t{^T}
%\newcommand*\h{^H}
\typeout{====Start of macro file====}

\providecommand\Fig{}
\renewcommand\Fig[1]{Fig.~\ref{#1}}
\providecommand\Tab{}
\renewcommand\Tab[1]{Table~\ref{#1}}
\providecommand\eq{}
\renewcommand\eq[1]{(\ref{#1})}

\providecommand\Grey[1]{{\color{Grey}#1}}
\providecommand\Red[1]{{\color{Red}#1}}
\providecommand\Blue[1]{{\color{Blue}#1}}
\providecommand\DarkBlue[1]{{\color{DarkBlue}#1}}
\providecommand\LightBlue[1]{{\color{LightBlue}#1}}
\providecommand\Brown[1]{{\color{Brown}#1}}
\providecommand\Green[1]{{\color{Green}#1}}
\providecommand\SeaGreen[1]{{\color{SeaGreen}#1}}
\providecommand\Yellow[1]{{\color{yellow}#1}}
\providecommand\Orange[1]{{\color{orange}#1}}

\providecommand*\nn{}
\renewcommand*\nn{\nonumber\\}

\providecommand\nmat{}
\renewcommand\nmat[1]{\begin{matrix}#1\end{matrix}}
\providecommand\bmat{}
\renewcommand\bmat[1]{\begin{bmatrix*}[c]#1\end{bmatrix*}}
\providecommand\case{}
\renewcommand\case[1]{\begin{cases}#1\end{cases}}
\providecommand\textbox{}
\renewcommand\textbox[2]{\footnotesize\text{\parbox{#1}{\centering\emph{#2}}}}

\providecommand\rand{}
\renewcommand\rand{\text{rand}}
\providecommand\randn{}
\renewcommand\randn{\text{randn}}
\providecommand\rect{}
\renewcommand\rect{\text{rect}}
\providecommand\sinc{}
\renewcommand\sinc{\text{sinc}}
\providecommand\tr{}
\renewcommand\tr{\text{tr}}
\providecommand\adj{}
\renewcommand\adj{\text{adj}}

% \renewcommand\max{\text{max}}
\providecommand\argmin{}
\renewcommand\argmin[1]{\text{arg}\;\underset{#1}{\text{min}}}

\providecommand\qqquad{}
\renewcommand\qqquad{\quad\qquad}
\providecommand\qqqquad{}
\renewcommand\qqqquad{\qquad\qquad}

% renewcommand\l[1]{\left#1}
% renewcommand\r[1]{\right#1}

% {\text{\parbox{1.5cm}{\centering volume hyper- sphere}}}

%Keyword colouring:
\providecommand\kw{}
\renewcommand\kw[1]{#1}
\providecommand\parm{}
\renewcommand\parm[1]{#1}%\color{Black}#1\color{Black}}

\providecommand\of{}
\renewcommand\of[1]{\scriptstyle(\parm{#1})\displaystyle}
\providecommand\df{}
\renewcommand\df[1]{\scriptstyle[\parm{#1}]\displaystyle}
\providecommand\var{}
\renewcommand\var[3]{#1_\text{#2}\of{#3}}

\providecommand\diag{}
\renewcommand\diag{\text{diag}}

% \raisebox{lift}[extend-above-baseline][extend-below-baseline]{text}
\providecommand\mt{}
\renewcommand\mt[1]{\text{\emph{#1}}} %mt = mathtext
\providecommand\mathnorm{}
\renewcommand\mathnorm{\textstyle}
\providecommand\mathbig{}
\renewcommand\mathbig[1]{\displaystyle#1\mathnorm}
\providecommand\mathsmall{}
\renewcommand\mathsmall[1]{\scriptstyle#1\mathnorm}
\providecommand\mathtiny{}
\renewcommand\mathtiny[1]{\scriptscriptstyle#1\mathnorm}
\providecommand\sfrac{}
\renewcommand\sfrac[2]{\scriptstyle\raisebox{0.25pt}[0pt][0pt]{$\frac{#1}{#2}$}\mathnorm}
\providecommand\nfrac{}
\renewcommand\nfrac[2]{\textstyle\frac{#1}{#2}\displaystyle}

\providecommand\sumu{}
\renewcommand\sumu[1]{\sum\limits^{#1}\;}
\providecommand\suml{}
\renewcommand\suml[1]{\sum\limits_{#1}\;}
\providecommand\sumb{}
\renewcommand\sumb[2]{\sum\limits_{#1}^{#2}\;}

\providecommand\produ{}
\renewcommand\produ[1]{\prod\limits^{#1}\;}
\providecommand\prodl{}
\renewcommand\prodl[1]{\prod\limits_{#1}\;}
\providecommand\prodb{}
\renewcommand\prodb[2]{\prod\limits_{#1}^{#2}\;}

\providecommand\defeq{}
\renewcommand\defeq{\overset{\underset{\mathrm{def}}{}}{=}}

%Math macros:
\providecommand*\T{}
\renewcommand*\T{^{\scriptscriptstyle T}}
\providecommand\H{}
\renewcommand\H{^{\scriptscriptstyle H}}

\providecommand\udot{}  
\renewcommand\udot[1]{\underaccent{\dot}{#1}}
\providecommand\ubar{}  
\renewcommand\ubar[1]{\underaccent{\bar}{#1}}
%\renewcommand\ubar[1]{\underaccent{\bar}{#1}}
\providecommand\uvec{}  
% \renewcommand\uvec[1]{\underaccent{\vec}{\kern0pt#1}}
\renewcommand\uvec[1]{\underaccent{\vec}{#1}}
\providecommand\dvec{}                  % Decomposed vector
\renewcommand\dvec[1]{\boldsymbol{#1}}
\providecommand\dhat{}                  % Decomposed unit vector
\renewcommand\dhat[1]{\boldsymbol{\hat{#1}}}
\providecommand\mat{}
\renewcommand\mat[1]{\boldsymbol{#1}}
\providecommand\dmat{}                  % Decomposed vector
\renewcommand\dmat[3]{\boldsymbol{#1}_{#2}^{#3}}

\providecommand\Om{}
\renewcommand\Om{O_\text{m}}
\providecommand\Oa{}
\renewcommand\Oa{O_\text{a}}
\providecommand\Nl{}
\renewcommand\Nl{N_\text{l}}
\providecommand\Nk{}
\renewcommand\Nk{N_\text{k}}
\providecommand\1{}
\renewcommand\1{\vec 1}
\providecommand\I{}
\renewcommand\I{\mat I}
\providecommand\a{}
\renewcommand*\a{\vec a}
\providecommand\c{}
\renewcommand*\c{\vec c}
\providecommand\f{}
\renewcommand*\f{\vec f}
\providecommand\i{}
\renewcommand*\i{\vec i}
\providecommand\k{}
\renewcommand*\k{\vec k}
\providecommand\n{}
\renewcommand*\n{\vec n}
\providecommand\p{}
\renewcommand*\p{\vec p}
\providecommand\s{}
\renewcommand*\s{\vec s}
\providecommand\w{}
\renewcommand*\w{\vec w}
\providecommand\x{}
\renewcommand*\x{\vec x}
\providecommand\y{}
\renewcommand*\y{\vec y}

\providecommand\A{}
\renewcommand*\A{\mat A}
%\renewcommand*\B{\mat B}
%\renewcommand*\C{\mat C}
\providecommand\E{}
\renewcommand*\E{\mat E}
\providecommand\P{}
\renewcommand*\P{\mat P}
\providecommand\eP{}
\renewcommand*\eP{\mat{\hat P}}
\providecommand\R{}
\renewcommand*\R{\mat R}
\providecommand\Ri{}
\renewcommand*\Ri{\R^{-1}}
\providecommand\eR{}
\renewcommand*\eR{\mat{\hat R}}
\providecommand\eRi{}
\renewcommand*\eRi{\hat{\mat R}\;\!^{-1}}
\providecommand*\M{}
\renewcommand*\M{\mat M}
\providecommand\Navg{}
\renewcommand*\Navg{N_\text{avg}}
\providecommand*\V{}
\renewcommand*\V{\mat V}
\providecommand\W{}
\renewcommand*\W{\mat W}
\providecommand\X{}
\renewcommand*\X{\mat X}
\providecommand\Xd{}
\renewcommand*\Xd{\X_{\!\Delta}}
\providecommand\Y{}
\renewcommand*\Y{\mat Y}



%\providecommand\L{}
%\renewcommand*\L{\mat \Lambda}
%\renewcommand*\U{\mat U}
% renewcommand*\t{\mathtiny{^T}}
% \renewcommand*\h{\mathtiny{^H}}
\providecommand\t{}
\renewcommand*\t{^T}
\providecommand\h{}
\renewcommand*\h{^H}

\providecommand\minus{}
\renewcommand\minus{\scalebox{0.75}[1.0]{$-$}}

%\ifPeerReview
%\graphicspath{./submission/submit1}
%\fi




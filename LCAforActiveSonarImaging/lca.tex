
%%%%%%%%%%%%%%%%                                ~~~~~~~~~~~~~~~~~~~~~~~~~~~~~~~~~~~~~~~~~~~~~~~~~~
% CONDITIONALS %
%%%%%%%%%%%%%%%%                                ~~~~~~~~~~~~~~~~~~~~~~~~~~~~~~~~~~~~~~~~~~~~~~~~~~

\newif\ifPeerReview\PeerReviewfalse             % Whether to create the PeerReview version or
                                                % Journal version
\newif\ifFlatArchive\FlatArchivefalse           % Whether archive is flat (messy) or contain 
                                                % subfolders for graphics etc.
\newif\ifFloatAtEnd\FloatAtEndfalse             % Available in PeerReview mode:
                                                % Place floats at end of document?
\newif\ifTODO\TODOtrue                        % Use todo notes?

%%%%%%%%%%%%                                    ~~~~~~~~~~~~~~~~~~~~~~~~~~~~~~~~~~~~~~~~~~~~~~~~~~
% IEEEtran %
%%%%%%%%%%%%                                    ~~~~~~~~~~~~~~~~~~~~~~~~~~~~~~~~~~~~~~~~~~~~~~~~~~

\ifPeerReview
\documentclass[12pt,journal,captionsoff,onecolumn]{IEEEtran}
\newcommand\CLASSINPUTbaselinestretch{1.66}     % http://theoval.cmp.uea.ac.uk/~nlct/latex/thesis/node17.html
\else
\documentclass[journal]{IEEEtran}
\fi



%%%%%%%%%%%%%%%%%%%%%%%%%%%%%%                  ~~~~~~~~~~~~~~~~~~~~~~~~~~~~~~~~~~~~~~~~~~~~~~~~~~
% IEEE ''APPROVED'' PACKAGES %
%%%%%%%%%%%%%%%%%%%%%%%%%%%%%%                  ~~~~~~~~~~~~~~~~~~~~~~~~~~~~~~~~~~~~~~~~~~~~~~~~~~


\ifCLASSINFOpdf
   \usepackage[dvips]{graphicx}                 % Might not work. Use 'latex' instead of 
   \ifFlatArchive\else                          % 'pdflatex'
      \graphicspath{./gfx/}
   \fi
\else
   \usepackage[dvips]{graphicx}
   \ifFlatArchive\else
      \graphicspath{./gfx/}
   \fi
\fi

\RequirePackage[table,dvipsnames,svgnames]{xcolor}

\usepackage[cmex10]{amsmath}                    % cmex10 option to be IEEE explore compliant
\interdisplaylinepenalty=2500                   % Allows multiline equations to be broken

% \RequirePackage{amssymb}

\RequirePackage{array}

\ifCLASSOPTIONcompsoc
   \usepackage[caption=false,font=normalsize,labelfont=sf,textfont=sf]{subfig}
\else
   \usepackage[caption=false,font=footnotesize]{subfig}
\fi
\ifCLASSOPTIONcaptionsoff                       % IEEE promoted hack to turn off captions from the 
   \let\MYorigsubfloat\subfloat                 % subfloat package should the captionsoff option
   \renewcommand{\subfloat}[2][\relax]{\MYorigsubfloat[]{#2}} % be specified.
\fi

\ifFloatAtEnd
\ifCLASSOPTIONcaptionsoff                       % Places float at the end of the document when the
  \usepackage[nomarkers]{endfloat}              % captionsoff options is specified to IEEEtrans.cls
  \let\MYoriglatexcaption\caption               % (PeerReview mode)
  \renewcommand{\caption}[2][\relax]{\MYoriglatexcaption[#2]{#2}}
\fi
\fi

\usepackage{fixltx2e}                           % Fix some twocolumn float problems

%\usepackage{stfloats}                          % Allows: \begin{figure*}[!b]
                                                % (double column figures on top/bottom)

\usepackage{url}                                % Support for handling and breaking URLs

% NOTE: PDF hyperlink and bookmark features are not required in IEEE
%       papers and their use requires extra complexity and work.
\newcommand\MYhyperrefoptions{bookmarks=true,bookmarksnumbered=true,
pdfpagemode={UseOutlines},plainpages=false,pdfpagelabels=true,
colorlinks=true,linkcolor={black},citecolor={black},urlcolor={black},
pdftitle={Low Complexity Adaptive Beamformer for Active Sonar Imaging},
pdfsubject={},
pdfauthor={Jo Inge Buskenes},
pdfkeywords={adaptive beamforming, beamforming, complexity, sonar, active}}%
\ifCLASSINFOpdf
   \usepackage[\MYhyperrefoptions,pdftex]{hyperref}
\else
   \usepackage[\MYhyperrefoptions,breaklinks=true,dvips]{hyperref}
   \usepackage{breakurl}                        % Allows 'dvips' driver to break links
\fi

%%%%%%%%%%%%%%%%%%%%%%%                         ~~~~~~~~~~~~~~~~~~~~~~~~~~~~~~~~~~~~~~~~~~~~~~~~~~
% ADDITIONAL PACKAGES %       
%%%%%%%%%%%%%%%%%%%%%%%                         ~~~~~~~~~~~~~~~~~~~~~~~~~~~~~~~~~~~~~~~~~~~~~~~~~~

\newcounter{todoidx}
% \setcounter{todoidx}

\ifTODO
   \setlength\marginparsep{3pt}
   \setlength\marginparwidth{42pt}

   \RequirePackage{tikz}

%    \define@key{boxedtheorem}{titlecolor}{\def\titlecolor{#1}}
%    \define@key{boxedtheorem}{titlebackground}{\def\titlebackground{#1}}
%    \define@key{boxedtheorem}{background}{\def\background{#1}}
%    \define@key{boxedtheorem}{titleboxcolor}{\def\titleboxcolor{#1}}
%    \define@key{boxedtheorem}{boxcolor}{\def\boxcolor{#1}}
%    \define@key{boxedtheorem}{thcounter}{\def\thcounter{#1}}
%    \define@key{boxedtheorem}{size}{\def\size{#1}}
%    \presetkeys{boxedtheorem}{titlecolor = black, titlebackground = white, background = white,%
%                              titleboxcolor = black, boxcolor = black, thcounter=, size = .9\textwidth}{}
%    \newcommand{\colors}[1][]{%
%       \setkeys{boxedtheorem}{#1}
%       \tikzstyle{fancytitle} =[draw=\titleboxcolor, rounded corners, fill=\titlebackground,
%                               text= \titlecolor]
%       \tikzstyle{mybox} = [draw=\boxcolor, fill=\background, very thick,
%                            rectangle, rounded corners, inner sep=10pt, inner ysep=20pt]
%    }
% 
%    \newsavebox{\boiboite}
%    \newcommand{\titre}{Titre}
%    \newenvironment{boite}[2][]%
%       {%
%       \renewcommand{\titre}{#2}
%       \colors[#1]
%       \begin{lrbox}{\boiboite}%
%       \begin{minipage}[!h]{\size}
%       }%
%       {%
%       \end{minipage}
%       \end{lrbox}
%       \begin{center}
%       \begin{tikzpicture}
%       \node [mybox] (box){\usebox{\boiboite}};
%       \node[fancytitle, right=10pt] at (box.north west) {\titre};
%       \end{tikzpicture}
%       \end{center}
%       }
% 
%    \newcommand{\newboxedtheorem}[4][]{%
%       \colors[#1]
%       \@ifnotempty{#4}{%
%          \@ifundefined{the#4}{\@ifundefined{\thcounter}{\newcounter{#4}}{%
%          \newcounter{#4}[\thcounter ] } } { }%
%       }
%       \newenvironment{#2}[1][]{%
%       \@ifnotempty{#4}{\refstepcounter{#4}}
%       \begin{boite}[#1]{\textbf{#3\@ifnotempty{#4}{ \csname the#4\endcsname}}\@ifnotempty{##1}{
%       (##1)}\textbf{.}}
%       }%
%       {%
%       \end{boite}
%       }
%    }

      \newsavebox{\notebox}
      \tikzstyle{fancytitle} =[draw=black, rounded corners, fill=white,
                              text=black]
      \tikzstyle{mybox} = [draw=gray, fill=gray, very thick,
                           rectangle, rounded corners, inner sep=10pt, inner ysep=20pt]
   
%    \newlength\marginparwidthsmall
%    \setlength\marginparwidthsmall{\marginparwidth}
%    \addtolength\marginparwidthsmall{-7pt}
     \newcommand\todo[1]{%
         \marginpar{%
            \begin{tikzpicture}
            \node [mybox] (box){\usebox{\notebox}};
            \node[fancytitle, right=10pt] at (box.north west) {Note};
            \end{tikzpicture}}}
%    \newcommand\todo[1]{%
%       \addtocounter{todoidx}{1}%
%       {\color{Red}\fbox{\bf\thetodoidx{}}}%
%       \marginpar{%
%          {\vspace*{-10pt}\color{Red}\fbox{\bf\thetodoidx{}}}\\%
%          \colorbox{red}{\fbox{\parbox{\marginparwidthsmall}{#1}}}}}
\else
%    \usepackage[disable]{./todonotes} 
   \newcommand\todo[1]{}
\fi


% correct bad hyphenation here
% \hyphenation{op-tical net-works semi-conduc-tor}


%%%%%%%%%%%%%%%%%%                              ~~~~~~~~~~~~~~~~~~~~~~~~~~~~~~~~~~~~~~~~~~~~~~~~~~
% DOCUMENT START %
%%%%%%%%%%%%%%%%%%                              ~~~~~~~~~~~~~~~~~~~~~~~~~~~~~~~~~~~~~~~~~~~~~~~~~~
  \RequirePackage{layout}  
\begin{document}
\layout
%
\title{Low Complexity Adaptive Beamformer\\ for Active Sonar Imaging}

\author{Jo~Inge~Buskenes, %
        Andreas~Austeng, %
        Carl-Inge~Columbo~Nilsen%
\IEEEcompsocitemizethanks{\IEEEcompsocthanksitem All authors are with the Department
of Informatics, University of Oslo, Norway.}% <-this % stops a space

% \thanks{Manuscript received April%       \node [mybox] (box){\usebox{\boiboite}};
%       \node[fancytitle, right=10pt] at (box.north west) {\titre};
%       \end{tikzpicture}}} 19, 2005; revised January 11, 2007.}
}

% The paper headers
\markboth{IEEE Journal of Oceanic Engineering}%
{Low Complexity Adaptive Beamformer for Active Sonar Imaging}

% Publishers ID mark:
%\IEEEpubid{0000--0000/00\$00.00~\copyright~2007 IEEE}

% use for special paper notices
%\IEEEspecialpapernotice{(Invited Paper)}

% for Computer Society papers, we must declare the abstract and index terms
% PRIOR to the title within the \IEEEcompsoctitleabstractindextext IEEEtran
% command as these need to go into the title area created by \maketitle.
\IEEEcompsoctitleabstractindextext{%
\begin{abstract}
The angular resolution and contrast in active sonar images depend on the beamformer's ability to receive signals from directions of interest, while suppressing noise and interference emanating from other directions. For sonar arrays, this is achieved by applying weights to the array channels.

While classical beamformers use predefined windows, adaptive beamformers estimate the optimal window by analytical evaluation of the data. The minimum variance (MV) beamformer, for instance, calculates the set of weights that minimises the variance of the beamformer's output. 

We have implemented a Low Complexity Adaptive (LCA) beamformer, which adaptively selects a window from a predefined set. The set is comprised of windows that are typical solutions found by the Minimum Variance method. The LCA beamformer was tested using simulated and experimental data from the Kongsberg Maritime HISAS 1030 sonar. On a simulated scene with speckle, highlight and shadow, the beamformer offered better lateral edge definition compared to the MV beamformer, and speckle intensity and shape comparable to DAS and MV beamformers. These results were verified by the experimental data.

An attactive trait of the LCA is its low computational complexity; while the MV beamformer is of O($M^3$), the LCA method is of O($MW$), with $M$ being the number of channels and $W$ the number of windows. We made the LCA perform like the MV method using a well designed set of 30 windows. Hence, unless the array is very small, the proposed method will perform like the MV beamformer or better, and at a fraction of the computational cost.
\end{abstract}

% Keywords (normally not used for peer reviews)
\ifPeerReview\else
\begin{IEEEkeywords}
Beamforming, adaptive beamforming, sonar, active, complexity.
\end{IEEEkeywords}
\fi}
% \fi

% make the title area
\maketitle

% This command fixes abstract positioning for compsoc articles:
\IEEEdisplaynotcompsoctitleabstractindextext

% (Optional) Add some extra info on cover page of peer review papers:
% \ifCLASSOPTIONpeerreview
% \begin{center} \bfseries EDICS Category: 3-BBND \end{center}
% \fi

% Insert page break and insert second title (peer review mode)
\IEEEpeerreviewmaketitle



\section{Introduction}
% Computer Society journal papers do something a tad strange with the very
% first section heading (almost always called "Introduction"). They place it
% ABOVE the main text! IEEEtran.cls currently does not do this for you.
% However, You can achieve this effect by making LaTeX jump through some
% hoops via something like:
%
%\ifCLASSOPTIONcompsoc
%  \noindent\raisebox{2\baselineskip}[0pt][0pt]%
%  {\parbox{\columnwidth}{\section{Introduction}\label{sec:introduction}%
%  \global\everypar=\everypar}}%
%  \vspace{-1\baselineskip}\vspace{-\parskip}\par
%\else
%  \section{Introduction}\label{sec:introduction}\par
%\fi
%
% Admittedly, this is a hack and may well be fragile, but seems to do the
% trick for me. Note the need to keep any \label that may be used right
% after \section in the above as the hack puts \section within a raised box.



\IEEEPARstart{A}{daptive} beamformers have only recently been introduced in active sonar imaging.\todo{break at something} This is because , as demo file is intended to serve as a ``starter file''
for IEEE Computer Society journal papers produced under \LaTeX\ using
IEEEtran.cls version 1.7 and later.

We have implemented a Low Complexity Adaptive (LCA) beamformer, which adaptively selects a window from a predefined set. The set is comprised of windows that are typical solutions found by the Minimum Variance method. The LCA beamformer was tested using simulated and experimental data from the Kongsberg Maritime HISAS 1030 sonar. On a simulated scene with speckle, highlight and shadow, the beamformer offered better lateral edge definition compared to the MV beamformer, and speckle intensity and shape comparable to DAS and MV beamformers. These results were verified by the experimental data.

An attactive trait of the LCA is its low computational complexity; while the MV beamformer is of O($M^3$), the LCA method is of O($MW$), with $M$ being the number of channels and $W$ the number of windows. We made the LCA perform like the MV method using a well designed set of 30 windows. Hence, unless the array is very small, the proposed method will perform like the MV beamformer or better, and at a fraction of the computational cost.

We have implemented a Low Complexity Adaptive (LCA) beamformer, which adaptively selects a window from a predefined set. The set is comprised of windows that are typical solutions found by the Minimum Variance method. The LCA beamformer was tested using simulated and experimental data from the Kongsberg Maritime HISAS 1030 sonar. On a simulated scene with speckle, highlight and shadow, the beamformer offered better lateral edge definition compared to the MV beamformer, and speckle intensity and shape comparable to DAS and MV beamformers. These results were verified by the experimental data.

An attactive trait of the LCA\todo{break at something} is its low computational complexity; while the MV beamformer is of O($M^3$), the LCA method is of O($MW$), with $M$ being the number of channels and $W$ the number of windows. We made the LCA perform like the MV method using a well designed set of 30 windows. Hence, unless the array is very small, the proposed method will perform like the MV beamformer or better, and at a fraction of the computational cost.

We have implemented a Low Complexity Adaptive (LCA) beamformer, which adaptively selects a window from a predefined set. The set is comprised of windows that are typical solutions found by the Minimum Variance method. The LCA beamformer was tested using simulated and experimental data from the Kongsberg Maritime HISAS 1030 sonar. On a simulated scene with speckle, highlight and shadow, the beamformer offered better lateral edge definition compared to the MV beamformer, and speckle intensity and shape comparable to DAS and MV beamformers. These results were verified by the experimental data.

An attactive trait of the LCA is its low computational complexity; while the MV beamformer is of O($M^3$), the LCA method is of O($MW$), with $M$ being the number of channels and $W$ the number of windows. We made the LCA perform like the MV method using a well designed set of 30 windows. Hence, unless the array is very small, the proposed method will perform like the MV beamformer or better, and at a fraction of the computational cost.

\hfill mds
 
\hfill January 11, 2007

\subsection{Subsection Heading Here}
Subsection text here.

% needed in second column of first page if using \IEEEpubid
%\IEEEpubidadjcol

\subsubsection{Subsubsection Heading Here}
Subsubsection text here.


% An example of a floating figure using the graphicx package.
% Note that \label must occur AFTER (or within) \caption.
% For figures, \caption should occur after the \includegraphics.
% Note that IEEEtran v1.7 and later has special internal code that
% is designed to preserve the operation of \label within \caption
% even when the captionsoff option is in effect. However, because
% of issues like this, it may be the safest practice to put all your
% \label just after \caption rather than within \caption{}.
%
% Reminder: the "draftcls" or "draftclsnofoot", not "draft", class
% option should be used if it is desired that the figures are to be
% displayed while in draft mode.
%
%\begin{figure}[!t]
%\centering
%\includegraphics[width=2.5in]{myfigure}
% where an .eps filename suffix will be assumed under latex, 
% and a .pdf suffix will be assumed for pdflatex; or what has been declared
% via \DeclareGraphicsExtensions.
%\caption{Simulation Results}
%\label{fig_sim}
%\end{figure}

% Note that IEEE typically puts floats only at the top, even when this
% results in a large percentage of a column being occupied by floats.
% However, the Computer Society has been known to put floats at the bottom.


% An example of a double column floating figure using two subfigures.
% (The subfig.sty package must be loaded for this to work.)
% The subfigure \label commands are set within each subfloat command, the
% \label for the overall figure must come after \caption.
% \hfil must be used as a separator to get equal spacing.
% The subfigure.sty package works much the same way, except \subfigure is
% used instead of \subfloat.
%
%\begin{figure*}[!t]
%\centerline{\subfloat[Case I]\includegraphics[width=2.5in]{subfigcase1}%
%\label{fig_first_case}}
%\hfil
%\subfloat[Case II]{\includegraphics[width=2.5in]{subfigcase2}%
%\label{fig_second_case}}}
%\caption{Simulation results}
%\label{fig_sim}
%\end{figure*}
%
% Note that often IEEE papers with subfigures do not employ subfigure
% captions (using the optional argument to \subfloat), but instead will
% reference/describe all of them (a), (b), etc., within the main caption.


% An example of a floating table. Note that, for IEEE style tables, the 
% \caption command should come BEFORE the table. Table text will default to
% \footnotesize as IEEE normally uses this smaller font for tables.
% The \label must come after \caption as always.
%
%\begin{table}[!t]
%% increase table row spacing, adjust to taste
%\renewcommand{\arraystretch}{1.3}
% if using array.sty, it might be a good idea to tweak the value of
% \extrarowheight as needed to properly center the text within the cells
%\caption{An Example of a Table}
%\label{table_example}
%\centering
%% Some packages, such as MDW tools, offer better commands for making tables
%% than the plain LaTeX2e tabular which is used here.
%\begin{tabular}{|c||c|}
%\hline
%One & Two\\
%\hline
%Three & Four\\
%\hline
%\end{tabular}
%\end{table}


% Note that IEEE does not put floats in the very first column - or typically
% anywhere on the first page for that matter. Also, in-text middle ("here")
% positioning is not used. Most IEEE journals use top floats exclusively.
% However, Computer Society journals sometimes do use bottom floats - bear
% this in mind when choosing appropriate optional arguments for the
% figure/table environments.
% Note that, LaTeX2e, unlike IEEE journals, places footnotes above bottom
% floats. This can be corrected via the \fnbelowfloat command of the
% stfloats package.



\section{Conclusion}
The conclusion goes here.





% if have a single appendix:
%\appendix[Proof of the Zonklar Equations]
% or
%\appendix  % for no appendix heading
% do not use \section anymore after \appendix, only \section*
% is possibly needed

% use appendices with more than one appendix
% then use \section to start each appendix
% you must declare a \section before using any
% \subsection or using \label (\appendices by itself
% starts a section numbered zero.)
%


\appendices
\section{Proof of the First Zonklar Equation}
Appendix one text goes here.

% you can choose not to have a title for an appendix
% if you want by leaving the argument blank
\section{}
Appendix two text goes here.


% use section* for acknowledgement
\ifCLASSOPTIONcompsoc
  % The Computer Society usually uses the plural form
  \section*{Acknowledgments}
\else
  % regular IEEE prefers the singular form
  \section*{Acknowledgment}
\fi


The authors would like to thank...


% Can use something like this to put references on a page
% by themselves when using endfloat and the captionsoff option.
\ifCLASSOPTIONcaptionsoff
  \newpage
\fi



% trigger a \newpage just before the given reference
% number - used to balance the columns on the last page
% adjust value as needed - may need to be readjusted if
% the document is modified later
%\IEEEtriggeratref{8}
% The "triggered" command can be changed if desired:
%\IEEEtriggercmd{\enlargethispage{-5in}}

% references section

% can use a bibliography generated by BibTeX as a .bbl file
% BibTeX documentation can be easily obtained at:
% http://www.ctan.org/tex-archive/biblio/bibtex/contrib/doc/
% The IEEEtran BibTeX style support page is at:
% http://www.michaelshell.org/tex/ieeetran/bibtex/
%\bibliographystyle{IEEEtran}
% argument is your BibTeX string definitions and bibliography database(s)
%\bibliography{IEEEabrv,../bib/paper}
%
% <OR> manually copy in the resultant .bbl file
% set second argument of \begin to the number of references
% (used to reserve space for the reference number labels box)
\begin{thebibliography}{1}

\bibitem{IEEEhowto:kopka}
H.~Kopka and P.~W. Daly, \emph{A Guide to {\LaTeX}}, 3rd~ed.\hskip 1em plus
  0.5em minus 0.4em\relax Harlow, England: Addison-Wesley, 1999.

\end{thebibliography}

% biography section
% 
% If you have an EPS/PDF photo (graphicx package needed) extra braces are
% needed around the contents of the optional argument to biography to prevent
% the LaTeX parser from getting confused when it sees the complicated
% \includegraphics command within an optional argument. (You could create
% your own custom macro containing the \includegraphics command to make things
% simpler here.)
%\begin{biography}[{\includegraphics[width=1in,height=1.25in,clip,keepaspectratio]{mshell}}]{Michael Shell}
% or if you just want to reserve a space for a photo:

\begin{IEEEbiography}{Michael Shell}
Biography text here.
\end{IEEEbiography}

% if you will not have a photo at all:
\begin{IEEEbiographynophoto}{John Doe}
Biography text here.
\end{IEEEbiographynophoto}

% insert where needed to balance the two columns on the last page with
% biographies
%\newpage

\begin{IEEEbiographynophoto}{Jane Doe}
Biography text here.
\end{IEEEbiographynophoto}

% You can push biographies down or up by placing
% a \vfill before or after them. The appropriate
% use of \vfill depends on what kind of text is
% on the last page and whether or not the columns
% are being equalized.

%\vfill

% Can be used to pull up biographies so that the bottom of the last one
% is flush with the other column.
%\enlargethispage{-5in}



% that's all folks
\end{document}



\documentclass[
    beamer                                       % Document type (non-standard)
%  , handout
%   , xelatex                                      % Use the XeLaTeX compiler
%  , movie
  ,table,dvipsnames,svgnames
]{common/mytemplate}

\pdfpageattr {/Group << /S /Transparency /I true /CS /DeviceRGB>>}

% \mode<handout>{\setbeamercolor{background canvas}{bg=black!5}}

% \RequirePackage[table,dvipsnames,svgnames]{xcolor}
\definecolor{tabBlue}{HTML}{AACCFF}
\usepackage{template/beamerthemeUiO}
% \usepackage{common/oxygentheme}
% \setbeameroption{show only notes}
% \setbeameroption{notes on second screen=left}

% \newcommand\remotesource{http://www.joinge.net/compet}

\title{\newline\newline Adapting the MVDR beamformer to a GPU for differently sized active sonar imaging systems}
\subtitle{}
\author[]{\textbf{Jo Inge Buskenes}$^{\text{a}}$, Jon Petter �sen$^{\text{b}}$, Carl-Inge~Colombo~Nilsen$^{\text{a}}$, Andreas~Austeng$^{\text{a}}$}
\date[Jo Inge Buskenes at ASA/ICA, Montreal, June\ 2013]{Montreal, Canada, 2013}
\institute[Dept.\ of Informatics, University of Oslo]{\bf$^{\text{a}}$Department of Informatics, University of Oslo, Norway\\
                                                      \bf$^{\text{b}}$MI-Lab, Norwegian University of Science and Technology}
% \date{Sist revidert: 27.01.2009}

\definecolor{LightBlue}{rgb}{0.3,0.3,1}
\definecolor{DarkBlue}{rgb}{0.2,0.2,0.7}
\setbeamercolor{block title}{bg=DarkBlue,fg=white}%bg=background, fg= foreground
\setbeamercolor{block body}{bg=gray!20,fg=black}%bg=background, fg= foreground
\setbeamertemplate{blocks}[rounded]

\RequirePackage[latin1]{inputenc}%              % Set input encoding (optionally latin1) utf8
\RequirePackage[T1]{fontenc}%                 % Set font encoding

\begin{document}

\begin{frame}
\vspace*{2\baselineskip}
  \titlepage
\end{frame}
\note{\null{}}

% \AtBeginSection[]
% {
%   \frame<handout:0>
%   {
%     \frametitle{Agenda}
%     \tableofcontents %[currentsection,hideallsubsections]
%   }
% }

% \AtBeginSubsection[]
% {
%   \frame<handout:0>
%   {
%     \frametitle{Agenda}
%     \tableofcontents[sectionstyle=show/hide,subsectionstyle=show/shaded/hide]
%   }
% }

\newcommand<>{\highlighton}[1]{%
  \alt#2{\structure{#1}}{{#1}}
}

\newcommand{\icon}[1]{\pgfimage[height=1em]{#1}}


%%%%%%%%%%%%%%%%%%%%%%%%%%%%%%%%%%%%%%%%%
%%%%%%%%%% Content starts here %%%%%%%%%%
%%%%%%%%%%%%%%%%%%%%%%%%%%%%%%%%%%%%%%%%%

% {



% \section{Case study}
% \subsection{Realtime sectorcan imaging}
% 
% {
% \setbeamertemplate{background}{\graphicsAI[width=\paperwidth]{gfx/use_case_sectorscan.svg}}%
% \begin{frame}
% \vspace{-80pt}\hspace{-25pt}
% \frametitle{\color{white}Realtime requirements?}
% \framesubtitle{\vspace{-10pt}\color{white}Sectorscan imaging}
% \vspace{100pt}\ 
% % \begin{figure}[H]
% % \begin{narrow}{-1.6cm}{-1.6cm}
% % \graphicsAI<1>[drawing,width=\linewidth]{gfx/use_case_sectorscan3.svg}
% % \end{narrow}
% % \end{figure}   
% \end{frame}
% }



% \renewcommand{\frametitle}[2]{{\vspace*{10pt}\bf\Large #1\par}}
% \begin{frame}
% \frametitle{Context}
% \framesubtitle{Active Sonar}
% \vspace{-25pt}
% \begin{figure}[H]
% \begin{narrow}{-0.5cm}{0pt}
% \hspace{-10pt}\graphicsAI<1>[drawing,width=1.05\linewidth]{gfx/SonarPrinciple.svg}
% \end{narrow}
% \end{figure}
% % \begin{itemize}
% % \item Each pixel value is estimated by focusing the receiver on that point.
% % \end{itemize}
% \end{frame}

% \begin{frame}
% \frametitle{Context}
% \framesubtitle{Imaging techniques}
% \vspace{-10pt}
% \begin{figure}[H]
% \begin{narrow}{-0.5cm}{0pt}
% \hspace{-10pt}\graphicsAI<1>[drawing,width=1.05\linewidth]{gfx/imaging_concepts.png}
% \end{narrow}
% \end{figure}
% A phased array can be used in several imaging modes:
% \begin{itemize}
% \item \emph{Sector scan}: Image computed from a single ping
% \item \emph{Sidescan}: Image created by stacking images from several subsequent pings
% \item \emph{Synthetic aperture sonar}: Image computed from overlapping pings
% \end{itemize}
% % \begin{block}{Proposed method: Generic, can be used in all modes}
% % \vspace{-25pt}
% % \end{block}
% \end{frame}
% }

% \section{}
% \begin{frame}
%   \frametitle{Agenda}
%   \tableofcontents%[%section=1,
%                    %hidesubsections]
% \end{frame}

\renewcommand{\frametitle}[2]{{\vspace*{10pt}\bf\Large #1\par}}




% \begin{frame}
% \frametitle{Acknowledgements}
% \framesubtitle{}
% \vspace{-3pt}
% \begin{itemize}
% \item Kongsberg Maritime.
% \item FFI guys.
% \end{itemize}
% \end{frame}

% General system, may be applied to several different datasets

% \section{Introduction}
% 
% 
% \begin{frame}
% \frametitle{What we do}
% \framesubtitle{}
% \vspace{-10pt}
% \begin{figure}[H]
% \begin{narrow}{-0.5cm}{0pt}
% \hspace{-10pt}\graphicsAI<1>[drawing,width=1.05\linewidth]{gfx/applications.svg}
% \end{narrow}
% \end{figure}
% % A phased array can be used in several imaging modes:
% % \begin{itemize}
% % \item \emph{Sector scan}: Image computed from a single ping
% % \item \emph{Sidescan}: Image created by stacking images from several subsequent pings
% % \item \emph{Synthetic aperture sonar}: Image computed from overlapping pings
% % \end{itemize}
% % % \begin{block}{Proposed method: Generic, can be used in all modes}
% % % \vspace{-25pt}
% % % \end{block}
% \end{frame}
% 


\begin{frame}
\frametitle{Image Reconstruction}
\framesubtitle{Medical Ultrasound Imaging}
% \vspace{-25pt}
\begin{figure}[H]
\begin{narrow}{-0.5cm}{0pt}
\graphicsAI<1>[drawing,width=1.05\linewidth]{gfx/applications.svg}
\end{narrow}
\end{figure}
\vspace{-10pt}
\begin{itemize}
\item Videos and 3D processing $\rightarrow$ High computational requirements.
\end{itemize}
\end{frame}


\begin{frame}
\frametitle{\hspace{-.5cm}Image Reconstruction}
\framesubtitle{\hspace{-.5cm}Synthetic\\\hspace{-.5cm}Aperture\\\hspace{-.5cm}Sonar}
\framesubtitle{Active Sonar}
\vspace{-45pt}
\begin{figure}[H]
\begin{narrow}{.1cm}{-0.6cm}
\flushright\hspace{-10pt}\graphicsAI<1>[drawing,width=.92\linewidth]{gfx/rock_formation.svg}
\end{narrow}
\end{figure}
\vspace{-10pt}
\begin{narrow}{-0.5cm}{-0.5cm}
\begin{itemize}\small
\item SAS image of a 25 x 25m wide by 5m high rock formation at 70m depth.
\item Theoretical resolution of 3-4cm throughout the image.
\item Data collected by HUGIN AUV carrying the HISAS interferometric SAS. Courtesy of the Norwegian Defence Research Establishment.
\end{itemize}
\end{narrow}
\end{frame}

\begin{frame}
\frametitle{Beamforming}
\framesubtitle{The Delay and Sum beamformer (conventional method)}
\begin{figure}[H]
\begin{narrow}{-0.9cm}{-0.9cm}
\graphicsAI<1>[drawing,width=\linewidth]{gfx/beamforming_das.svg} %{../Docs/GeiloSrc/Figs/das.pdf}
\end{narrow}
\end{figure}                                                                                                                                                                                                                                                                                                   
\end{frame}


{
\setbeamertemplate{background}{\graphicsAI[width=\paperwidth]{gfx/scenario_interference.svg}}%
\begin{frame}
\vspace{-151pt}
\frametitle{\hspace{-.5cm}Simulated Responses}
\framesubtitle{\hspace{-.5cm}What determines a\\\hspace{-.5cm}beamformer's ability to\\\hspace{-.5cm}suppress interference?}
\end{frame}
}


{
\setbeamertemplate{background}{\graphicsAI[width=\paperwidth]{gfx/scenario_interference_das.svg}}%
\begin{frame}
\vspace{-160pt}
\frametitle{\hspace{-.5cm}Simulated Responses}
\framesubtitle{\hspace{-.5cm}The Delay-and-Sum\\\hspace{-.5cm}(DAS) beamformer}
\end{frame}
}

{
\setbeamertemplate{background}{\graphicsAI[width=\paperwidth]{gfx/scenario_interference_mvdr.svg}}%
\begin{frame}
\vspace{-142pt}
\frametitle{\hspace{-.5cm}Simulated Responses}
\framesubtitle{\hspace{-.5cm}The Minimum Variance\\\hspace{-.5cm}Distortionless Response\\\hspace{-.5cm}(MVDR)\\\hspace{-.5cm}beamformer}
\end{frame}
}


% \begin{frame}
% \frametitle{Beamforming}
% \framesubtitle{Minimum Variance Distortionless Response (MVDR) beamformer}
% \begin{figure}[H]
% \begin{narrow}{-0.9cm}{-0.9cm}
% \graphicsAI<1>[drawing,width=\linewidth]{gfx/beamforming_mv.svg} %{../Docs/GeiloSrc/Figs/das.pdf}
% \end{narrow}
% \end{figure}                                                                                                                                                                                                                                                                                                   
% \end{frame}


\begin{frame}
\frametitle{MVDR beamformer}
\framesubtitle{Computational steps and complexity}\\
% \vspace{-15pt}
\begin{figure}[H]
\begin{narrow}{-.5cm}{-.5cm}
\graphicsAI<1>[drawing,width=\linewidth]{gfx/beamforming_mv_light.svg} %{../Docs/GeiloSrc/Figs/das.pdf}
\end{narrow}
\end{figure}
\vspace{-5pt}\textbf{Complexity increases fast with system size, but:}
\begin{itemize}
\item Each pixel in the image can be processed independently.
\item Use Graphics Processing Units (GPUs)?
\end{itemize}
\end{frame}
% 
% 
% % \begin{frame}
% % \frametitle{MVDR}
% % \framesubtitle{Estimating $\R$}
% % \vspace{15pt}
% % % An estimate $\eR$ of $\R$ can be computed as
% % % \begin{align*}
% % % \R = E\{\x[n]\x\H[n]\} = %\frac{1}{(2N_{\text{avg}}}\sumb{n'=n-Navg}{Navg}\sumb{k=0}{M-L+1} \x_k[n]\x_k\H[n]
% % % \end{align*}
% % \begin{itemize}
% % \item Spatial (subarray) averaging to improve estimate
% % \item Temporal averaging to improve speckle statistics
% % \item Diagonal loading improve numerical conditioning
% % \end{itemize}
% % \end{frame}
% 
% \subsection{Graphics processing Units (GPUs)}
% 
\begin{frame}
\makeatletter%
% \special{pdf: put @thispage <</Group << /S /Transparency /I true /CS /DeviceRGB>> >>}%
\makeatother%
\frametitle{GPUs for image reconstruction}
\framesubtitle{}
\vspace{-3pt}
\begin{figure}[H]
\begin{narrow}{0cm}{0cm}
\graphicsAI<1>[drawing,width=\linewidth]{gfx/gpus.svg}%{../Docs/GeiloSrc/Figs/das.pdf}
\end{narrow}
\end{figure}
\begin{itemize}
\item CPUs: Optimized for running \emph{a few} potentially complex threads, that may have \emph{complex} data dependencies.
\item GPUs: Optimized for running \emph{hundreds} of light-weight threads, with \emph{simple} data dependencies.
\end{itemize}
\end{frame}

% 
% \section{Implementation on a GPU}
% 
\begin{frame}
\frametitle{Implementation}
\framesubtitle{MVDR on GPU}
\vspace{-15pt}
\begin{figure}[H]
\begin{narrow}{0cm}{0cm}
\graphicsAI<1>[drawing,width=\linewidth]{gfx/implementation.svg}%{../Docs/GeiloSrc/Figs/das.pdf}
\end{narrow}
\end{figure}
\end{frame}

\begin{frame}
\frametitle{Implementation}
\framesubtitle{Beamspace MVDR on GPU (frequency domain version)}
\vspace{-13pt}
\begin{figure}[H]
\begin{narrow}{0cm}{0cm}
\graphicsAI<1>[drawing,width=\linewidth]{gfx/implementation_beamspace.svg}%{../Docs/GeiloSrc/Figs/das.pdf}
\end{narrow}
\end{figure}
\end{frame}


\begin{frame}
\frametitle{Bandwidth limited design!!!}
\framesubtitle{}
\begin{center}
\begin{minipage}{4cm}
\begin{block}{}
\centering $A = B\;c + D$
\end{block}
\end{minipage}
\end{center}
\begin{itemize}
\item 4 floats over I/O (3 reads, 1 write) 
\item 2 floating point operations.
\end{itemize}
\begin{table}[H]\centering%\normalsize
\begin{tabular}[c]{l r r r@{}  l}\hline
\rowcolor{tabBlue} &
\multicolumn{1}{>{\columncolor{tabBlue}}c}{\bf B$_\text{arith}$} & \multicolumn{1}{>{\columncolor{tabBlue}}c}{\bf B$_\text{mem}$} & \multicolumn{2}{>{\columncolor{tabBlue}}c}{\bf B$_\text{mem}$/B$_\text{arith}$} \\\hline
Arithmetic & 1.03 Tflop/s & & &\\
Global memory & & 36 Gfloats/s & \hspace{30pt} 1 &:30 \\
Shared memory & & 257 Gfloats/s & 1 &:4 \\
Registers & & $>$1.5 Tfloats/s & $>$3 &:2
\end{tabular}
\caption{Nvidia Quadro 6000: Memory throughput, $B_{\lowercase{\text{mem}}}$, compared to arithmetic throughput, $B_\text{arith}$.}
\end{table}
\end{frame}

% \begin{frame}
% \frametitle{Experimental Setup}
% \framesubtitle{HISAS 1030}
% \begin{figure}[H]
% \graphicsAI<1>[drawing,width=0.6\linewidth]{gfx/hisas_on_hugin.jpg} %{../Docs/GeiloSrc/Figs/das.pdf}
% \end{figure}
% \begin{itemize}
% \item High resolution interferometric SAS
% \item 2x32 element phased array transmitter/receiver
% \item Array length: 120cm
% \item Sampling frequency: 100kHz
% \item Opening angle: 15deg (TX), 23deg (RX)
% \end{itemize}
% \end{frame}

% \section{Results}
% 
% \subsection{Image Quality}

\begin{frame}
\frametitle{Results (benchmarks)}
\framesubtitle{}
\vspace{-3pt}
\begin{itemize}
\item Test system: Intel Xeon quad core CPU, and a Nvidia Quadro 6000 GPU (equivalent to Geforce GTX480).
\end{itemize}
\begin{narrow}{-.5cm}{-.5cm}
\begin{figure}[H]\centering
\graphicsAI<1>[drawing,width=1\linewidth]{gfx/benchmark1_tagged.svg}
\end{figure}
\end{narrow}
\vspace{-10pt}
\begin{itemize}
\item GPU version $\approx$ 2 orders of magnitude faster than (plain) C.
\end{itemize}
\end{frame}


\section{Case study}
% \subsection{Realtime sectorcan imaging}

{
\setbeamertemplate{background}{\graphicsAI[width=\paperwidth]{gfx/use_case_sectorscan3.svg}}%
\begin{frame}
\vspace{-150pt}\hspace{-25pt}
\frametitle{\color{white}Realtime requirements?}
\framesubtitle{\vspace{-10pt}\color{white}Sectorscan imaging}
% \vspace{100pt}\ 
% \begin{figure}[H]
% \begin{narrow}{-1.6cm}{-1.6cm}
% \graphicsAI<1>[drawing,width=\linewidth]{gfx/use_case_sectorscan3.svg}
% \end{narrow}
% \end{figure}   
\end{frame}
}

{
\setbeamertemplate{background}{\graphicsAI[width=\paperwidth]{gfx/scenario_interference_lca.svg}}%
\begin{frame}
\vspace{-142pt}
\frametitle{\hspace{-.5cm}Alternative for very large systems}
\framesubtitle{\hspace{-.5cm}The Low Complexity\\\hspace{-.5cm}Adaptive\\\hspace{-.5cm}(LCA)\\\hspace{-.5cm}beamformer}
\end{frame}
}

\begin{frame}
\frametitle{Image Quality}
\framesubtitle{\scriptsize Shipwreck Holmengraa imaged with HISAS1030 from Kongsberg Maritime}
\vspace{-5pt}
\begin{figure}[H]
\begin{narrow}{-2cm}{-2cm}\centering
\graphicsAI<1>[width=0.85\linewidth]{gfx/plot_holmengraa_L16_Navg1.pdf}
\end{narrow}
\end{figure}
\vspace{-10pt}
\begin{itemize}
\item All adaptive beamformers improve edge definition and suppress noise compared to DAS.
\end{itemize}
\end{frame}

% \subsection{Benchmarks}




{
\setbeamertemplate{background}{\graphicsAI[width=\paperwidth]{gfx/AUVenv.png}}%
\begin{frame}
\vspace{-25pt}\frametitle{\color{white}Conclusion}
\framesubtitle{}
\vspace{-5pt}\color{white}
\begin{itemize}
\item \color{white}GPUs are well suited to accelerate algorithms such as the MVDR beamformer
\item \color{white}Our GPU implementation of the MVDR beamformer
\begin{itemize}
\item \color{white}fully supports active sonar systems,
\item \color{white}easily outperforms CPU implementations, and
\item \color{white}can run in beamspace mode to maintain imaging speeds for large systems.
\end{itemize}
\item \color{white}LCA: A lightweight and easily implementable alternative for very large systems.
\end{itemize}
\vfill
\textbf{\color{black}Acknowledgements}:
\begin{itemize}
\item Thanks to the Norwegian Defence Research Establishment (FFI) and Kongsberg Maritime for providing us with data.
\end{itemize}
\end{frame}
}

% 
% {
% \setbeamertemplate{background}{\graphicsAI[width=\paperwidth]{gfx/AUVenv.png}}%
% \begin{frame}
% \vspace{-15pt}\frametitle{\color{white}Conclusion}
% \framesubtitle{}
% \vspace{-5pt}\color{white}
% \begin{itemize}
% \item \color{white}GPUs are well suited to accelerate algorithms such as the MVDR beamformer
% \item \color{white}Our GPU implementation of the MVDR beamformer
% \begin{itemize}
% \item \color{white}fully supports active sonar systems,
% \item \color{white}easily outperforms CPU implementations, and
% \item \color{white}runs fast enough to support realtime sectorscan imaging.
% \end{itemize}
% \end{itemize}
% \begin{figure}[H]%[width=0.5\linewidth]
% \graphicsAI<1>[width=0.3\linewidth]{gfx/happy_fish.png}\\
% \color{black}\centering\LARGE\textbf{Questions?}
% \end{figure}
% % \vspace{-10pt}
% \end{frame}
% }
% 
% % \begin{frame}
% % \frametitle{}
% % \framesubtitle{}
% % 
% % \begin{figure}[H]%[width=0.5\linewidth]
% % \graphicsAI<1>[width=0.3\linewidth]{gfx/happy_fish.png}
% % \end{figure}
% % \vspace{-10pt}
% % \centering\LARGE\textbf{Questions?}
% % 
% % \end{frame}
% 


\end{document}




% \documentclass[
%   ucs,
%   utf8
%   12pt,                   %extsize
%   draft,                  %for empty decorations
%     hyperref={
%       breaklinks=true,      %Break links when necessary
%       linktocpage=true,     %Enable link to page?
%       linkcolor=PineGreen,  %Colour of links to labels within document
%       citecolor=Brown,      %Colour of links to the biliography
%       filecolor=Red,        %Colour of links to local files
%       pagecolor=Red,        %Colour of links to other pages withing document
%       urlcolor=Blue,        %Colour of the links to external URLs
%       colorlinks=true,      %??
%       plainpages=false,     %Store roman/arabic numbering differently to avoid
%       bookmarksnumbered
%     },                      %``duplicate'' warning.
% %   usepdftitle={},         %
%     xcolor={
%       table,                %For colour in tabulars (will pull in colortbl)
%       dvipsnames,
%       svgnames
%     }
%   c,                      %centered
%   t,                      %top aligned
%   compress,               %
%   trans,                  %
%   noamsthm,               %
%   notheorems,             %
%   envcountsec,            %
%   ignoreonframetext,      %
%   handout,                %
%   notes={}                %
% ]{beamer}
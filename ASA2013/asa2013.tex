
%===============%                               %~~~~~~~~~~~~~~~~~~~~~~~~~~~~~~~~~~~~~~~~~~~~~~~~~%
% DOCUMENTCLASS %                                See full option description in "mytemplate.cls"
%===============%                               %~~~~~~~~~~~~~~~~~~~~~~~~~~~~~~~~~~~~~~~~~~~~~~~~~%
 
\documentclass[
   article                                      % Document type (non-standard)
 , 12pt                                         % Text size
%  , draft
%  , final                                        % Quality
 , xelatex                                      % Use the XeLaTeX compiler
%  , biblatex                                     % Use 'biblatex' for references (best by far)
 , bibtex                                       % Use 'bibtex' for references (oldschool :)
%  , movie
%  , notodos                                      % Disable todos
 , layout
%  , defaultformat                                % Attempt to use default formatting options
%  , glossary                                     % Use a glossary
]{common/mytemplate}

% \bibliography{references}
\hypersetup{
   bookmarksopen=true
 , bookmarksopenlevel=2
}
% Load the references.bib file
% \bibliography{references}



\begin{document}
% \setlength{\headrulewidth}{0.0pt}

%~~ TitlePage ~~%                               %~~~~~~~~~~~~~~~~~~~~~~~~~~~~~~~~~~~~~~~~~~~~~~~~~%
\thispagestyle{empty}
\pagenumbering{arabic} %Normal numbers

\begin{abstract}
The use of adaptive beamformers in active sonar use of a Minimum Variance (MV) beamforming have demostrated 

The quality of active sonar images can be improved by the use of a Minimum Variance (MV) beamformer. 

Several studies have shown that using a Minimum Variance (MV) beamformer when forming active sonar images

The Minimum Variance Distortionless Reponse (MVDR) beamformer has lately been applied to active sonar imaging. It displays a potential for improving the image quality beyond the limits of conventional methods, but rely on the estimation and inversion of a spatial covariance matrix. This translates to a complexity that is cubic with the number of elements $M$, O($M^3$), major complexity involved,  its use is discouraged due to its major computational complexity and need for robustification. This is because it involves the estimation and inversion of a spatial covariance matrix, 


Its ability to be It is reported to improve . In many scenarios it offers Its  can be applied to active sonar imaging, and  to image quality of active sonar systems   beamformer have in recent studies displayed shown to improve image quality in active sonar systems.

  is an adaptive technique with the potential to improve active sonar imaging. One such method is the Minimum Variance Distorionless 

Their  beamforming active sonar imaging 
Adaptive beamformers have recently been introduced in active sonar imaging 
- If algorithm is slow - What technology will help me?
- What skills are needed?
- What tools are available and what's the cost?
- What acceleration factor are we looking at?
- How can I get started?

- What we did. Which tools? Was it hard? Speedup?
\end{abstract}


\section{Introduction}

\end{document}


%===============%                               %~~~~~~~~~~~~~~~~~~~~~~~~~~~~~~~~~~~~~~~~~~~~~~~~~%
% DOCUMENTCLASS %                                See full option description in "mytemplate.cls"
%===============%                               %~~~~~~~~~~~~~~~~~~~~~~~~~~~~~~~~~~~~~~~~~~~~~~~~~%
 
\documentclass[
   article                                      % Document type (non-standard)
 , 12pt                                         % Text size
%  , draft
%  , final                                        % Quality
 , xelatex                                      % Use the XeLaTeX compiler
%  , biblatex                                     % Use 'biblatex' for references (best by far)
 , bibtex                                       % Use 'bibtex' for references (oldschool :)
%  , movie
%  , notodos                                      % Disable todos
 , layout
%  , defaultformat                                % Attempt to use default formatting options
%  , glossary                                     % Use a glossary
]{common/mytemplate}

% \bibliography{references}
\hypersetup{
   bookmarksopen=true
 , bookmarksopenlevel=2
}
% Load the references.bib file
% \bibliography{references}



\begin{document}
% \setlength{\headrulewidth}{0.0pt}

%~~ TitlePage ~~%                               %~~~~~~~~~~~~~~~~~~~~~~~~~~~~~~~~~~~~~~~~~~~~~~~~~%
\thispagestyle{empty}
\pagenumbering{arabic} %Normal numbers

\begin{abstract}
The Minimum Variance Distortionless Reponse (MVDR) beamformer has lately been applied to active sonar imaging. While displaying an ability to improve the image quality beyond the limits of conventional beamformers, it relies on the estimation and inversion of a spatial covariance matrix for each the image pixels. This may discourage its use, since robustification methods must be introduced to ensure numerical and statistical validity\todo{denne henger}, and the computational complexity is cubic with the number of elements, O($M^3$).

We will show that for systems with less than 32 channels, the estimation process is the main bottleneck. In this case, we suggest arithmetic optimizations for the estimation step, and explain how the MVDR can be mapped to a high-end GPU to yield processing rates of more than 1\,Mpx/s. When $M$ exceeds 32, we show that frequency domain processing can be employed instead, which allows a comparable performance to be maintained, at a neglible reduction in image quality. Compared to a reference C and Matlab implementation, the GPU implementation offer a speedup of 2-3 orders of magnitude, but is not a trivial design.

An alternative to the MVDR is the Low Complexity Adaptive beamformer (LCA). It does not compute a weightset, but merely computes the beamformer output for a range of predefined weight sets, and selects the one that best fulfills the MVDR optimization criteria. We will show that with little effort the LCA can produce images comparable to the MVDR, with hardly any development cost or complexity, and in a way that is extremely well suited for the GPU. 

\ \\
Total: 258 words (200 max)
\ \\
Questions they wanted answered:
\begin{itemize}
\item If algorithm is slow - What technology will help me?
\item What skills are needed?
\item What tools are available and what's the cost?
\item What acceleration factor are we looking at?
\item How can I get started?
\item What we did. Which tools? Was it hard? Speedup?
\end{itemize}
\end{abstract}


% \section{Introduction}

\end{document}


%===============%                               %~~~~~~~~~~~~~~~~~~~~~~~~~~~~~~~~~~~~~~~~~~~~~~~~~%
% DOCUMENTCLASS %                                See full option description in "mytemplate.cls"
%===============%                               %~~~~~~~~~~~~~~~~~~~~~~~~~~~~~~~~~~~~~~~~~~~~~~~~~%
 
\documentclass[
   article                                      % Document type (non-standard)
 , 12pt                                         % Text size
%  , draft
%  , final                                        % Quality
 , xelatex                                      % Use the XeLaTeX compiler
%  , biblatex                                     % Use 'biblatex' for references (best by far)
 , bibtex                                       % Use 'bibtex' for references (oldschool :)
%  , movie
%  , notodos                                      % Disable todos
 , layout
%  , defaultformat                                % Attempt to use default formatting options
%  , glossary                                     % Use a glossary
]{common/mytemplate}

% \bibliography{references}
\hypersetup{
   bookmarksopen=true
 , bookmarksopenlevel=2
}
% Load the references.bib file
% \bibliography{references}



\begin{document}
% \setlength{\headrulewidth}{0.0pt}

%~~ TitlePage ~~%                               %~~~~~~~~~~~~~~~~~~~~~~~~~~~~~~~~~~~~~~~~~~~~~~~~~%
\thispagestyle{empty}
\pagenumbering{arabic} %Normal numbers

\begin{abstract}
The Minimum Variance Distortionless Reponse (MVDR) beamformer has lately been applied to active sonar imaging. While displaying an ability to improve the image quality beyond the limits of conventional beamformers, it relies on the estimation and inversion of a spatial covariance matrix for each the image pixels. This may discourage its use, since robustification methods must be introduced to ensure numerical and statistical validity\todo{denne tror jeg vi kutter}, and the computational complexity is cubic with the number of channels, O($M^3$).

For systems with less than 32 channels, we suggest arithmetic optimizations for the estimation step, and explain how the MVDR can be mapped to a high-end GPU to yield processing rates of more than 1\,Mpx/s. When $M$ exceeds 32, we show that frequency domain processing can be employed instead, which allows a comparable performance to be maintained, at a neglible reduction in image quality. Compared to our reference C and Matlab implementation, the GPU implementation offered a speedup of 2-3 orders of magnitude, but at the expense of a non-trivial design.

A very good alternative to the MVDR is the Low Complexity Adaptive beamformer (LCA). It does not compute a weightset, but merely computes the beamformer output for each of a each predefined set of weights, and selects the one that best fulfills the MVDR criteria. The images it produces are comparable to the MVDR, without the development cost or complexity of the MVDR, and it is perfectly suited for a GPU. 

\ \\
Total: 239 words (200 max)

\ \\
Noen tanker jeg har selv nå:
\begin{itemize}
\item Dette blir mye beamforming, beamforming, beamforming. Mulig jeg må riste av meg ``JOE formen'' og fokusere litt mer på implementasjonsaspektet.
\item Det er mye ord, og litt muntlig. Det går seg til så snart jeg har bestemt meg for innhold.
\item Det mangler noe om bruksområder.
\item Må passe på å ikke selge LCA så bra at MVDR ser helt håpløs ut(?)
\item 200 ord er ingenting. :)
\end{itemize}


\ \\
Questions they wanted answered:
\begin{itemize}
\item This session will feature work on adapting computationally demanding modeling and signal processing tasks to take advantage of various parallel architectures, such as cloud computing, computing clusters, multicore CPUs, multi-core graphic processing units (GPUs), DSP chips, and FPGAs.
\item If algorithm is slow - What technology will help me?
\item What skills are needed?
\item What tools are available and what's the cost?
\item What acceleration factor are we looking at?
\item How can I get started?
\item What we did. Which tools? Was it hard? Speedup?
\end{itemize}
\end{abstract}


% \section{Introduction}

\end{document}


%===============%                               %~~~~~~~~~~~~~~~~~~~~~~~~~~~~~~~~~~~~~~~~~~~~~~~~~%
% DOCUMENTCLASS %                                See full option description in "mytemplate.cls"
%===============%                               %~~~~~~~~~~~~~~~~~~~~~~~~~~~~~~~~~~~~~~~~~~~~~~~~~%
 
\documentclass[
   article                                      % Document type (non-standard)
 , 12pt                                         % Text size
%  , draft
%  , final                                        % Quality
 , xelatex                                      % Use the XeLaTeX compiler
%  , biblatex                                     % Use 'biblatex' for references (best by far)
 , bibtex                                       % Use 'bibtex' for references (oldschool :)
%  , movie
%  , notodos                                      % Disable todos
 , layout
%  , defaultformat                                % Attempt to use default formatting options
%  , glossary                                     % Use a glossary
]{common/mytemplate}

% \bibliography{references}
\hypersetup{
   bookmarksopen=true
 , bookmarksopenlevel=2
}
% Load the references.bib file
% \bibliography{references}



\begin{document}
% \setlength{\headrulewidth}{0.0pt}

%~~ TitlePage ~~%                               %~~~~~~~~~~~~~~~~~~~~~~~~~~~~~~~~~~~~~~~~~~~~~~~~~%
\thispagestyle{empty}
\pagenumbering{arabic} %Normal numbers

\begin{abstract}
The MVDR adaptive beamformer has lately been shown to improve image quality in active sonar systems, but at a major computational expense. This is because the estimation and inversion of a spatial covariance matrix is required for each image pixel. We target this problem by altering and mapping the MVDR algorithm GPU, and suggest three different solutions depending on the system size.

For systems with less than 32 channels, we suggest arithmetic optimizations for the estimation step, and show how a careful mapping to a high-end GPU to yield processing rates of more than 1\,Mpx/s. When the system exceeds this size, we show that frequency domain processing can be employed instead, which allows a comparable performance to be maintained, at a neglible reduction in image quality. These GPU implementations consistently reduced the runtime by 2-3 orders of magnitude compared to our reference C and Matlab implementations.

Should this this still be insufficient, we suggest having a look that the LCA beamformer. It does not compute a weightset, but merely computes the beamformer output for each of a each predefined set of weights, and selects the one that best fulfills the MVDR criteria. The images it produces are comparable to the MVDR, without the development cost or complexity of the MVDR, and it is perfectly suited for a GPU. 

\ \\
Total: 215 words (200 max)

\ \\
Noen tanker jeg har selv nå:
\begin{itemize}
\item Det mangler noe om bruksområder.
\item Må passe på å ikke selge LCA så bra at MVDR ser helt håpløs ut(?)
\item 200 ord er ingenting. :)
\end{itemize}


\ \\
Questions they wanted answered:
\begin{itemize}
\item This session will feature work on adapting computationally demanding modeling and signal processing tasks to take advantage of various parallel architectures, such as cloud computing, computing clusters, multicore CPUs, multi-core graphic processing units (GPUs), DSP chips, and FPGAs.
\item If algorithm is slow - What technology will help me?
\item What skills are needed?
\item What tools are available and what's the cost?
\item What acceleration factor are we looking at?
\item How can I get started?
\item What we did. Which tools? Was it hard? Speedup?
\end{itemize}
\end{abstract}


% \section{Introduction}

\end{document}

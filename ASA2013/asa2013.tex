
%===============%                               %~~~~~~~~~~~~~~~~~~~~~~~~~~~~~~~~~~~~~~~~~~~~~~~~~%
% DOCUMENTCLASS %                                See full option description in "mytemplate.cls"
%===============%                               %~~~~~~~~~~~~~~~~~~~~~~~~~~~~~~~~~~~~~~~~~~~~~~~~~%
 
\documentclass[
   article                                      % Document type (non-standard)
 , 12pt                                         % Text size
%  , draft
%  , final                                        % Quality
 , xelatex                                      % Use the XeLaTeX compiler
%  , biblatex                                     % Use 'biblatex' for references (best by far)
 , bibtex                                       % Use 'bibtex' for references (oldschool :)
%  , movie
%  , notodos                                      % Disable todos
 , layout
%  , defaultformat                                % Attempt to use default formatting options
%  , glossary                                     % Use a glossary
]{common/mytemplate}

% \bibliography{references}
\hypersetup{
   bookmarksopen=true
 , bookmarksopenlevel=2
}
% Load the references.bib file
% \bibliography{references}



\begin{document}
% \setlength{\headrulewidth}{0.0pt}

%~~ TitlePage ~~%                               %~~~~~~~~~~~~~~~~~~~~~~~~~~~~~~~~~~~~~~~~~~~~~~~~~%
\thispagestyle{empty}
\pagenumbering{arabic} %Normal numbers

\begin{abstract}
The Minimum Variance Distortionless Reponse (MVDR) beamformer has lately been applied to active sonar imaging. While displaying an ability to improve the image quality beyond the limits of conventional beamformers, it relies on the estimation and inversion of a spatial covariance matrix for each the image pixels. This may discourage its use, since robustification methods must be introduced to ensure numerical and statistical validity, and the computational complexity is cubic with the number of elements, O($M^3$).

We will show that for systems with less than $M$=32 channels, the estimation process is the main bottleneck. In this case, we suggest arithmetic optimizations for the estimation step, and explain how the the MVDR can be mapped to a high-end GPU to yield processing rates of more than 1\,Mpx/s. When $M$ exceeds 32, the inversion step dominates, and 


This represents a speedup of 2-3 orders of magnitude over a moderately optimized C and Matlab reference implementation. For higher channel systems, we employ MVDR beamspace processing on the GPU


 With a channel count that exceeds 32, we show that we switch to using frequency based methods we run on a some algorithmic optimizations and porting of the  to a GPU 

Depending on the number of channels in the system, we target these issues in 3 different ways.

  The computational complexity involved, as well as the need to introduce robustification techniques to make  This may discurage its use, since the estimation process requires  process The computational complexity involved, and the means required to make the well suited, 


Obtaining a robust estimate This translates to a complexity that is cubic with the number of elements $M$, O($M^3$), major complexity involved,  its use is discouraged due to its major computational complexity and need for robustification. This is because it involves the estimation and inversion of a spatial covariance matrix, 


Its ability to be It is reported to improve . In many scenarios it offers Its  can be applied to active sonar imaging, and  to image quality of active sonar systems   beamformer have in recent studies displayed shown to improve image quality in active sonar systems.

  is an adaptive technique with the potential to improve active sonar imaging. One such method is the Minimum Variance Distorionless 

Their  beamforming active sonar imaging 
Adaptive beamformers have recently been introduced in active sonar imaging 
- If algorithm is slow - What technology will help me?
- What skills are needed?
- What tools are available and what's the cost?
- What acceleration factor are we looking at?
- How can I get started?

- What we did. Which tools? Was it hard? Speedup?
\end{abstract}


% \section{Introduction}

\end{document}

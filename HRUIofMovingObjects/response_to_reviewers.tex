\documentclass{article}

\begin{document}

Answers to the reviewers in \textit{italic}.

\section*{Reviewer: 1}

As my scores indicate, I think the paper addresses an interesting problem and does it very well. Beside two VERY minor typos, I think it can be published unaltered.
1. "In" missing at the beginning of the Introduction section.
2. In section IIB, the opening sentence, there is an 's' missing in 'produce'.

\textit{We want to thank the reviewer for an encouraging review. Typos have been corrected.}

\section*{Reviewer: 2}

Thanks to the authors for addressing the problem of the temporal aspects in minimum variance beamforming, demonstrating the extra requirements on spatial sampling, and demonstrating how to efficiently oversample. The paper is good but I would like it to go through a second review round based on the following comments:
\\\\
Page2 column2 lines 34-35, and Fig. 1, L, M and d have not been introduced (need to have read the previous literature or equation (4) to know their meaning). I would suggest that in the introduction we do not need these details.
\\\\
Equation (8) the two factors p and q are confusing. Because q is always one, I would suggest omitting it.  Let us say that the oversampling is used both for Capon and for reducing folding of frequencies over the bandwidth.
\\\\
At what depth are the LSV plots taken with respect to the focal depth? Does it matter?
\\\\
Please specify the characteristics of the M5S probe (size, number of elements) if GE permits. Is it a phased or a linear array? What is the physical beam spacing, the field of view, the number of transmit and receive beams in the base (non oversampled) configuration? What is the transmit focal depth? Are transmit and receive apodization applied? Without this information, the work is not reproductible.
\\\\
In section “IV-A. Oversampling in transmit”: please confirm (specify) that 1x1 receive beamforming is performed at this point.
\\\\
In section “IV-B. Oversampling by parallel receive beamforming”, could you explain in more details what are the differences between Capon + STB and the technique proposed by Rabinovitch (2013) – reference [27]?
\\\\
Page10 Column1 line56  write “should be removed” instead of “should be remove”
\\\\
Page 10 Column 1 lines 40-60, you state that with synthetic transmit beamforming, some of the artifacts seen in oversampling in receive by phase rotation could be eliminated. It would be interesting to see such technique implemented.
\\\\
P10 column 2 line 28-29 write “too extreme” instead of “to extreme”
\\\\
Fig 6a: specify which beamforming is applied here (I suspect delay and sum with no multiline receive?). Maybe you want to consider showing still images of Capon and Capon-oversampled as well (not necessarily on fig. 6), just in case some people won’t look at the videos.

\end{document}
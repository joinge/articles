
%% bare_jrnl.tex
%% V1.3
%% 2007/01/11
%% by Michael Shell
%% see http://www.michaelshell.org/
%% for current contact information.
%%
%% This is a skeleton file demonstrating the use of IEEEtran.cls
%% (requires IEEEtran.cls version 1.7 or later) with an IEEE journal paper.
%%
%% Support sites:
%% http://www.michaelshell.org/tex/ieeetran/
%% http://www.ctan.org/tex-archive/macros/latex/contrib/IEEEtran/
%% and
%% http://www.ieee.org/



% *** Authors should verify (and, if needed, correct) their LaTeX system  ***
% *** with the testflow diagnostic prior to trusting their LaTeX platform ***
% *** with production work. IEEE's font choices can trigger bugs that do  ***
% *** not appear when using other class files.                            ***
% The testflow support page is at:
% http://www.michaelshell.org/tex/testflow/


%%*************************************************************************
%% Legal Notice:
%% This code is offered as-is without any warranty either expressed or
%% implied; without even the implied warranty of MERCHANTABILITY or
%% FITNESS FOR A PARTICULAR PURPOSE! 
%% User assumes all risk.
%% In no event shall IEEE or any contributor to this code be liable for
%% any damages or losses, including, but not limited to, incidental,
%% consequential, or any other damages, resulting from the use or misuse
%% of any information contained here.
%%
%% All comments are the opinions of their respective authors and are not
%% necessarily endorsed by the IEEE.
%%
%% This work is distributed under the LaTeX Project Public License (LPPL)
%% ( http://www.latex-project.org/ ) version 1.3, and may be freely used,
%% distributed and modified. A copy of the LPPL, version 1.3, is included
%% in the base LaTeX documentation of all distributions of LaTeX released
%% 2003/12/01 or later.
%% Retain all contribution notices and credits.
%% ** Modified files should be clearly indicated as such, including  **
%% ** renaming them and changing author support contact information. **
%%
%% File list of work: IEEEtran.cls, IEEEtran_HOWTO.pdf, bare_adv.tex,
%%                    bare_conf.tex, bare_jrnl.tex, bare_jrnl_compsoc.tex
%%*************************************************************************

% Note that the a4paper option is mainly intended so that authors in
% countries using A4 can easily print to A4 and see how their papers will
% look in print - the typesetting of the document will not typically be
% affected with changes in paper size (but the bottom and side margins will).
% Use the testflow package mentioned above to verify correct handling of
% both paper sizes by the user's LaTeX system.
%
% Also note that the "draftcls" or "draftclsnofoot", not "draft", option
% should be used if it is desired that the figures are to be displayed in
% draft mode.
%
\documentclass[journal]{IEEEtran}
%
% If IEEEtran.cls has not been installed into the LaTeX system files,
% manually specify the path to it like:
% \documentclass[journal]{../sty/IEEEtran}





% Some very useful LaTeX packages include:
% (uncomment the ones you want to load)


% *** MISC UTILITY PACKAGES ***
%
%\usepackage{ifpdf}
% Heiko Oberdiek's ifpdf.sty is very useful if you need conditional
% compilation based on whether the output is pdf or dvi.
% usage:
% \ifpdf
%   % pdf code
% \else
%   % dvi code
% \fi
% The latest version of ifpdf.sty can be obtained from:
% http://www.ctan.org/tex-archive/macros/latex/contrib/oberdiek/
% Also, note that IEEEtran.cls V1.7 and later provides a builtin
% \ifCLASSINFOpdf conditional that works the same way.
% When switching from latex to pdflatex and vice-versa, the compiler may
% have to be run twice to clear warning/error messages.






% *** CITATION PACKAGES ***
%
\usepackage{cite}
% cite.sty was written by Donald Arseneau
% V1.6 and later of IEEEtran pre-defines the format of the cite.sty package
% \cite{} output to follow that of IEEE. Loading the cite package will
% result in citation numbers being automatically sorted and properly
% "compressed/ranged". e.g., [1], [9], [2], [7], [5], [6] without using
% cite.sty will become [1], [2], [5]--[7], [9] using cite.sty. cite.sty's
% \cite will automatically add leading space, if needed. Use cite.sty's
% noadjust option (cite.sty V3.8 and later) if you want to turn this off.
% cite.sty is already installed on most LaTeX systems. Be sure and use
% version 4.0 (2003-05-27) and later if using hyperref.sty. cite.sty does
% not currently provide for hyperlinked citations.
% The latest version can be obtained at:
% http://www.ctan.org/tex-archive/macros/latex/contrib/cite/
% The documentation is contained in the cite.sty file itself.






% *** GRAPHICS RELATED PACKAGES ***
%
\ifCLASSINFOpdf
  % \usepackage[pdftex]{graphicx}
  % declare the path(s) where your graphic files are
  % \graphicspath{{../pdf/}{../jpeg/}}
  % and their extensions so you won't have to specify these with
  % every instance of \includegraphics
  % \DeclareGraphicsExtensions{.pdf,.jpeg,.png}
\else
  % or other class option (dvipsone, dvipdf, if not using dvips). graphicx
  % will default to the driver specified in the system graphics.cfg if no
  % driver is specified.
  % \usepackage[dvips]{graphicx}
  % declare the path(s) where your graphic files are
  % \graphicspath{{../eps/}}
  % and their extensions so you won't have to specify these with
  % every instance of \includegraphics
  % \DeclareGraphicsExtensions{.eps}
\fi
% graphicx was written by David Carlisle and Sebastian Rahtz. It is
% required if you want graphics, photos, etc. graphicx.sty is already
% installed on most LaTeX systems. The latest version and documentation can
% be obtained at: 
% http://www.ctan.org/tex-archive/macros/latex/required/graphics/
% Another good source of documentation is "Using Imported Graphics in
% LaTeX2e" by Keith Reckdahl which can be found as epslatex.ps or
% epslatex.pdf at: http://www.ctan.org/tex-archive/info/
%
% latex, and pdflatex in dvi mode, support graphics in encapsulated
% postscript (.eps) format. pdflatex in pdf mode supports graphics
% in .pdf, .jpeg, .png and .mps (metapost) formats. Users should ensure
% that all non-photo figures use a vector format (.eps, .pdf, .mps) and
% not a bitmapped formats (.jpeg, .png). IEEE frowns on bitmapped formats
% which can result in "jaggedy"/blurry rendering of lines and letters as
% well as large increases in file sizes.
%
% You can find documentation about the pdfTeX application at:
% http://www.tug.org/applications/pdftex





% *** MATH PACKAGES ***
%
%\usepackage[cmex10]{amsmath}
% A popular package from the American Mathematical Society that provides
% many useful and powerful commands for dealing with mathematics. If using
% it, be sure to load this package with the cmex10 option to ensure that
% only type 1 fonts will utilized at all point sizes. Without this option,
% it is possible that some math symbols, particularly those within
% footnotes, will be rendered in bitmap form which will result in a
% document that can not be IEEE Xplore compliant!
%
% Also, note that the amsmath package sets \interdisplaylinepenalty to 10000
% thus preventing page breaks from occurring within multiline equations. Use:
%\interdisplaylinepenalty=2500
% after loading amsmath to restore such page breaks as IEEEtran.cls normally
% does. amsmath.sty is already installed on most LaTeX systems. The latest
% version and documentation can be obtained at:
% http://www.ctan.org/tex-archive/macros/latex/required/amslatex/math/





% *** SPECIALIZED LIST PACKAGES ***
%
%\usepackage{algorithmic}
% algorithmic.sty was written by Peter Williams and Rogerio Brito.
% This package provides an algorithmic environment fo describing algorithms.
% You can use the algorithmic environment in-text or within a figure
% environment to provide for a floating algorithm. Do NOT use the algorithm
% floating environment provided by algorithm.sty (by the same authors) or
% algorithm2e.sty (by Christophe Fiorio) as IEEE does not use dedicated
% algorithm float types and packages that provide these will not provide
% correct IEEE style captions. The latest version and documentation of
% algorithmic.sty can be obtained at:
% http://www.ctan.org/tex-archive/macros/latex/contrib/algorithms/
% There is also a support site at:
% http://algorithms.berlios.de/index.html
% Also of interest may be the (relatively newer and more customizable)
% algorithmicx.sty package by Szasz Janos:
% http://www.ctan.org/tex-archive/macros/latex/contrib/algorithmicx/




% *** ALIGNMENT PACKAGES ***
%
%\usepackage{array}
% Frank Mittelbach's and David Carlisle's array.sty patches and improves
% the standard LaTeX2e array and tabular environments to provide better
% appearance and additional user controls. As the default LaTeX2e table
% generation code is lacking to the point of almost being broken with
% respect to the quality of the end results, all users are strongly
% advised to use an enhanced (at the very least that provided by array.sty)
% set of table tools. array.sty is already installed on most systems. The
% latest version and documentation can be obtained at:
% http://www.ctan.org/tex-archive/macros/latex/required/tools/


%\usepackage{mdwmath}
%\usepackage{mdwtab}
% Also highly recommended is Mark Wooding's extremely powerful MDW tools,
% especially mdwmath.sty and mdwtab.sty which are used to format equations
% and tables, respectively. The MDWtools set is already installed on most
% LaTeX systems. The lastest version and documentation is available at:
% http://www.ctan.org/tex-archive/macros/latex/contrib/mdwtools/


% IEEEtran contains the IEEEeqnarray family of commands that can be used to
% generate multiline equations as well as matrices, tables, etc., of high
% quality.


%\usepackage{eqparbox}
% Also of notable interest is Scott Pakin's eqparbox package for creating
% (automatically sized) equal width boxes - aka "natural width parboxes".
% Available at:
% http://www.ctan.org/tex-archive/macros/latex/contrib/eqparbox/





% *** SUBFIGURE PACKAGES ***
%\usepackage[tight,footnotesize]{subfigure}
% subfigure.sty was written by Steven Douglas Cochran. This package makes it
% easy to put subfigures in your figures. e.g., "Figure 1a and 1b". For IEEE
% work, it is a good idea to load it with the tight package option to reduce
% the amount of white space around the subfigures. subfigure.sty is already
% installed on most LaTeX systems. The latest version and documentation can
% be obtained at:
% http://www.ctan.org/tex-archive/obsolete/macros/latex/contrib/subfigure/
% subfigure.sty has been superceeded by subfig.sty.



%\usepackage[caption=false]{caption}
%\usepackage[font=footnotesize]{subfig}
% subfig.sty, also written by Steven Douglas Cochran, is the modern
% replacement for subfigure.sty. However, subfig.sty requires and
% automatically loads Axel Sommerfeldt's caption.sty which will override
% IEEEtran.cls handling of captions and this will result in nonIEEE style
% figure/table captions. To prevent this problem, be sure and preload
% caption.sty with its "caption=false" package option. This is will preserve
% IEEEtran.cls handing of captions. Version 1.3 (2005/06/28) and later 
% (recommended due to many improvements over 1.2) of subfig.sty supports
% the caption=false option directly:
\usepackage[caption=false,font=footnotesize]{subfig}
%
% The latest version and documentation can be obtained at:
% http://www.ctan.org/tex-archive/macros/latex/contrib/subfig/
% The latest version and documentation of caption.sty can be obtained at:
% http://www.ctan.org/tex-archive/macros/latex/contrib/caption/




% *** FLOAT PACKAGES ***
%
%\usepackage{fixltx2e}
% fixltx2e, the successor to the earlier fix2col.sty, was written by
% Frank Mittelbach and David Carlisle. This package corrects a few problems
% in the LaTeX2e kernel, the most notable of which is that in current
% LaTeX2e releases, the ordering of single and double column floats is not
% guaranteed to be preserved. Thus, an unpatched LaTeX2e can allow a
% single column figure to be placed prior to an earlier double column
% figure. The latest version and documentation can be found at:
% http://www.ctan.org/tex-archive/macros/latex/base/



%\usepackage{stfloats}
% stfloats.sty was written by Sigitas Tolusis. This package gives LaTeX2e
% the ability to do double column floats at the bottom of the page as well
% as the top. (e.g., "\begin{figure*}[!b]" is not normally possible in
% LaTeX2e). It also provides a command:
%\fnbelowfloat
% to enable the placement of footnotes below bottom floats (the standard
% LaTeX2e kernel puts them above bottom floats). This is an invasive package
% which rewrites many portions of the LaTeX2e float routines. It may not work
% with other packages that modify the LaTeX2e float routines. The latest
% version and documentation can be obtained at:
% http://www.ctan.org/tex-archive/macros/latex/contrib/sttools/
% Documentation is contained in the stfloats.sty comments as well as in the
% presfull.pdf file. Do not use the stfloats baselinefloat ability as IEEE
% does not allow \baselineskip to stretch. Authors submitting work to the
% IEEE should note that IEEE rarely uses double column equations and
% that authors should try to avoid such use. Do not be tempted to use the
% cuted.sty or midfloat.sty packages (also by Sigitas Tolusis) as IEEE does
% not format its papers in such ways.


%\ifCLASSOPTIONcaptionsoff
%  \usepackage[nomarkers]{endfloat}
% \let\MYoriglatexcaption\caption
% \renewcommand{\caption}[2][\relax]{\MYoriglatexcaption[#2]{#2}}
%\fi
% endfloat.sty was written by James Darrell McCauley and Jeff Goldberg.
% This package may be useful when used in conjunction with IEEEtran.cls'
% captionsoff option. Some IEEE journals/societies require that submissions
% have lists of figures/tables at the end of the paper and that
% figures/tables without any captions are placed on a page by themselves at
% the end of the document. If needed, the draftcls IEEEtran class option or
% \CLASSINPUTbaselinestretch interface can be used to increase the line
% spacing as well. Be sure and use the nomarkers option of endfloat to
% prevent endfloat from "marking" where the figures would have been placed
% in the text. The two hack lines of code above are a slight modification of
% that suggested by in the endfloat docs (section 8.3.1) to ensure that
% the full captions always appear in the list of figures/tables - even if
% the user used the short optional argument of \caption[]{}.
% IEEE papers do not typically make use of \caption[]'s optional argument,
% so this should not be an issue. A similar trick can be used to disable
% captions of packages such as subfig.sty that lack options to turn off
% the subcaptions:
% For subfig.sty:
% \let\MYorigsubfloat\subfloat
% \renewcommand{\subfloat}[2][\relax]{\MYorigsubfloat[]{#2}}
% For subfigure.sty:
% \let\MYorigsubfigure\subfigure
% \renewcommand{\subfigure}[2][\relax]{\MYorigsubfigure[]{#2}}
% However, the above trick will not work if both optional arguments of
% the \subfloat/subfig command are used. Furthermore, there needs to be a
% description of each subfigure *somewhere* and endfloat does not add
% subfigure captions to its list of figures. Thus, the best approach is to
% avoid the use of subfigure captions (many IEEE journals avoid them anyway)
% and instead reference/explain all the subfigures within the main caption.
% The latest version of endfloat.sty and its documentation can obtained at:
% http://www.ctan.org/tex-archive/macros/latex/contrib/endfloat/
%
% The IEEEtran \ifCLASSOPTIONcaptionsoff conditional can also be used
% later in the document, say, to conditionally put the References on a 
% page by themselves.





% *** PDF, URL AND HYPERLINK PACKAGES ***
%
%\usepackage{url}
% url.sty was written by Donald Arseneau. It provides better support for
% handling and breaking URLs. url.sty is already installed on most LaTeX
% systems. The latest version can be obtained at:
% http://www.ctan.org/tex-archive/macros/latex/contrib/misc/
% Read the url.sty source comments for usage information. Basically,
% \url{my_url_here}.





% *** Do not adjust lengths that control margins, column widths, etc. ***
% *** Do not use packages that alter fonts (such as pslatex).         ***
% There should be no need to do such things with IEEEtran.cls V1.6 and later.
% (Unless specifically asked to do so by the journal or conference you plan
% to submit to, of course. )


% correct bad hyphenation here
\hyphenation{op-tical net-works semi-conduc-tor}

%%% My packages
\usepackage{amsmath,epsfig}
\usepackage{amsfonts}
\usepackage{color}
\usepackage{epstopdf}

%%% My commands
\newcommand{\mat}[1]{\mathbf{#1}}
\renewcommand{\vec}[1]{\mathbf{#1}}
\newcommand{\R}{$\mat{\hat{R}}$ }

\newcommand{\img}{img/}
\epstopdfsetup{outdir=\img}

\newcommand\multimedia[1]{\textbf{{\color{red}#1}}}
\newcommand\comment[1]{\textit{{\color{red}(#1)}}}

\begin{document}
%
% paper title
% can use linebreaks \\ within to get better formatting as desired
\title{Capon Beamforming Applied on Moving Objects - An Analysis of Local Shift-Invariance}
%
%
% author names and IEEE memberships
% note positions of commas and nonbreaking spaces ( ~ ) LaTeX will not break
% a structure at a ~ so this keeps an author's name from being broken across
% two lines.
% use \thanks{} to gain access to the first footnote area
% a separate \thanks must be used for each paragraph as LaTeX2e's \thanks
% was not built to handle multiple paragraphs
%

\author{
   J. P. \AA{}sen, A. Austeng and S. Holm%\IEEEmembership{Member,~IEEE,}% <-this % stops a space
   
   \thanks{...}
}
%M. Shell is with the Department
%of Electrical and Computer Engineering, Georgia Institute of Technology, Atlanta,
%GA, 30332 USA e-mail: (see http://www.michaelshell.org/contact.html).}% <-this % stops a space
%\thanks{J. Doe and J. Doe are with Anonymous University.}% <-this % stops a space
%\thanks{Manuscript received April 19, 2005; revised January 11, 2007.}}

% note the % following the last \IEEEmembership and also \thanks - 
% these prevent an unwanted space from occurring between the last author name
% and the end of the author line. i.e., if you had this:
% 
% \author{....lastname \thanks{...} \thanks{...} }
%                     ^------------^------------^----Do not want these spaces!
%
% a space would be appended to the last name and could cause every name on that
% line to be shifted left slightly. This is one of those "LaTeX things". For
% instance, "\textbf{A} \textbf{B}" will typeset as "A B" not "AB". To get
% "AB" then you have to do: "\textbf{A}\textbf{B}"
% \thanks is no different in this regard, so shield the last } of each \thanks
% that ends a line with a % and do not let a space in before the next \thanks.
% Spaces after \IEEEmembership other than the last one are OK (and needed) as
% you are supposed to have spaces between the names. For what it is worth,
% this is a minor point as most people would not even notice if the said evil
% space somehow managed to creep in.



% The paper headers
\markboth{Transactions on Ultrasonics and Ferroelectrics and Frequency Control,~Vol.~x, No.~y, January~201\AA{}}%
{\AA{}sen \MakeLowercase{\textit{et al.}}: The Capon Adaptive Beamformer Applied on Moving Objects }
% The only time the second header will appear is for the odd numbered pages
% after the title page when using the twoside option.
% 
% *** Note that you probably will NOT want to include the author's ***
% *** name in the headers of peer review papers.                   ***
% You can use \ifCLASSOPTIONpeerreview for conditional compilation here if
% you desire.




% If you want to put a publisher's ID mark on the page you can do it like
% this:
%\IEEEpubid{0000--0000/00\$00.00~\copyright~2007 IEEE}
% Remember, if you use this you must call \IEEEpubidadjcol in the second
% column for its text to clear the IEEEpubid mark.



% use for special paper notices
%\IEEEspecialpapernotice{(Invited Paper)}




% make the title area
\maketitle


\begin{abstract}
%\boldmath
\comment{The abstract goes here.}
\end{abstract}
% IEEEtran.cls defaults to using nonbold math in the Abstract.
% This preserves the distinction between vectors and scalars. However,
% if the journal you are submitting to favors bold math in the abstract,
% then you can use LaTeX's standard command \boldmath at the very start
% of the abstract to achieve this. Many IEEE journals frown on math
% in the abstract anyway.

% Note that keywords are not normally used for peerreview papers.
\begin{IEEEkeywords}
Adaptive beamforming, Capon beamformer, Shift Invariance, Ultrasound Imaging.
\end{IEEEkeywords}






% For peer review papers, you can put extra information on the cover
% page as needed:
% \ifCLASSOPTIONpeerreview
% \begin{center} \bfseries EDICS Category: 3-BBND \end{center}
% \fi
%
% For peerreview papers, this IEEEtran command inserts a page break and
% creates the second title. It will be ignored for other modes.
\IEEEpeerreviewmaketitle



\section{Introduction}
% The very first letter is a 2 line initial drop letter followed
% by the rest of the first word in caps.
% 
% form to use if the first word consists of a single letter:
% \IEEEPARstart{A}{demo} file is ....
% 
% form to use if you need the single drop letter followed by
% normal text (unknown if ever used by IEEE):
% \IEEEPARstart{A}{}demo file is ....
% 
% Some journals put the first two words in caps:
% \IEEEPARstart{T}{his demo} file is ....
% 
% Here we have the typical use of a "T" for an initial drop letter
% and "HIS" in caps to complete the first word.

\IEEEPARstart{W}{ith} the introduction of a fast implementation of the Capon adaptive beamformer, a stream of images can now be processed at interactive frame rates \cite{Asen}. It is therefore an interesting question how a moving object appears after Capon beamforming has been applied. Earlier work on Capon beamforming for medical ultrasound imaging has investigated the method's capabilities in single frames only, and temporal behavior and effects has therefore been neglected. 

An important property of an ultrasound imaging system is its local shift-invariance. Local shift-invariance means that an object’s appearance should be preserved when the object, or the ultrasound probe, is moved slightly. This invariance for local movements is what makes ultrasound imaging feasible in general, however it is specially important for quantitative techniques that correlate time-varying features, like pulsed doppler and flow imaging, and speckle tracking of tissue and blood. In this paper we analyze the shift-invariance property of the Capon beamformer by studying the beamformer output when applied on simulated and \textit{in vitro} images of moving objects. 

Capon beamforming has been shown to provide increased lateral resolution and contrast in ultrasound images \cite{Synnevag2007, Synnevag2009, Chen2011}. The method accomplish this by adjusting the array weights based on element-to-element covariance. Hence, the method yields data dependent weights. Capon beamforming is however also known for its suboptimal performance if the sample size available for covariance estimation is limited \cite{Mestre2006}, if correlated signals are present \cite{Widrow1982}, and if the assumed steering vector does not match the signal propagation vector \cite{Wax1996, Wax1996a}. This has led to the development of several robust versions of the Capon beamformer with e.g. spatial smoothing \cite{Shan1985} or diagonal loading \cite{JianLi2003} of the covariance estimate, or by applying steering vector uncertainty sets (SVUS) \cite{Lorenz2005, Rubsamen2013}. The first two, together with temporal smoothing of the covariance estimate, have been applied in earlier work on Capon beamforming for medical ultrasound imaging \cite{Synnevag2009}. The first two trade resolution for robustness and temporal smoothing helps to preserve speckle statistics \cite{Synnevag2007a}. SVUS has not been applied, even though it is well known that the Capon beamformer exhibits severe signal-nulling if the signal is present during the covariance estimation process and if this estimate is later used with a steering vector that does not match the signal propagation vector exactly. This self-nulling actually contribute a lot to the method's super resolution capability in passive-narrowband applications. 

In \cite{Cox1973}, Cox derives formulas for the resolution of the Capon beamformer (called the $\vec{k}_3$ processor), and presents in great detail how the amount of mismatch between the selected steering vector and the signal propagation vector impacts the beamformer output in simple situations. Cox states the following important fact about the Capon beamformer (the $\vec{k}_3$ processor): 
\begin{quote}
"\textit{... a $\vec{k}_3$ processor requires more closely spaced beams than a ... $\vec{k}_1$-processor in order to avoid serious signal suppression effects being introduced on signals arriving from directions between the beams}", 
\end{quote}
where $\vec{k}_1$ is the delay-and-sum (DAS) beamformer. In Fig.\,\ref{fig:das_capon_beams} we have plotted the steered response for DAS and the Capon beamformer using (26) and (32) in \cite{Cox1973}. A 64-element linear array with $\lambda/2$ element spacing, where $\lambda$ is the wavelength, is scanned over a sector equal to three times the system resolution. Impinging on the array is a monochromatic plane wave with a propagation vector orthogonal to the array surface normal with no spatially white noise present. Hence, the interference-noise matrix ($\sigma_0^2\mat{Q}$) is the identity. We have also added the effect of typical subarray averaging and diagonal loading \cite{Asen} to the Capon beamformer output. The figure shows that when the DAS beamformer is used, only a 3.92 dB scallop loss \cite{Harris1978} should be observed if the source of the plane wave starts moving parallel to the array. However if the Capon beamformer is used with the beam density sufficient for DAS the scallop loss could be as large as $31$ dB. 

\begin{figure}[!t]
\centerline{
\includegraphics[width=\linewidth]{\img steerd_respons_capon_das_resolution.eps}
}
\caption{Comparison of drop in output power in between beams (half the system resolution) for delay-and-sum (DAS) and Capon beamforming.}
\label{fig:das_capon_beams}
\end{figure}

In previous literature on Capon beamforming for medical ultrasound imaging this effect has been passed over in silence (in single frames) by either oversampling on transmit, or carefully position point scatterers on transmit-receive beam pairs, or by ignorance. In this paper we show that when imaging a moving point scatterer with ultrasound and the Capon beamformer with the same beam spacing sufficient for DAS the same scallop loss is observed, manifested as oscillating point scatteres. The question is then, what can we do in order to improve on the situation without affecting the imaging frame rate.

\comment{Next sections summary.}

% needed in second column of first page if using \IEEEpubid
%\IEEEpubidadjcol

\section{Background}

The standard form of time domain array beamforming is the delay-and-sum beamformer (DAS):
\begin{align}\label{eq:das}
z[n] = \sum_{m = 0}^{M-1}\vec{w}_m^*\vec{x}_m[n - \Delta_m[n]] = \vec{w}^H\vec{x}[n],
\end{align}
where $M$ is the number of elements or channels in the array, $\Delta_m[n]$ is the per-element focusing and steering delay, and $\vec{w}$ is a weight vector or window function. The weight vector is typically selected in order to trade resolution for a lower side lobe level.

\subsection{Capon Beamforming}
The Capon beamformer produce a weight vector, like the one in (\ref{eq:das}), based on the impinging signal and an optimization criteria. The minimization problem is as follow:
\begin{align}
&\min_{\vec{w}} E\{|z[n]|^2\} \rightarrow \min_{\vec{w}} \vec{w}^H \mat{\hat{R}} \vec{w} \label{eq:capon_optimization_criteria} \\
&\text{subjected to } \vec{w}^H\vec{a} = 1,
\end{align}
where $\mat{\hat{R}}$ is a sample covariance matrix. Hence, the resulting weight vector will minimizes the output power while maintaining unit gain in the steering direction $\vec{a}$. %The covariance matrix is usually unknown and has to be estimated from the element data. The covariance matrix in (\ref{eq:capon_optimization_criteria}) is therefore substituted with the sample covariance matrix.

The sample covariance matrix can be estimated for an active system in the following way \cite{Synnevag2009}:
\begin{align}
\mat{\breve{R}}[n] = \frac{1}{N_LN_K}\sum_{n'=n-K}^{n+K} \sum_{l=0}^{N_L-1} \vec{x}_l[n']\vec{x}_l[n']^H,
\end{align}
where $K = M-L+1$ typically is proportional to the pulse length, $N_K = 2K + 1$, $N_L = M-L+1$, and $\vec{x}_l$ is the $l\text{th}$ subarray $[x_l[n], \dotso x_{l+L}[n]]$ of length $L$. Finally the matrix is loaded with a diagonal factor for numerical stability and increased robustness, 
\begin{align}
\epsilon &= d*\text{trace}\{\mat{\breve{R}}\}/L\\
\mat{\hat{R}} &= \mat{\breve{R}} + \epsilon\mat{I}.
\end{align} 

The solution to the minimization problem in (\ref{eq:capon_optimization_criteria}) is
\begin{align}\label{eq:capon_weights}
\vec{w}[n] = \frac{\mat{\hat{R}}[n]^{-1}\vec{a}}{\vec{a}^H\mat{\hat{R}}[n]^{-1}\vec{a}} \in \mathbb{C}^L.
\end{align}

Note that one matrix has to be constructed and inverted for each data vector $\vec{x}[n]$ received by the system. This is indeed computational demanding, but following the innovation in GPU computing, and by applying a novel method in order to reduce the size of the matrix inversion problem \cite{Nilsen2009}, real time processing is now possible \cite{Asen}.

%Nilsen et. al has proposed to apply beamspace processing to further reduce the computations required. With beamspace we mean transforming the channel data from element space to a fan of beams covering the imaging sector, $\vec{x}_{BS} = \mat{B}x$.  The steering vectors in $\mat{B}$, the so-called Butler matrix, determines each beams direction in space. We can therefore easily remove beams where no interference is present by removing rows in $\mat{B}$. For a focused system like medical ultrasound, where the received signal is concentrated in a narrow band around broadside, there is a small percentage of the beams that contains almost all of the energy. Nilsen et. al has shown that as little as three beams still produced results comparable with applying the full Capon estimation in element space.

\subsection{Lateral Sampling and Shift-Invariance}

The lateral sampling required in an ultrasound image is given by the combined aperture size on transmit ($D_{tx}$) and receive ($D_{rx}$) \cite{Hergum2007}, and the center frequency ($f_c = c/\lambda$), where $c$ is the speed of sound in the body and $\lambda_c$ is the wavelength of the center frequency. The required Nyquist beam spacing is then:
\begin{align}
\Delta < \frac{\lambda_c}{D_{tx} + D_{rx}}. \label{eq:resolution}
\end{align}
If this requirement is not fulfilled, the scallop loss will be to high and the system will not be lateral shift-invariant.

In order to visualize the local lateral shift-invariance of a given imaging system, we will use lateral shift-variance plots (LSV-plots), introduced by Hergum \textit{et al.} \cite{Hergum2007}. A LSV-plot is constructed for a given imaging system by imaging a laterally moving point scatterer. For each lateral position the root-mean-square (RMS) is calculated in range, and the resulting beam profiles are stacked in order to produce an image. The LSV-plots in this paper are displayed as a contour plot with contours on -1, -2, -3, -6, -12 and -24 dB. In a ultrasound image with 256 gray levels and 50 dB dynamic range 1 dB will corresponds to 5 gray levels. Depending on the surroundings, this is around the lowest difference a human being can detect \comment{need reference}. A loss larger than 1dB will therefore be visible in the image. The maximum for each beam profile is also marked with a black dot. The system is said to be local lateral shift-invariant if the contours are diagonal, hence the diagonals have constant amplitude.

To further quantify the shift-invariance we have calculated the mean absolute error (MAE) of the maximum point in each beam profile versus $f(x)=x$, and the MAE of the amplitude variation for the same set of points. The first number quantifies the amount of geometric distortion, and the latter quantifies the gain variation or in our case the loss of signal. The maximum amplitude deviation along the max-point trace is also calculated and reported. We present all these numbers together with the LSV-plots.

Fig.\,\ref{fig:das_shiftinvariant} presents a LSV-plot for an imaging system using DAS beamforming with a beam spacing equal to (\ref{eq:resolution}). The red dotted vertical lines indicates the position of tx-rx beam pairs. The point scatterer response has been simulated using Field II \cite{Jensen1992, Jensen1996a} \comment{add table with simulation settings}. We see that the contours are approximately diagonal and therefore we can conclude that the system is locally lateral shift-invariant. The P-MAE and A-MAE are also minimal. 

The scallop loss (max A-MAE) is smaller than what was predicted by Fig.\,\ref{fig:das_capon_beams}. To understand why note that Fig.\,\ref{fig:das_capon_beams} shows the one-way response of a passive array, and the system resolution has therefore been calculated based on the receive aperture only. Whereas the beamspacing in (\ref{eq:resolution}) is derived from the fact that the two-way lateral response of an active rectangular aperture is a $D_{tx}+D_{rx}$ wide triangle in K-space \cite{Hergum2009}. Even though the lateral bandwidth is increased, the distance to the first zero (The Rayleigh criteria) remains the same when multiplying two sinc functions. Sampling according to (\ref{eq:resolution}) is therefore more than what is strictly needed in practise. The loss in Fig.\,\ref{fig:das_shiftinvariant} is because of this smaller than what was predicted.

In Fig.\,\ref{fig:das_shiftvariant} the beam spacing is reduced by a factor of two. We see how the scallop loss is now increased to a level where the contours no longer are diagonal. Hence, the imaging system is no longer locally lateral shift-invariant. We observe that the maximum peak is lost when the point is located between two beams and the distance between the beams are too high.

\begin{figure*}[!t]
\centerline{
\subfloat[]{\includegraphics[width=0.5\linewidth]{\img das_normal2.eps}\label{fig:das_shiftinvariant}}
\hfill{}
\subfloat[]{\includegraphics[width=0.5\linewidth]{\img das_undersampled2_2x.eps}\label{fig:das_shiftvariant}}
}
\caption{LSV-plots of DAS beamforming. If the contours are diagonal the imaging system is said to be lateral local shif-invariant. Red vertical lines are plotted where tx-rx beams are located. a) Sampling corresponding to (\ref{eq:resolution}). b) Undersampling with a factor two.}
\label{fig:das}
\end{figure*}

\section{Local Lateral Shift-Invariance of the Capon Beamformer}
We will now investigate the local lateral shift-invariance of the Capon beamformer with parameters equal to the once used in earlier work \comment{cite}. In Fig.\,\ref{fig:capon1}. the same point scatterer response as presented in Fig.\,\ref{fig:das_shiftinvariant} has been weighted with Capon weights calculated using (\ref{eq:capon_weights}) with $L = M/2 = 32$, $K=1$ and $d=1/100$. The set of lateral displacements is constructed in such a way that the center beam is the only one hitting the point scatterer exactly. We see that the scallop loss is tremendous, and not far from the value predicted by Fig\,\ref{fig:das_capon_beams}. The difference between the predicted and the simulated loss is due to several factors. The most important one being that Fig.\,\ref{fig:das_capon_beams} shows a single frequency whereas the RMS used in Fig.\,\ref{fig:capon1} average a wideband signal in range. For a slice through the center of the pulse the scallop loss is below -30dB.

The diagonal factor $d$ and the effective adaptive aperture size $L$ can be used to control the mainlobe width of the Capon beamformer. In both extreme cases, $d=\infty$ and $L=1$, the Capon beamformer becomes the DAS beamformer, and will require a beamspacing according to (\ref{eq:resolution}). In Fig.\,\ref{fig:capon2} we show how the scallop loss is reduced when the amount of diagonal loading is increased. We observe that Fig.\,\ref{fig:capon2} contains the same pattern as present in Fig.\,\ref{fig:das_shiftvariant}. From Fig.\,\ref{fig:image_capon} it should be obvious that the Capon beamformer either require an increase in the lateral sampling in order to make the imaging system locally lateral shift-invariant, or the resolution has to be reduced to the same level as DAS. It would be unwise to do all the computation involved with Capon beamforming with a high $d$ value and end up with a resolution equal to DAS. However, we see that the diagonal factor can be a useful tool if resolution has to be traded for the need of oversampling.
  

\begin{figure*}[!t]
\centerline{
\subfloat[]{
\includegraphics[width=0.5\linewidth]{\img capon_L=32_K=1_d=001_2.eps}\label{fig:capon1}
}
\subfloat[]{
\includegraphics[width=0.5\linewidth]{\img capon_L=32_K=1_d=1_2.eps}\label{fig:capon2}
}
}
\caption{...}
\label{fig:image_capon}
\end{figure*}

\section{Methods}
We will now discuss different ways to improve on the lateral shift-invariance when using the Capon beamformer. From the previous section it should now be clear that we somehow need to increase the lateral sampling. We will investigate how this oversampling should be conducted, how large it needs to be and how the different methods will affect the system frame rate.

\subsection{Oversampling on Transmit}
The typical way of obtaining sufficient sampling in an active broadband system is by increasing the number of transmit beams until (\ref{eq:resolution}) is fulfilled. In Fig.\,\ref{fig:capon_oversampling_a} and Fig.\,\ref{fig:capon_oversampling_b} we have increased the lateral sampling on transmit by a factor four and ten respectively. As expected the scallop loss decreases with increased lateral sampling. Note that Fig.\,\ref{fig:capon_oversampling_b} is zoomed in order to better see the beam-to-beam variation. Even though increasing the number of transmit improves the lateral shift-invariance it will also reduce the imaging frame rate with a factor equal to the oversampling factor. It is therefore not a feasible option for real time ultrasound imaging.

From Fig.\,\ref{fig:capon_oversampling_b} we see that more than ten times oversampling has to be applied when imaging a moving point scatterer with no spatially white noise and $1/100$ in diagonal loading. A diagonal loading factor of $1/100$ will introduce a white noise component in the sample covariance matrix $40$ dB lower than the signal present. Hence, the signal-to-noise ratio before the array gain is applied is $40$ dB. \comment{do calculations}.

\begin{figure*}[!t]
\centerline{
\subfloat[]{
\includegraphics[width=0.5\linewidth]{\img capon_L=32_K=1_d=001_4x_oversampling_2_zoomed.eps}\label{fig:capon_oversampling_a}
}
\subfloat[]{
\includegraphics[width=0.5\linewidth]{\img capon_L=32_K=1_d=001_10x_oversampling_2_zoomed.eps}\label{fig:capon_oversampling_b}
}
}
\caption{Oversampling on transmit ...}
\label{fig:capon_oversampling}
\end{figure*}

\subsection{Oversampling by Parallel Recieve Lines}
In (\comment{Cite paper where they use STB + Capon if its out}) it is shown that  synthetic transmit beamforming (STB) \cite{Hergum2007} works with Capon beamforming. In  \cite{Asen} we also used STB in order to increase the beam density of a cardiac recording. STB has been shown to remove the blocky artifacts introduced by parallel receive lines (PRL) per transmit line. As long as some degree of motion compensation is applied (\comment{cite bastien's paper on motion compensation }) we can regard PRL as being equal to oversampling on transmit. It will also increase the number of covariance matrices, and thereby inversions with a factor equal to the number of parallel lines.
%\begin{figure}[!t]
%	\centering
%	\includegraphics[width=\linewidth]{\img capon_L=32_K=1_d=001_2MLA_and_2x_microsteering_correct_pitch.eps}
%	\label{fig:image_capon_bs}
%\end{figure}

\subsection{Oversampling by Phase Rotation}
The Capon beamformer formula in (\ref{eq:capon_weights}) gives a direct opportunity for oversampling using phase rotation. The steering vector $\vec{a}$ has in previous work been set to $\vec{1}$ since the ultrasound data are pre-delayed. The steering vector could be varied over a set of pre-defined steering vectors. \comment{Look at the poster for notation}

\begin{figure*}[!t]
	\centerline{
		\subfloat[]{
			\includegraphics[width=0.5\linewidth]{\img capon_L=32_K=1_d=001_4x_oversampling_2_PR_zoomed.eps}
		}
		\subfloat[]{
			\includegraphics[width=0.5\linewidth]{\img capon_L=32_K=1_d=001_10x_oversampling_2_zoomed_PR.eps}
		}
	}
	\caption{Oversampling using phase rotation ...}
	\label{fig:benchmark_capon_bs}
\end{figure*}

\section{Results}\label{sec:res}

Acquire channel data from a phantom containing point wire targets and cysts while moving the probe laterally. Confirm the improved shift-invariance when using phase rotation. See if improved sampling also increase cyst contrast.
Make plot of contrast between two areas in a simulation for different values of $L$. Make the same plot for different number of rx beams. Do the same for an in-vivo image. The rational being that contrast is also improved by denser rx-sampling not just shift-invariance. 

\begin{figure}\caption{Plot of contrast vs oversampling factor (phase rotation) in in-vivo data.}
\end{figure}

\subsection{In-vitro Images}
Acquire channel data from a phantom containing point wire targets and cysts while moving the probe laterally. Confirm the improved shift-invariance when using phase rotation. 

\begin{figure*}[!t]
\centerline{
%\subfloat[]{\includegraphics[width=0.3\linewidth]{\img das_normal2.eps}\label{fig:das_shiftinvariant}}
\hfill{}
%\subfloat[]{\includegraphics[width=0.3\linewidth]{\img das_normal2.eps}\label{fig:das_shiftinvariant}}
\hfill{}
%\subfloat[]{\includegraphics[width=0.3\linewidth]{\img das_undersampled2_2x.eps}\label{fig:das_shiftvariant}}
}
\caption{LSV-plots of in-vitro data. a) Image of the data. b) Normal. c) Oversampling with phase rotation}
\label{fig:das}
\end{figure*}

\subsection{In-Vivo Images}
Investigate if phase rotation improves contrast in in-vivo images of the heart.

\begin{figure*}[!t]
\centerline{
%\subfloat[]{\includegraphics[width=0.3\linewidth]{\img das_normal2.eps}\label{fig:das_shiftinvariant}}
\hfill{}
%\subfloat[]{\includegraphics[width=0.3\linewidth]{\img das_normal2.eps}\label{fig:das_shiftinvariant}}
\hfill{}
%\subfloat[]{\includegraphics[width=0.3\linewidth]{\img das_undersampled2_2x.eps}\label{fig:das_shiftvariant}}
}
\caption{LSV-plots of in-vivo data. a) Image of the data. b) Normal. c) Oversampling with phase rotation}
\label{fig:das}
\end{figure*}

\section{Discussion}\label{sec:dis}
Discuss results presented in Section \ref{sec:res}.

When signal detection (absolute value and square root) is performed before an ultrasound image is displayed the lateral bandwidth is increased. In order to not introduce aliasing in the image, IQ-interpolation is typically used on ultrasound systems to double the lateral beam density before detection. For the same reason we need to increase the lateral sampling when Capon beamforming is applied, but now on a channel data level. The Capon beamformer improves the lateral resolution, hence the lateral bandwidth is increased, and higher lateral sampling is required. 

In this paper we have shown that this oversampling can be accomplished by phase rotation and the number of transmit beams are therefore left unchanged. 

The figure below shows a lateral shift invariance plots for a well sampled delay-and-sum image (a), the same sampling with Capon processing (b), and 4x over sampling processed with Capon. The shift invariance plots the beamprofile (TODO: Make the shift invariance plot). The results show that when using the Capon beamformer we have to over sample with a factor four to remove the blinking artifact. This can be done without affecting frame rate by using the available parallel receive-lines found on most ultrasound systems. May also use phase rotation. 

TODO: Add citations – Hergum/Bjåstad Local shift invariance and MLA’s… Is there any publications stating that Capon do not work on moving objects?

\subsection{Impact on Real Time Performance}
Discuss, and show with benchmarks how the different methods for making Capon beamforming shift-invariant impact the total execution time.

\section{Conclusion}\label{sec:con}
...


% An example of a floating figure using the graphicx package.
% Note that \label must occur AFTER (or within) \caption.
% For figures, \caption should occur after the \includegraphics.
% Note that IEEEtran v1.7 and later has special internal code that
% is designed to preserve the operation of \label within \caption
% even when the captionsoff option is in effect. However, because
% of issues like this, it may be the safest practice to put all your
% \label just after \caption rather than within \caption{}.
%
% Reminder: the "draftcls" or "draftclsnofoot", not "draft", class
% option should be used if it is desired that the figures are to be
% displayed while in draft mode.
%
%\begin{figure}[!t]
%\centering
%\includegraphics[width=2.5in]{myfigure}
% where an .eps filename suffix will be assumed under latex, 
% and a .pdf suffix will be assumed for pdflatex; or what has been declared
% via \DeclareGraphicsExtensions.
%\caption{Simulation Results}
%\label{fig_sim}
%\end{figure}

% Note that IEEE typically puts floats only at the top, even when this
% results in a large percentage of a column being occupied by floats.


% An example of a double column floating figure using two subfigures.
% (The subfig.sty package must be loaded for this to work.)
% The subfigure \label commands are set within each subfloat command, the
% \label for the overall figure must come after \caption.
% \hfil must be used as a separator to get equal spacing.
% The subfigure.sty package works much the same way, except \subfigure is
% used instead of \subfloat.
%
%\begin{figure*}[!t]
%\centerline{\subfloat[Case I]\includegraphics[width=2.5in]{subfigcase1}%
%\label{fig_first_case}}
%\hfil
%\subfloat[Case II]{\includegraphics[width=2.5in]{subfigcase2}%
%\label{fig_second_case}}}
%\caption{Simulation results}
%\label{fig_sim}
%\end{figure*}
%
% Note that often IEEE papers with subfigures do not employ subfigure
% captions (using the optional argument to \subfloat), but instead will
% reference/describe all of them (a), (b), etc., within the main caption.


% An example of a floating table. Note that, for IEEE style tables, the 
% \caption command should come BEFORE the table. Table text will default to
% \footnotesize as IEEE normally uses this smaller font for tables.
% The \label must come after \caption as always.
%
%\begin{table}[!t]
%% increase table row spacing, adjust to taste
%\renewcommand{\arraystretch}{1.3}
% if using array.sty, it might be a good idea to tweak the value of
% \extrarowheight as needed to properly center the text within the cells
%\caption{An Example of a Table}
%\label{table_example}
%\centering
%% Some packages, such as MDW tools, offer better commands for making tables
%% than the plain LaTeX2e tabular which is used here.
%\begin{tabular}{|c||c|}
%\hline
%One & Two\\
%\hline
%Three & Four\\
%\hline
%\end{tabular}
%\end{table}


% Note that IEEE does not put floats in the very first column - or typically
% anywhere on the first page for that matter. Also, in-text middle ("here")
% positioning is not used. Most IEEE journals use top floats exclusively.
% Note that, LaTeX2e, unlike IEEE journals, places footnotes above bottom
% floats. This can be corrected via the \fnbelowfloat command of the
% stfloats package.



%\section{Conclusion}
%The conclusion goes here.





% if have a single appendix:
%\appendix[Proof of the Zonklar Equations]
% or
%\appendix  % for no appendix heading
% do not use \section anymore after \appendix, only \section*
% is possibly needed

% use appendices with more than one appendix
% then use \section to start each appendix
% you must declare a \section before using any
% \subsection or using \label (\appendices by itself
% starts a section numbered zero.)
%


%\appendices
%\section{Proof of the First Zonklar Equation}
%Appendix one text goes here.

% you can choose not to have a title for an appendix
% if you want by leaving the argument blank
%\section{}
%Appendix two text goes here.


% use section* for acknowledgement
\section*{Acknowledgment}


The authors would like to thank... \cite{Synnevag2009}


% Can use something like this to put references on a page
% by themselves when using endfloat and the captionsoff option.
\ifCLASSOPTIONcaptionsoff
  \newpage
\fi



% trigger a \newpage just before the given reference
% number - used to balance the columns on the last page
% adjust value as needed - may need to be readjusted if
% the document is modified later
%\IEEEtriggeratref{8}
% The "triggered" command can be changed if desired:
%\IEEEtriggercmd{\enlargethispage{-5in}}

% references section

% can use a bibliography generated by BibTeX as a .bbl file
% BibTeX documentation can be easily obtained at:
% http://www.ctan.org/tex-archive/biblio/bibtex/contrib/doc/
% The IEEEtran BibTeX style support page is at:
% http://www.michaelshell.org/tex/ieeetran/bibtex/
\bibliographystyle{IEEEtran}
% argument is your BibTeX string definitions and bibliography database(s)
%\bibliography{IEEEabrv,../bib/paper}
\bibliography{mybib}
%
% <OR> manually copy in the resultant .bbl file
% set second argument of \begin to the number of references
% (used to reserve space for the reference number labels box)
%\begin{thebibliography}{1}

%\bibitem{IEEEhowto:kopka}
%H.~Kopka and P.~W. Daly, \emph{A Guide to \LaTeX}, 3rd~ed.\hskip 1em plus
%  0.5em minus 0.4em\relax Harlow, England: Addison-Wesley, 1999.

%\end{thebibliography}

% biography section
% 
% If you have an EPS/PDF photo (graphicx package needed) extra braces are
% needed around the contents of the optional argument to biography to prevent
% the LaTeX parser from getting confused when it sees the complicated
% \includegraphics command within an optional argument. (You could create
% your own custom macro containing the \includegraphics command to make things
% simpler here.)
%\begin{biography}[{\includegraphics[width=1in,height=1.25in,clip,keepaspectratio]{mshell}}]{Michael Shell}
% or if you just want to reserve a space for a photo:

% insert where needed to balance the two columns on the last page with
% biographies
%\newpage

\begin{IEEEbiography}{Jon Petter \AA{}sen}
Biography text here.
\end{IEEEbiography}

%\begin{IEEEbiography}{Carl-Inge Colombo Nilsen}
%Biography text here.
%\end{IEEEbiography}

% insert where needed to balance the two columns on the last page with
% biographies
%\newpage

\begin{IEEEbiography}{Andreas Austeng}
Biography text here.
\end{IEEEbiography}

\begin{IEEEbiography}{Sverre Holm}
Biography text here.
\end{IEEEbiography}


% You can push biographies down or up by placing
% a \vfill before or after them. The appropriate
% use of \vfill depends on what kind of text is
% on the last page and whether or not the columns
% are being equalized.

%\vfill

% Can be used to pull up biographies so that the bottom of the last one
% is flush with the other column.
%\enlargethispage{-5in}



% that's all folks
\end{document}



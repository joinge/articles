
\newpage

\section{Wave Equation}



The wave equation describes how a wave move as a function of both time and space. It forms the basis for all mathematical modeling of waves, so we will spend some time to derive it here. We will treat acoustical waves since they are most relevant to the topic in this thesis, but we could have derived it from electromagnetic waves using Maxwells equations with some boundary conditions.

\begin{figure}[ht]
\begin{floatrow}

\floatbox{figure}[\linewidth][\FBheight][t]{%
\caption{Mass flow}}{%
\graphicsAI[drawing,width=.77\linewidth]{gfx/wave_equation_mass_flow.svg}}%

\floatbox{table}[\linewidth][\FBheight][t]{%
\caption{Symbol description}}{%
\begin{tabular}[c]{c c l}\hline
\rowcolor{tabBlue}\bf Symbol & \bf Unit & \bf Description \\\hline
$p$ & Pa = N/m$^2$ & Pressure \\
$\rho$ & kg/m$^3$ & Density \\
$S$ & Nm/K & Entropy \\
$\vec v$ & m/s & Mass velocity \\
$\dot{\boldsymbol{m}}$ & kg/s & Mass flow
\end{tabular}}

\end{floatrow}
\end{figure}


We begin by defining the net \textbf{mass flow} through a volumetric element $dV = dx\,dy\,dz$:
%
\begin{align}
\frac{\dot m_\text{net}}{dV} = \frac{\partial \rho\,\v}{\partial x} \ux + \frac{\partial \rho\,\v}{\partial y} \uy + \frac{\partial \rho\,\v}{\partial z} \uz = \nabla \cdot (\rho\v):
\end{align}
%
The space and time dependence of the mass density $\rho = \rho(\p,t)$ and particle velocity $\v = \v(\p,t)$ was omitted here for brevity. If the volume of the element is fixed a change in mass must equal a corresponding change in density. This is the law of \textbf{mass conservation}:
%
\begin{align}
\frac{\partial\rho}{\partial t} = -\nabla \cdot (\rho\v)
\end{align}
%


\section{Roughness}

For a plane monochromatic wave the Rayleigh roughness parameter is quantified by:
\begin{align}
\Delta\varphi = 2k\xi\cos\theta.
\end{align}
Here $\theta$ is the incidence angle, $k = \frac{2\pi}{\lambda}$ is the wavenumber of the incoming wave, $\xi$ is the distance from the scattering point to the mean surface plane, and $\Delta\varphi$ is the phase difference.

\begin{figure}[th]
\graphicsAI[drawing,width=\linewidth]{gfx/roughness.svg}
\end{figure}

To see how roughness affects the specular reflection we compute the mean reflected wave as
\begin{align}
<p> &= V\int\limits_{-\infty}^{\infty} e^{-j\Delta\varphi(\xi)} p(\xi) d\xi,
\end{align}
where $p(\xi)$ is the probability density function of $\xi$. If we assume normal distribution, we get:
\begin{align}
<p> &= V\int\limits_{-\infty}^{\infty} e^{-j2k\xi\cos\theta} \frac{1}{\sqrt{2\pi}\sigma}e^{-\frac{\xi^2}{2\sigma^2}} d\xi \nn
&= \frac{V}{\sqrt{2\pi}\sigma} \int\limits_{-\infty}^{\infty} e^{-\frac{\xi^2}{2\sigma^2}-j2k\xi\cos\theta} d\xi &\Big|{\begin{matrix}\scriptscriptstyle\hspace{-0.64cm} a = \frac{1}{2\sigma^2}\\
\scriptscriptstyle b = -j2k\cos\theta\end{matrix}}\nn
&= A \int\limits_{-\infty}^{\infty} e^{-a\xi^2 + b\xi} d\xi \nn
&= A \int\limits_{-\infty}^{\infty} e^{-a(\xi - \frac{b}{2a})^2 - \frac{b^2}{4a}} d\xi \nn
&= A e^{\frac{-b^2}{4a}} \int\limits_{-\infty}^{\infty} e^{-a(\xi - \frac{b}{2a})^2} d\xi & \Big|_{y = \xi - \frac{b}{2a}}\nn
&= A e^{\frac{-b^2}{4a}} \int\limits_{-\infty}^{\infty} e^{-ay^2} dy & \Big|_{z = \sqrt{a} y}\nn
&= A e^{\frac{-b^2}{4a}}\frac{1}{\sqrt{a}} \int\limits_{-\infty}^{\infty} e^{-z^2} dz \nn
&= A e^{\frac{-b^2}{4a}}\sqrt{\frac{\pi}{a}} \nn
&= \frac{V}{\sqrt{2\pi}\sigma} \sqrt{\frac{\pi}{\frac{1}{2\sigma^2}}} e^{\frac{-(-j2k\cos\theta)^2}{4\frac{1}{2\sigma^2}}} \nn
&= V e^{-2k^2\sigma^2\cos^2\theta}
\end{align}
This reflection coefficient is valid when the Rayleigh parameter is small (i.e. small frequency and relief amplitude), and at grazing incidence. Conventional limit of validity is $\frac{\pi}{2}$, which corresponds to $\sigma = \frac{\lambda}{8\cos\theta}$ and a coherent loss of 10.7dB.


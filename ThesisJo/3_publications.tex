%&header
% !TeX root = 2_publications
\input{build/subheader.tex}
\endofdump

\ifRootBuild\else
  \newglossaryentry{ASIC}{name={ASIC},
                  description={Application Specific Integrated Circuit} } 
                  
\newglossaryentry{ATR}{name={ATR},
                  description={Automatic Target Recognition} } 

\newglossaryentry{CPU}{name={CPU},
						description={Central Processing Unit} } 

\newglossaryentry{GPGPU}{name={GPGPU},
						description={General Purpose Graphics Processing Unit} } 

\newglossaryentry{GPU}{name={GPU},
						description={Graphics Processing Unit} } 
					
\newglossaryentry{MVDR}{name={MVDR},
						description={Minimum Variance Distortionless Response} } 
  \makeglossaries
\fi

\begin{document}


\chapter{Publications}
 
%The upcoming papers will investigate various ways in which acoustical imaging systems can be improved by making them smarter in an operational sense. This involves granting them the ability to adapt their behavior to the information at hand, and the processing power needed to handle the information fast enough.

%%~~~~~~~~~~~~~~~~~~~~~~~~~~~~~~~~~~~~~~~~~~~~~~~~~~~~~~~~~~~~~~~~~~~~~~~~~~~~~~~~~~~~~~~~~~~~~~~~~~~
%\section{Intro}\label{sec:intro} %
%%~~~~~~~~~~~~~~~~~~~~~~~~~~~~~~~~~~~~

%
%\paper{I} describes the use and implementation of the \gls{MVDR} beamformer in active sonar imaging in its most general and exact form. 
%
%Its widespread use and generic implementation makes it the reference. thus will be considered the reference method. , and this will be referred to as a fully adaptive technique. Due to its ability to analytically  to apply the minimum variance criteria in an exact manner, Due to its popularity and We view this as the reference method for We refer to it as fully adaptive since it computes analytically evaluates   yields  This is a fully adaptive method   This method is  on a \gls{GPGPU} for active sonar imaging. This is to prove that using a \gls{MVDR}
%
%Paper II takes the \gls{MVDR} method and applies it to clinical Ultrasound Imaging
%
%
%We point out important hardware limitations for these devices, and assess the design in terms of how efficiently it is able to use the GPU's resources. On a quad-core Intel Xeon system with a high-end Nvidia GPU, our GPU implementation renders more than a million pixels per second (1\;MP/s). Compared to our initial central processing unit (CPU) implementation the optimizations described herein led to a speedup of more than two orders of magnitude, or an expected five to ten times improvement had the CPU received similar optimization effort. This throu

% Classification accuracy
% Logarithmic loss
% Confusion matrix
% Area under curve
% F1 score
% Mean absolute error
% Mean squared error
%
% 


We present four journal published papers concerned with \emph{adaptive} and \emph{hardware accelerated} formation of acoustic images: the first three through \emph{reconstruct} from echoes and the last \emph{synthesizes} through simulation. 

\begin{figure}[tp]\label{III_fig_article_relation}
\makebox[\linewidth][c]{%
\graphicsAI[drawing,width=1.3\linewidth]{gfx/article_relation.svg}}
\caption{Article relations (repost of Figure \ref{I_fig_article_relation} in chapter 1). The first three articles reconstructs images from acoustic echoes. The last article synthesizes image templates from 3D models and object parameters estimated from the reconstructed image. Accelerated adaptivity is key: it enables rapid dynamic responses to mission events. }
\end{figure}


%
%Common:
%
%GPUs
%Experimental validation on HISAS1030 data.


%\begin{itemize}\setlength\baselineskip{0pt}
%\item \emph{Optimization method}, or criteria for adaptiveness and constraints;
%\item \emph{Solution space}, or range of possible values for an optimization problem that satisfy a problem's constraints;
%\item \emph{Domain}, or data representation, 
%\item \emph{Application}, 
%\item \emph{Computational complexity}, $O(N^3)$.
%\item Hardware acceleration. 
%\item Information significance: Minimal temporal sample support (~single snapshot)
%\end{itemize}
%
%All:
%- GPU
% MVDR: Needs acceleration
% LCA: No particular need of acceleration
% Sim: Needs acceleration

%~~~~~~~~~~~~~~~~~~~~~~~~~~~~~~~~~~~~~~~~~~~~~~~~~~~~~~~~~~~~~~~~~~~~~~~~~~~~~~~~~~~~~~~~~~~~~~~~~~~
\section{Paper I}\label{sec:paperI} %
%~~~~~~~~~~~~~~~~~~~~~~~~~~~~~~~~~~~~
\textbf{An Optimized GPU implementation of the\\
MVDR Beamformer for Active Sonar Imaging}\\
J.I. Buskenes, J.P. Åsen, C.-I.C. Nilsen and A. Austeng\\
\textit{IEEE Journal of Oceanic Engineering}\\
Volume 40, Issue 2, April 2015

This article accelerates element-space MVDR with GPUs for use in active sonar imaging. Element-space MVDR is the reference method for creating sonar images using minimum variance optimization, aimed at reducing noise and interference while preserving signal integrity. We refer to it as fully adaptive because it computes array weights analytically from spatial features in the data, yielding array weights from a continuous solution space that are optimal in the minimum variance sense. Like any data-driven algorithm, it requires statistically distinguishable signal and noise. However, in typical active systems, the signal is highly correlated with the noise and has poor temporal sample support. Therefore, in its unconstrained form, MVDR overfits the data, yields high variance and fails to generate representative images.

We relax MVDR through a combination of temporal and spatial averaging, and regularization. This makes it robust and produces images with slightly more details and less noise than delay-and-sum. The resulting solution space is still continuous, but more tightly bound.

We find that implementing MVDR on GPUs instead of CPUs accelerates it by two orders of magnitude, but requires considerable and delicate efforts to reduce arithmetic operations and fine-tune cache usage. The design is limited to 32 channels due to data dependencies that are complicated and not inherently parallel, combined with the relatively limited bandwidth of this hardware architecture. 

%Time-domain MVDR is the reference method for creating sonar images using the minimum variance criterion. It computes array weights analytically, yielding optimal weights in the minimum variance sense. Initially, this may seem ideal, but its use is cumbersome: parameters for temporal averaging, spatial averaging and regularization must often be fine-tuned ad-hoc to obtain improved image quality. GPU acceleration mitigates the high computational complexity resulting from estimating and inverting a spatial covariance matrix, yielding real-time performance.

%
%%\begin{itemize}\setlength\baselineskip{0pt}
%\item Solution space is continuous but strictly constrained due to robustification,
%\item Time domain
%\item Computationally complex, $O(N^3)$
%\item Information significance: Minimal temporal sample support (~single snapshot)
%\end{itemize}

%This is a popular method that analytically estimates the array's weight set that best comply with the minimum variance criterion. Its analytic nature allows it to chose from an infinite number of possible weight sets, but with a solution space limited by the constraint of unit gain in the look direction. 
%


%In functionality this method acts as our reference method, but its practicality is limited by a high computational complexity. The paper suggests means for reducing the arithmetic complexity and provide extensive details for fitting the implementation on a \gls{GPU} for a major speedup. It also relates the performance to theoretical maximums and investigates remaining bottlenecks.


%
%When performing image processing data from phased arrays an inherent question arise 
%In the pursuit of creating images with more detail and less noise from phased-array data, a fully adaptive beamformer such as the the Minimum Variance Distortionless Response (MVDR) can be used in the receive 

% The minimum variance distortionless response (MVDR) beamformer has recently been proposed as an attractive alternative to conventional beamformers in active sonar imaging. Unfortunately, it is very computationally complex because a spatial covariance matrix must be estimated and inverted for each image pixel. This may discourage its use unnecessarily in sonar systems which are continuously being pushed to ever higher imaging ranges and resolutions.
% 
% In this study we show that for active sonar systems up to 32 channels, the computation time can be significantly reduced by performing arithmetic optimizations, and by implementing the MVDR beamformer on a graphics processing unit (GPU). We point out important hardware limitations for these devices, and assess the design in terms of how efficiently it is able to use the GPU's resources. On a quad-core Intel Xeon system with a high-end Nvidia GPU, our GPU implementation renders more than a million pixels per second (1\;MP/s). Compared to our initial central processing unit (CPU) implementation the optimizations described herein led to a speedup of more than two orders of magnitude, or an expected five to ten times improvement had the CPU received similar optimization effort. This throughput enables real-time processing of sonar data, and makes the MVDR a viable alternative to conventional methods in practical systems.



%~~~~~~~~~~~~~~~~~~~~~~~~~~~~~~~~~~~~~~~~~~~~~~~~~~~~~~~~~~~~~~~~~~~~~~~~~~~~~~~~~~~~~~~~~~~~~~~~~~~
\section{Paper II}\label{sec:paperII} %
%~~~~~~~~~~~~~~~~~~~~~~~~~~~~~~~~~~~~~~
\textbf{Implementing Capon Beamformer on a GPU for\\ Real-Time Cardiac Ultrasound Imaging}\\
J.P. Åsen, J.I. Buskenes, C.-I.C. Nilsen, A. Austeng and S. Holm\\
\textit{IEEE Transactions on Ultrasonics, Ferroelectrics, and Frequency Control}\\
Volume 61, Issue 1, January 2015

This article evaluates frequency domain (or beamspace) MVDR on GPUs for use in cardiac ultrasound imaging. These systems typically have real-time requirements and exceed 32 channels, so we apply a space reduction technique to reduce computational effort. Since the transmitted beams are narrow, most acoustic energy falls within three to five center frequencies, and a good approximation follows from just computing those.

We find that beamspace MVDR delivers real-time performance independent of channel-count and image quality indistinguishable from time domain MVDR.


%The first article implements the fully adaptive MVDR beamformer on a GPU for use in active sonar imaging. Like any data-driven algorithm, it requires data with statistically distinguishable signal and noise. However, in typical active systems, the signal is highly correlated with the noise and has weak temporal sample support. Therefore, MVDR in its unconstrained form overfits the data, yields high variance and fails.
%
%We relax MVDR through a combination of temporal and spatial averaging, and regularization. This makes it robust and produces images of slightly higher quality than delay-and-sum. 
%
%Implementing MVDR on a GPU instead of a CPU led to a speed-up of two orders of magnitude, but required considerable and delicate efforts to reduce arithmetic operations and fine-tune cache usage. The data dependencies are complicated and not inherently parallel. 


%~~~~~~~~~~~~~~~~~~~~~~~~~~~~~~~~~~~~~~~~~~~~~~~~~~~~~~~~~~~~~~~~~~~~~~~~~~~~~~~~~~~~~~~~~~~~~~~~~~~
\section{Paper III}\label{sec:paperIII} %
%~~~~~~~~~~~~~~~~~~~~~~~~~~~~~~~~~~~~~~~~
\textbf{Low Complexity Adaptive Beamforming}~\cite{Buskenes2014}\\
J.I. Buskenes, R.E. Hansen and A. Austeng\\
\textit{IEEE Journal of Oceanic Engineering}\\
Volume 42, Issue 1, January 2017

This article describes LCA, an alternative to MVDR that iterates through precomputed sets of array weights and picks the one that best fulfills the minimum variance criterion. It can be viewed as either a simplification of MVDR or as an adaptive extension to DAS.

We find that constrained MVDR typically generates spatial responses between that of uniform weights and Hamming weights. When LCA is configured with six weight sets covering this range, it matches MVDR in image quality. As a bonus, it is inherently robust and easy to work with, understand and accelerate with GPUs.



%~~~~~~~~~~~~~~~~~~~~~~~~~~~~~~~~~~~~~~~~~~~~~~~~~~~~~~~~~~~~~~~~~~~~~~~~~~~~~~~~~~~~~~~~~~~~~~~~~~~
\section{Paper IV}\label{sec:paperIV} %
%~~~~~~~~~~~~~~~~~~~~~~~~~~~~~~~~~~~~~~
\textbf{A Real-Time Synthetic Aperture Sonar Simulator for Automatic Target Recognition: Notation and application}\\
J.I. Buskenes, H. Midelfart\\
\textit{IEEE Journal of Oceanic Engineering}\\
Submitted DD.MM.2019%\\\\
%Volume 42, Issue 1, January 2017

This article describes an acoustic simulator for SAS, implemented on GPUs using OpenGL and OpenCL. It enables AUVs to compare its images on-the-fly to simulations of various known objects, for more responsive, accurate and flexible automatic target recognition (ATR).

We find that simulating objects on-the-fly instead of pre-simulating them consistently improves classification accuracy at false positive rates (FPR) below 20\%, and ties above. 

The design is discussed in-depth using a new navigational notation system, extended with topological concepts from robotics and homogeneous coordinates. We find it incredibly helpful for keeping track of complex geometries, numerical representations and higher level abstractions such as manipulating rigid objects. 



%Acoustic image reconstruction and synthesis are closely related, and 

%
%
%
%\begin{itemize}
%\item Template-generation
%\item Autonomy
%\item Sonar
%	\item Incredibly fast template matching possible.
%	\item Possibly a powerful tool to pass prior knowledge to deep learning
%	\item Extracting the essential information is key 
%	\item Exploit hardware resources available
%\end{itemize}





\end{document}

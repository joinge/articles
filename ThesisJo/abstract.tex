% \newif\ifPhdDoc\PhdDoctrue
\newif\ifMonolithic\Monolithicfalse

\newif\ifRootBuild\RootBuildfalse\ifMonolithic\else
% \documentclass[
% bibtex,
% xelatex]{common/uiophd}
 
\documentclass[
   uiophd
 , 12pt
 , english
 , floatrow
%  , xelatex
 , glossary                                     % Use a glossary
 , biblatex                                 % Use a bibliography
%  , draft
%  , layout
]{common/mytemplate}

% \typeout{++++import++++}
% \usepackage{import}

% \typeout{--------------}
%\hypersetup{
%   bookmarksopen=true
% , bookmarksopenlevel=2
%}

\newcommand*\cleartoleftpage{%
	\clearpage
	\ifodd\value{page}\hbox{}\newpage\fi
}
% Fonts:
% https://tex.stackexchange.com/questions/19898/getting-urw-garamond-and-the-license

\providecommand\1{}
\renewcommand*\1{\vec 1}
\providecommand\a{}
\renewcommand*\a{\vec a}
\providecommand\b{}
\renewcommand*\b{\vec b}
\providecommand\eR{}
\renewcommand*\eR{\mat{\hat R}}
\providecommand\eRi{}
\renewcommand*\eRi{\hat{\mat R}\;\!^{-1}}
\providecommand\k{}
\renewcommand*\k{\vec k}
\providecommand\n{}
\renewcommand*\n{\vec n}
\providecommand\p{}
\renewcommand*\p{\vec p}
\providecommand\R{}
\renewcommand*\R{\mat R}
\providecommand\Ri{}
\renewcommand*\Ri{\R^{-1}}
\providecommand\s{}
\renewcommand*\s{\vec s}

\providecommand\v{}
\renewcommand*\v{\vec v}
\providecommand\x{}
\renewcommand*\x{\vec x}
\providecommand\X{}
\renewcommand*\X{\mat X}
\providecommand\w{}
\renewcommand*\w{\vec w}

\providecommand\argmin{}
\renewcommand*\argmin[1]{\underset{#1}{\text{argmin}}\ }
\providecommand\norm{}
\renewcommand*\norm[1]{\left\lVert #1 \right\rVert}
\providecommand\min{}
\renewcommand*\min[1]{\underset{#1}{\text{min}}}
\providecommand\sinc{}
\renewcommand\sinc{\text{sinc}}

\providecommand*\nn{}
\renewcommand*\nn{\nonumber\\}

\providecommand\sumb{}
\renewcommand\sumb[2]{\sum\limits_{#1}^{#2}\;}

\typeout{++++floatrow++++}
\usepackage{floatrow}
%\usepackage{bibtex}
%\usepackage{comment}
\typeout{++++environ++++}
\usepackage{environ}
%\usepackage[runin]{abstract}
\typeout{++++parskip++++}
\usepackage[parfill]{parskip}
\typeout{++++glossaries++++}
\usepackage[nonumberlist,nogroupskip]{glossaries}
\typeout{++++afterpage++++}
\usepackage{afterpage}
% \typeout{++++placeins++++}
% \usepackage{afterpage}
\typeout{--------------}


% \newcommand\gls[1]{#1}

\renewcommand\figurename{Fig.}
\renewcommand\thefigure{\arabic{figure}}
\usepackage[margin=0pt,font={it,small},labelfont={bf,it},labelsep=colon]{caption}
% \addto\captionsenglish{\renewcommand{\figurename}{Fig.}}

\renewcommand\thetable{\arabic{table}} 
\renewcommand\theequation{\arabic{equation}} 

% Increase header a bit to allow two lines in it
\addtolength\headheight{\topmargin}
\setlength\topmargin{0pt}

\titleformat{\chapter}%command
[display]%shape
% {}%format
{\flushright}%format
{}%label
{0pt}%sep
{\color{Black}\bfseries\Huge}%before
[]%after
\titlespacing*{\chapter}{0pt}{*0}{*2}

\titleformat{\section}%command
[block]%shape
% {}%format
{\Large\bfseries}%format
{\thesection}%label
{10pt}%sep
{}%before
[]%after
\titlespacing*{\section}{0pt}{*0}{*0}

\titleformat{\subsection}%command
[block]%shape
% {}%format
{\large\bfseries}%format
{\thesubsection}%label
{10pt}%sep
{}%before
[]%after
\titlespacing*{\subsection}{0pt}{*0}{*0}

\input{subheader}\fi
%~~~~~~~~~~~~~~~~~~~~~~~~~~~~~~~~~~~~~~~~~~~~~~~~~~~~~~~~~~~~~~~~~~~~~~~~~~~~~~~~~~~~~~~~~~~~~~~~~~

The world is highly dynamic, constantly changing from moment to moment. An autonomous device aspiring to navigate it must be situationally perceptive and reactive. This requires numerous sensors gathering high quality data at high throughput, adaptive signal processing algorithms and powerful parallel hardware capable of near-realtime responses.

This thesis explores ways to accelerate adaptive acoustic image reconstruction and synthesis. The reconstruction process relies heavily on beamforming: the art of delaying and weighting sensor arrays to achieve desired spatial responses. A beamformer can be viewed as a spatial filter or model, and works best---as any model---when it is neither too simple nor too complex. 

All models have a sweet spot in complexity, a balance point where its estimates offer an ideal compromise between bias and variance. 

Such a spatial filter, or model, . model complexity should be just right. not too simple, not too complex. bias vs variance. 



sensory faculty

as adaptive as possible with the least possible computational load.




gather information
dynamically analyze it
and react to it
Får 

adapt
persist
prevail

Adaptive algorithm and the computation power to drive it.


Necessitates an algorithm that adapts to changing data. 



Cognition is the mental action or process of acquiring knowledge and understanding through thought, experience and the senses. 

The intelligence and autonomy of acoustical imaging devices require more sensory data, algorithmic adaptivity and processing power. All are vital and equally important. 



improves with more high quality sensory data, sophisticated adaptive signal processing and the hardware resources to deal with it all. 

more sensory data, algorithmic adaptivity and processing power.

adaptive processing and processing power. 

benefits from more sensors, sensory data,   detectors rely on 

autonomy improves with data amount, adaptive processing, processing power.

fast dynamic information gathering and processing. 


sensory input, sophisticated information and the processing power to drive it all. 

rely on fast adaptive data processing. have an intrinsic need for adaptive data processing and hardware acceleration. 

Accelerated adaptive image reconstruction and synthesis. 

information gathering and processing. 

One use case autonomy

Key challenges that are addressed include active sonar and

- Active sonar and clinical ultrasound
- Different ways to achieve adaptivity. Analytical assessment of the data, computing weights. Approximating it. Trial-and-error. Different means to the same adaptivity
- Computational complexity
- Hardware acceleration
- 

Spatial sensitivity is like a balloon; squeezing it on one side inflates it on the other. 

World is highly dynamic, constantly changing from moment to moment. Necessitates an algorithm that adapts to changing data. 



%~~~~~~~~~~~~~~~~~~~~~~~~~~~~~~~~~~~~~~~~~~~~~~~~~~~~~~~~~~~~~~~~~~~~~~~~~~~~~~~~~~~~~~~~~~~~~~~~~~
% \ifMonolithic\else\input{subfooter}\fi


\chapter{Introduction}

The best processing unit we know of is the human mind. Its several billion neurons process information at a staggering rate, and concurrently. As described by this year's nobel price winners, memories of visual objects gets stored in a similar spacial structure in the brain.  

Similar to how the process visual input from our eyes and turn them into the images we perceive, we want turn echo from sonar systems into images of the physical objects the sound was scattered by. 

Want to compare acoustic array to the eyes. Similar, this images from sound should be possible. Brain parallel, and superior. Thus, processing power must be essential to the continued improvement of current sonar systems. But *how* should be additional processing power be used?

One way to improve the performance of imaging devices is to make them more intelligent. 
To improve the performance of modern imaging devices is becoming increasingly intelligent, they adapt their behaviour to best fit the situation they are in. A regular compact camera for instance, adjusts its shutter opening and speed to 

The image quality of any imaging device is mostly its by its hardware components, and .

\section{Aims of this thesis}

\section{Motivation and scope}

\section{Thesis outline}


Signal processing is the art of combining physics, mathematics and computational engineering 

Wonders happen every day. What our senses reveal our brain make real. This is the beauty of signal processing.

- The wonder of how the brain process images. Great information carrier.

\gls{ASIC}

Electromagnetic radiation - poor range due to attenuation and scattering. Use sound instead. Inferior information carrier. Lower frequency and propagation speed.

Sonar. 

- Animals use sound both for imaging and communication underwater.

- Quality. Depends on e.g. sonar frequency, pulse length and directivity. Range resolution given (possibly pulse compressed) signal pulse width. Less data in along-track. resolution given on matched filtered pulse 

- Low PRF - long range. High PRF - short range, high along-track resolution. Propagation time large compared to  SAS, low PRF long antenna. Both long range and high along-track resolution.

- Can use models for sidescan. Main difference improved along-track resolution. Existing models accurate under certain assumptions, but can not be used directly to form side-looking image. Model should include physical processes such as acoustic propagation, seabed reverberation and transducer characteristics.

SAS: Range and frequency independent resolution.


Performance:
- AUV revisit. Better image quality, lower data collection price.


Faster, smarter, better.

big data. needed to evolve ai intelligence. e.g. deep learning.  need processing speed to handle it. especially near sensors

data now enough? we have all of evolution.

processimg loop small. learn to adapt behaviour then and there.

arrays allow focus and width.but requires processing.

human information 2000 titanics of thumbdrives. ai used to map genes 
Preface?


Perhaps the most advanced trait of human intelligence is the ability of ask why. We are masters at both understanding what a problem is, as well as finding solutions to it. 

What if we could create intelligent machines? 

When we face a problem we do not understand, To understand what the problem is

This thesis will eventially narrow down to adaptive processing sonar images, and technicalities of the implementation process.


\chapter{Introduction}

One of the greatest visions in science is that of artificial intelligence. If we could make machines not only do what we say, but also able to adapt their behavior to handle the unexpected, the possibilities are perhaps only limited by our imagination.

One of mankind's greatest dreams is that of creating intelligent machines. Imagine things you can strike a conversation with, that aides us in our daily lives, that never grow old, impatient or boring. Yesterday science fiction, tomorrow reality. It is not a question whether the intelligent machines are coming, but when and how. Around the corner are self driving cars, virtual reality, ...

To understand how we can evolve the intelligence of machines, perhaps it is best to look at how nature has done it. Our 5 basic senses are sight, smell, hearing, taste and touch. The information here is converted from the respective physical phenomena to electrical impulses through various cells. Then these electrical signals travel to a vast network of neurons that make up our brain. Since each neuron is linked to several others, the data dependency is many to many and very hard to understand.


Intelligence is by its essence a feedback loop. 
This requires them to sense their surroundings and To do so, we need equip it with sensors we grant them sensory input by converting energy in various physical forms into electrical signals, process these using analog and digital electronics, and imbue them with  Most of the components for doing so 

Two big topics in recent years are big data and articifical intelligence. 




The best processing unit we know of is the human mind. Its several billion neurons process information at a staggering rate, and concurrently. As described by this year's nobel price winners, memories of visual objects gets stored in a similar spacial structure in the brain.  

Similar to how the process visual input from our eyes and turn them into the images we perceive, we want turn echo from sonar systems into images of the physical objects the sound was scattered by. 

Want to compare acoustic array to the eyes. Similar, this images from sound should be possible. Brain parallel, and superior. Thus, processing power must be essential to the continued improvement of current sonar systems. But *how* should be additional processing power be used?

One way to improve the performance of imaging devices is to make them more intelligent. 
To improve the performance of modern imaging devices is becoming increasingly intelligent, they adapt their behaviour to best fit the situation they are in. A regular compact camera for instance, adjusts its shutter opening and speed to 

The image quality of any imaging device is mostly its by its hardware components, and .

There is no best algorithm/system for all tasks. Best would be to use them all as they complement each other. 

Hard to determine the most vital information in an image. The better the image the more information is contains, and the more processing power and memory bandwidth is needed to handle it.



\section{Aims of this thesis}

\section{Motivation and scope}

\section{Thesis outline}

JIB
Smarter, better, faster - On the future of machine intelligence in active sonar systems 

Preface
  Way too complex a topic to be covered in fully detail. Perhaps impossible?
  Interesting enough to be covered, even if just partially.
  Complexity issue - picture from essay.
     - We observe the world from one location. The internet lets machine see everything.
     - Mental constraint. Can machines break through it?
  
Introduction
   A marvel of science, the world, and beyond
      Intelligence / conscience. What is it?
      HPC bridging learning and big data
      Deep Patient, Google Brain, Nvidia self driving cars, Bloomberg money machine, image and speech analysis, ...
   Information processing
      Instrinsic and extrinsic information
      Data decomposition
      Data volume
      Amount of information, rank, dimensions
      
   Machine intelligence / adaptivity / Deep learning
      Means for intelligence?
      Adapts to data
      Proven powerful
      Hard to understand, hard to control
      Data hungry
      Fascinating perspectives:
      - Local dependencies work on e.g. images. Due to lack of data?
      - First layer similar to template matching (convolitional network)
      - If we can't understand it, perhaps we can parent it? Society's norms a giant assemble of prior knowledge.
      - Deep neural nets interconnected, learning from each other?
      - The brain adapts its wiring to solve tasks, does this carry over?
      - Neuman networks - recursive
      - Energy efficient. But the body does it MUCH better
      - Only make the choices that are relevant. Adaptivity vs. computational/data requirements
        - The body has a lot of information stored in the genome. It doesn't start from scratch
      - On images - convolutional networks
        - Layers offer image feature representation of different scales (intuition: create extrinsic shift invariance)
   Big data
      Fundament of intelligence?
      Amount of data recorded and stored immense. Sensors everywhere.
      Phones, internet searches, smart watches, cars, surveillance cams,   
   High performance computing
      Engine of intelligence?
      Combining deep learning with big data
      Improving at a rapid pace
      Quantum computers seem very adapt at deep learning, with node-like structure
      FPGA might fill a gap too. Hardware generated adaptively. Efficient information exchange often more effective than processing power. Current architectures limited by bandwidth for most applications.
      
      
Background
   Acoustic waves
      Wave equation
   Image reconstruction
      Conventional beamforming
      Adaptive beamforming
         LCA
         MVDR
   Image modeling
      Images based on prior knowledge
      Can be used as input to deep learning
      

Summary of publications
   - Red line: HPC, adaptivity,
   Journal:
      2017 - JOE - Simulator
               
      2017 - JOE - Low-Complexity Adaptive Sonar Imaging
      2015 - JOE - An optimized GPU implementation of the MVDR beamformer for active sonar imaging
      2014 - TUFFC - Implementing capon beamforming on a GPU for real-time cardiac ultrasound imaging 
      
   Related work:
      2014 - UAM2014 - A GPU sonar simulator for automatic target recognition
      2013 - ICA2013 - Adapting the minimum variance beamformer to a graphics processing unit for active sonar imaging systems
      2012 - ECUA2012 - GPU-Based Adaptive Beamforming for Active Sonar Imaging
      2012 - IEEE-US - Implementing Capon Beamforming on the GPU for Real Time Cardiac Ultrasound Imaging
      2011 - UAM2011 - A low complexity adaptive beamformer for active sonar imaging

Discussion and future work
   Summary and discussion
      Simulator:
      - The way to train and constrain deep learning algorithms?
      MVDR:
      - Analytical optimization - full adaptivity.
      - Heavy on computations
      LCA:
      - Optimization by trial and error - discrete solution space.
      - Works almost as well as MVDR in most cases
      - Fast!
      - Well suited for hardware
      
      Lessions from deep learning:
      - 
      
   Future work
   


C-I:

Introduction
   Beamforming
   Wave fields and Array Processing
   Conventional Beamforming
   Adaptive Beamforming
      MVDR
      Capon/MPDR
      APES
      Wiener
   Medical Ultrasound Imaging
   Medical Ultrasound Imaging
   Digital Beamforming
      Compact Beamformers \& Delta Sigma
      Adaptive Beamforming in Medical Ultrasound Imaging
         MVDR-based Beamformers and Related Methods
         The Coherence Factor
         Other Adaptive Methods
Summary of publications
   Paper I-VII
Discussion and future work
   Summary and discussion
   Future Work

   
J-F:

Propagating Waves
Beamforming
  Beampattern
Minimum Variance Beamforming
  Signal Cancellation
  Robust Minimum Variance Beamforming
  Other Adaptive Methods
  Other High Resolution Methods
Medical Ultrasound Imaging
  Broadband Near Field Minimum Variance Beamforming
  Estimation of the Spatial Covarance Matrix
Adaptive Beamforming in Medical Ultrasound Imaging: State of the Art
Blind Source Separation
  Independent Component Analysis
  Relation to Adaptive Beamforming
Summary  of Papers
Discussion
Conclusion and further research

Ann:

Introduction
   Aims of the thesis
   Motivation and scope
   Outline
Background
   Sonar
      Historical Perspective
      Sonar Imaging
      Scattering and Reflection
      Key challenges
   Propagating sound waves
   Array processing and beamforming
      Signal model
      Conventional beamforming
      The beampattern
   Adaptive Beamforming
      Optimal MPDR and MVDR
      Adaptive MVDR
      Adaptive APES
      Low Complexity LCA
      Adaptive methods based on aperture coherence
      Robustness of adaptive beamformers
   Performance metrics
Summary of publications
Summary and Discussion
Future work

Wei:

Introduction
   Overview of fractional calculus
   Challenges of fractional calculus
   Definition of fractional calculus
   Fractional calculus and anomalous phenomena
      Non-Fourier heat conduction
      Modeling arbitrary power law attentuation
   Solving fractional differential equations
      Analytical methods
      Numerical models
   Optimization and parallelization
      Locality and data reuse
      Vectorization
      Parallelization
   Performance modeling
      Serial Performance model
      Parallel Performance model
Summary of papers
Future work
   
   
JP

Background
   General-purpose computing on GPUs
      Comparing CPU and GPy performance
      Programming a GPU
   Medical Ultrasound Imaging
      Beamforming
      Sampling and processing complexity
      GPUs in medical ultrasound imaging
      Adaptive beamforming
      Capon beamforming of focused broadband beams
   Volume rendering
      Adaptive volume rendering
   Ultrasound field simulations
   Concluding remarks
Introduction to papers
   Motivation
   Aims of study
   Summary of papers
   Main contributions
   Discussion and future work
   Multimedia content
   Software
References
Papers


Signal processing is the art of combining physics, mathematics and computational engineering 

Wonders happen every day. What our senses reveal our brain make real. This is the beauty of signal processing.

- The wonder of how the brain process images. Great information carrier.

\gls{ASIC}

Electromagnetic radiation - poor range due to attenuation and scattering. Use sound instead. Inferior information carrier. Lower frequency and propagation speed.

Sonar. 

- Animals use sound both for imaging and communication underwater.

- Quality. Depends on e.g. sonar frequency, pulse length and directivity. Range resolution given (possibly pulse compressed) signal pulse width. Less data in along-track. resolution given on matched filtered pulse 

- Low PRF - long range. High PRF - short range, high along-track resolution. Propagation time large compared to  SAS, low PRF long antenna. Both long range and high along-track resolution.

- Can use models for sidescan. Main difference improved along-track resolution. Existing models accurate under certain assumptions, but can not be used directly to form side-looking image. Model should include physical processes such as acoustic propagation, seabed reverberation and transducer characteristics.

SAS: Range and frequency independent resolution.


Performance:
- AUV revisit. Better image quality, lower data collection price.


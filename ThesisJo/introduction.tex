
Wonders happen every day. What our senses reveal our brain make real. This is the beauty of signal processing.

- The wonder of how the brain process images. Great information carrier.


Electromagnetic radiation - poor range due to attenuation and scattering. Use sound instead. Inferior information carrier. Lower frequency and propagation speed.

Sonar. 

- Animals use sound both for imaging and communication underwater.

- Quality. Depends on e.g. sonar frequency, pulse length and directivity. Range resolution given (possibly pulse compressed) signal pulse width. Less data in along-track. resolution given on matched filtered pulse 

- Low PRF - long range. High PRF - short range, high along-track resolution. Propagation time large compared to  SAS, low PRF long antenna. Both long range and high along-track resolution.

- Can use models for sidescan. Main difference improved along-track resolution. Existing models accurate under certain assumptions, but can not be used directly to form side-looking image. Model should include physical processes such as acoustic propagation, seabed reverberation and transducer characteristics.

SAS: Range and frequency independent resolution.


Performance:
- AUV revisit. Better image quality, lower data collection price.

\section{Scattering models}

Defining parameters:
- Roughness
- Incident angle
- Frequency


Reflection and scattering from surfaces:

Roughness >> wavelength:
- Roughness > wavelength: Helmholtz-Kirchhoff (physical optics approach) (Hovem). Leads to Huygins priciple.
- Roughness < wavelength: Rayleigh-Rice (perturbation technique) (Hovem). More accurate at slightly rough surfaces.

Rayleigh. chi = 2 k sigma sin(theta): Rayleigh criterion: Surface is acoustically smooth when chi << 1.
%&header
% !TeX root = 3_discussion
\input{build/subheader.tex}
\endofdump

\ifRootBuild\else
  \newglossaryentry{ASIC}{name={ASIC},
                  description={Application Specific Integrated Circuit} } 
                  
\newglossaryentry{ATR}{name={ATR},
                  description={Automatic Target Recognition} } 

\newglossaryentry{CPU}{name={CPU},
						description={Central Processing Unit} } 

\newglossaryentry{GPGPU}{name={GPGPU},
						description={General Purpose Graphics Processing Unit} } 

\newglossaryentry{GPU}{name={GPU},
						description={Graphics Processing Unit} } 
					
\newglossaryentry{MVDR}{name={MVDR},
						description={Minimum Variance Distortionless Response} } 
  \makeglossaries
\fi

\begin{document}

\chapter{Discussion and Outlook}

The contributions of this thesis are:

Paper I - Time-domain MVDR beamforming on GPUs in active sonar imaging:
\begin{enumerate}
\item Accelerates the MVDR beamformer through algorithmic reductions and implementation on GPU.
\item Demonstrates speeds exceeding 1\,Mpx/s on systems with no more than 32 channels, two orders of magnitude faster than a semi-optimized C-implementation on a CPU.
\item Offers uncommon insights into hardware limitations, design implications and computation efficiency.
\end{enumerate}

Paper II - Frequency-domain MVDR beamforming on GPUs in clinical ultrasound imaging:
\begin{enumerate}
\item Alters the implementation in Paper I to operate in a frequency domain approximated by its three most significant components.
\item Demonstrates speeds exceeding 1\,Mpx/s on systems with 64 and 96 channels.
\item Yields similar image quality to time-domain MVDR.
\end{enumerate}

Paper III - LCA beamforming compared to MVDR in active sonar imaging:
\begin{enumerate}
\item Contrasts LCA, that selects the best performing weights from a predefined set, to MVDR, that computes the weights from an estimated covariance matrix formed from the input data.
\item 
\item Yields similar image quality to MVDR.
\end{enumerate}

Paper IV - Real-time sonar image synthesis:
\begin{enumerate}
\item Contrasts LCA, that selects the best performing weights from a predefined set, to MVDR, that computes the weights from an estimated covariance matrix formed from the input data.
\item Extends a navigational notation system with topological ideas from robotics and homogeneous coordinates, promoting geometrical clarity and validity while aiding the designer in focusing on higher level concepts, such as manipulating rigid objects.
\item Yields similar image quality to MVDR.
\end{enumerate}



\section{Trends}

The future is faster, smarter.

\subsection{Hardware}

\begin{itemize}
\item Ever increasing processing speed.
\begin{itemize}
	\item Choose the best tool (algorithm) for the job
	\item More complex not always better, but slower
	\item Extracting the essential information is key 
	\item Exploit hardware resources available
   \item Intel entering the discrete graphics stage
   \end{itemize}
\item Software defined transceiver/receivers
\begin{itemize}
\item System on Chips with integrated ADCs/DACs. Big power in small package.
\item Compact, flexible, direct conversion
\item Software defined signal generation, filtering, synchronization, rate-conversion and mixing.
\item Multipurpose, hardware reuse, easier implementation of advanced signal processing, faster development and lower price.
\item 
\item Compact multi-band Wifi
\item MIMO (in new wifi). Breaks array SNR boost. Not viable for long range.
\item Phones: Wifi 2.4-2.5G, 5G (massive MIMO, heavy SDR, 600M-6G), LTE (4G, Long Term Evolution 800M), GSM (2G, Global System for Mobile communications 900M,1.8-1.9M), UMTS (3G, Universal Mobile Telecommunications System 1.9-2.1M), GPS (Global Positioning System 1.5G), Bluetooth (2.4G), NFC (Near-Field Communications 13.56M), FM (100M). Multi-radio in a single SDR package!
\end{itemize}
\item Short feedback loops allow real-time adaptivity
\end{itemize}


\subsection{Algorithms}

\begin{itemize}
\item Intelligence - Ability to process data fast, and retain key information
\item Adaptive methods ever more attractive.
   \begin{itemize}
   \item Processing more data to increase significance of extracted information
   \item Reducing reaction time: Act when the data is fresh/valid.
   \end{itemize}
\item Better image quality
\item Improved autonomy
\end{itemize}


\subsection{Connectivity}

\begin{itemize}
\item Internet of Things
\item Big data
\item Intelligent mesh of nodes
\end{itemize}

\section{Risks \& Challenges}



\begin{itemize}
\item Improving imaging systems by making them smarter.
\item Big data - more processing power.
\item Faster IS smarter
 \begin{itemize}
 	\item Short feedback loops allow real-time adaptivity
 	\item More data contains more information
 	\item Faster learning
 \end{itemize}
\end{itemize}



%Discussion and future work
%   Summary and discussion
%      Simulator:
%      - The way to train and constrain deep learning algorithms?
%      MVDR:
%      - Analytical optimization - full adaptivity.
%      - Heavy on computations
%      LCA:
%      - Optimization by trial and error - discrete solution space.
%      - Works almost as well as MVDR in most cases
%      - Fast!
%      - Well suited for hardware
%      
%      Lessions from deep learning:
%      - 
%      
%   Future work


\section{MVDR}

Recent developments:
- A low complexity minimum variance beamformer for ultrasound imaging using dominant mode rejection~\cite{Jiang2016}
- Google deep beamforming

\begin{itemize}
\item SDR - Entirely software defined radio, radar and sonar, bigger potential for adaptive methods
\item Cognitive radar/sonar. 1. Adaptive signal processing, learn from data. 2. Feedback from receiver to transmitter. 3. Preservation of the information in received data. All of this is present in bats.
\end{itemize}


\section{Onlook}

Machine cognition:
\begin{itemize}
\item Ethical dilemmas. Decision and actions have consequences. Who's legally responsible for a machine's doing? Itself?
\item Safety. Machine flexibility allow more possible outcomes, and a wider range of security issues to consider.
\end{itemize}

R dim > 10000
- Inverting such matrices better to do with gradient decent


Deep learning. Simulator used to trained deep neural networks. Lots of data needed.




\section{}

\end{document}

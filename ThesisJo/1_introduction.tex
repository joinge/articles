%&header
% !TeX root = 1_introduction

\input{build/subheader.tex}
\endofdump

\ifRootBuild\else
  \newglossaryentry{ASIC}{name={ASIC},
                  description={Application Specific Integrated Circuit} } 
                  
\newglossaryentry{ATR}{name={ATR},
                  description={Automatic Target Recognition} } 

\newglossaryentry{CPU}{name={CPU},
						description={Central Processing Unit} } 

\newglossaryentry{GPGPU}{name={GPGPU},
						description={General Purpose Graphics Processing Unit} } 

\newglossaryentry{GPU}{name={GPU},
						description={Graphics Processing Unit} } 
					
\newglossaryentry{MVDR}{name={MVDR},
						description={Minimum Variance Distortionless Response} } 
  \makeglossaries
\fi

%\beforeepigraphskip

\begin{document}



%\renewcommand{\@listI}{%
%\leftmargin=25pt
%\rightmargin=0pt
%\labelsep=5pt
%\labelwidth=20pt
%\itemindent=0pt
%\listparindent=0pt
%\topsep=0pt plus 2pt minus 4pt
%\partopsep=0pt plus 1pt minus 1pt
%\parsep=0pt plus 1pt
%\itemsep=0pt%\parsep
%}

%\renewenvironment{itemize}
%{\begin{itemize}
%% \setlength\abovedisplayskip{0pt}
%% \setlength\belowdisplayskip{0pt}
%% \setlength\abovedisplayshortskip{0pt}
%% \setlength\belowdisplayshortskip{0pt}
%\setlength{\parindent}{0pt}
%\setlength{\itemsep}{0pt}
%\setlength{\parskip}{5pt plus0.3pt minus0.3pt}
%\setlength{\parsep}{0pt}
%
%%   \renewcommand\indent{}
%%   \renewcommand\parindent{}
%}
%{\end{itemize}}
   
%\begin{savequote}[15pc] \slshape
%% % "The best way to predict the future is to invent it."
%% % \qauthor{Alan Kay}
%""
%\qauthor{Douglas Adams---}
%% % The only way to keep your health is to eat what you don't want, drink what you don't like, and do what you'd rather not.
%% % Mark Twain (1835 - 1910)
%% % "For a successful technology, reality must take precedence over public relations, for Nature cannot be fooled." 
%% %   \qauthor{Richard P. Feynman}
%\end{savequote}


\chapter{Introduction}



% \epigraphhead[0]{\epigraph{Deep Thought supercomputer:\\The answer to life, the universe and everything is... 42.}{\textit{The Hitchhikers Guide to the Galaxy}\\\textsc{Douglas Adams}}}
% \epigraphhead[0]{\epigraph{\textit{Cognition:}\\The mental action or process or acquiring knowledge and understanding through thought, experience, and the senses.}{\textsc{Oxford Dictionary}}}

%\section{Aims of this thesis}
%
%No one size fits all. SDR.
%
%\begin{itemize}
%* Accelerate adaptive beamforming
%* Facilitate deep learning
%\end{itemize}

Waves are made of endless tiny energy transactions, a dance of particles effecting each other in search of equilibrium. They are caused by disturbances in the forces of nature. The drop of a rock into calm water or rapid displacement of still air cause mechanical waves, while synchronized oscillations of electric and magnetic fields cause electromagnetic waves. Regardless of whether the waves are mechanical or electromagnetic, move perpendicular or parallel to the axis of propagation, what medium they propagate through or the means by which they do so, they are---perhaps surprisingly---all governed by the same mathematical equations.

This shared mathematical foundation for waves allows the use of similar techniques to form images from waves in optics, medical applications, radars, sonars and seismics. To observe what our eyes and ears can not. To explore vast amounts of information with the sense that does it best: our vision.

Biological creatures sense and react almost instantaneously to waves, adapting their behavior on-the-fly to survive, filling in knowledge gaps, satisfying their curiosity, or honing a skill. These traits help them master the world, but seem notoriously hard to grant machines.


%Raw data -> preprocessing -> beamforming -> image
%                  ^               ^      -> ATR
%          shit in, shit out    adaptive
%          
%\begin{itemize}
%* Improving imaging systems by making them smarter.
%\begin{itemize}
%	* Choose the best tool (algorithm) for the job
%	* More complex not always better, but slower
%	* Extracting the essential information is key 
%	* Exploit hardware resources available
%\end{itemize}
%* Faster IS smarter
% \begin{itemize}
% 	* Short feedback loops allow real-time adaptivity
% 	* More data contains more information
% 	* Faster learning
% \end{itemize}
%* Smarter IS better
% \begin{itemize}
%	* Better image quality
%	* Improved autonomy


\IncludeWritingTipstrue
\newcommand\tip[1]{\ifIncludeWritingTips{}#1\fi}


%Title: Sonar cognition with accelerated acustic image reconstruction and syntesis

% 1. Tells you what question you have
% 2. Tells us what question the paper answers
% 3. Tells what the paper will argue
% 4. Tells us what the paper is about

% Focus on the reader! Tell them what they ought to question.

% 20-25pages max!

% Words of value: Tools, but, although, unrealistic, wrong, incorrect, flawed, etc.

% Introduction - 1 paragraph: Attention grabbing, clarifies purpose, outlays roadmap


% Adaptive methods for image reconstruction and synthesis often 
% Research into HPC, big data and Value added to society when a method 

%Cognitive machines crowning hallmark of human ingenuity. Reactive and proactive. General problem solvers that can aid mankind in its future endeavors. 
%
%Two dominating buzz words are echoing the landscape of modern data science. One is \emph{Deep Learning}, a field that strives to make machines learn the way the brain does. The other is \emph{Big Data}, on the other hand, refers to the fact that we continuously produce, process and store an ever increasing amount of data. 
%
%This is because the merits achieved by combining the two appears to be a huge leap towards the goal of making truly intelligent machines.

%\section{Relevance}
% Argue scientific relevance.
% Use citations (secondary evidence), focus on discussion chapters

%\subsection{Theoretical / scientific}
%\begin{itemize}
%\item Latency matters. Swift reaction key in autonomy, real-time acoustic imagery important in human-machine interactions. (vague)
%\item Faster research progress. Quick iterations.
%\end{itemize}
%
%\subsection{Practical}
%x
%\begin{itemize}
%\item AUVs: Cheaper / more effective missions. Higher quality data. 
%\item More data channels. more channels
%\item Real-time applications. Ultrasound. Responsive AUVs. 
%\item Software defined imaging devices. Radar/sonar/ultrasound. HW becoming cheaper, more compact.
%\end{itemize}

\section{Motivation and scope}

Technology tends to substitute biology's finesse with brute force: throwing superior sensors and massive computing power at the problem, but failing to replicate biology's intricate information processing. Computers never tire, hardly fail, handle information relentlessly at high bandwidth and low latency, and communicate extremely fast---but they lack the ability to truly adapt, learn, be curious and ask the question ``Why?''

%Technology provides better sensors than biology and processing power to deal with their data, but the algorithms lack finesse in doing so.

We seek to improve sonar cognition, or intelligence, through accelerated adaptive image reconstruction and synthesis. This  interdisciplinary challenge delves deeply into three fields:%acoustical imaging, high performance computing and machine learning.
%Three different fields form the foundation for sonar sonar cognition:
%
\begin{itemize}
%
\item \emph{Acoustical imaging}: making images of mechanical waves like sound, vibration, ultrasound and infrasound in gases, liquids and solids. We reconstruct images from 100\,kHz sonars and 3\,MHz clinical ultrasound devices. 
%
\item \emph{Machine learning}: making computers perform a task without explicit instructions, or learn by doing. Analogous to human brains, mathematical models perform \emph{focused learning} by strengthening or weakening connections between information nodes, but struggle to perform \emph{diffused learning} by forming new connections. This feat of biology, called brain plasticity, is unrivaled. We focus on a niche focused learning problem by harnessing today's massive computer power and data.
%
\item \emph{High performance computing}: making efficient use of hardware architectures such as central processing units (CPUs), graphics processing units (GPUs), field-programmable gate-arrays (FPGAs) and application-specific integrated circuits ASICs---listed in descending order with respect to flexibility, general purpose and ease of programming and in ascending order with respect to power efficiency and computing power. We accelerate our algorithms with GPUs.
%Our main inspiration is the human brain, which facilitates focused and diffused learning. Focused mode strengthens the connections between sets of neurons, while diffused mode forms new connections between neurons.
%The adaptive methods in this thesis offer learning with a weak resemblance to focused mode. Diffused mode is harder because hardware architectures are fixed. 
\end{itemize}


%Constrain attention to beamformers and simulators (not ATR?).

%Fast but not as flexible as the brain.
%
%Acceleration refers to speeding up algorithms through reduced computational complexity and implementation on high performance parallel hardware. 
%
%Figure \ref{fig_outline_simple} for aid.
%
%mimicking biology.
%
%Play on technology's strengths: processing speed and adaptive algorithms. Low latency. Massive data. 


% 
% reasoning
% comprehension
% understanding
%
% not enlightenened
% not aware

%it handles information from them inefficiently. 
%
%Want to explore ways to create images 
%
%
%Lots of data, lots of processing power, no skill.
%
% Concerned with acoustical. 
%
%Images great way to interpret lots of information
%Great computational complexity
%Want real-time performance
%Want adaptibility
%Want cognition

% in their search for balance. an endless search for balance. through a mechanical or electromagnetic force. 

%Adaptive
%  transmit
%  mechanics
%
%
%\begin{itemize}
%\item Autonomy (sonar)
%\item Fast-feedback (Ultrasound)
%\item Better research
%\end{itemize}

%\section{Scope}
% 1. Establish research territory
% 2. Introduce / review items of previous research in the area (obligatory)
%     Why important, central, interesting, problematic, etc. (optional)
% 3. Establish niche
%     Indicating gap in previous research or the need to extend it
% 4. Occupying the niche
%     Outline purpose or state nature of present research
%     List research questions or hypotheses
%     Annouce principal findings
%     State value of previous research
% * Be specific
% * Discuss context and background to research problem
% * Emphasize the need and relevance for the researc


%Three key ingredients are needed to brew sonar cognition:
%Facilitate machine learning
%- Acquire information
%- Process it
%- Use it
%
% Three different fields form the foundation for sonar sonar cognition:





%Improve sonar cognition
%
%Questions
%\begin{itemize}
%\item What are the advantages and disadvantages of adaptive methods?
%\item When to use each method?
%\item Are graphics processing units ideal for accelerating these adaptive methods?
%\item Can accelerated adaptive image reconstruction and synthesis improve sonar cognition?
%\end{itemize}


% Evaluate MVDR and LCA 
% - accelerated implementation
% - adaptivity assessment
% - 
% 
% 

% instruments

%Combined -> intelligent detector (focused learning). 


%\subsection{Research territory}
%
%\begin{itemize}
%  \item Image reconstruction / beamforming
%  \item Image synthesis / simulation
%\end{itemize}
%
%Applications in sonar, ultrasound, radar, ...\\
%Dynamic / adaptive:
%
%
%
%\subsection{Previous research}
%
%Most research on adaptive image reconstruction comes from passive systems. The minimum variance distortionless response
%
%\begin{itemize}
%\item Image reconstruction / beamforming
%  \begin{itemize}
%  \item Sonar
%  \item Ultrasound
%  \end{itemize}
%\item ATR
%\end{itemize}
%
%
%\subsection{Real-time adaptivity (niche)}
%
%\begin{itemize}
%\item High speed adaptive image reconstruction (beamforming) \& synthesis (simulation)
%\item Gap: When to use adaptive methods? What can they offer? Is GPU acceleration feasible?
%\end{itemize}
%
%\subsection{GPU acceleration (occupying niche)}
%\begin{itemize}
%\item GPU accelerated image reconstruction (beamforming) \& synthesis (simulation)
%\end{itemize}
%
%
%
%
%\section{Current scientific situation / background}
%% * Research territory -> Previous research -> Niche -> Occupy it
%% * Cite and explain most important related research
%% * Demonstrate field mastery / familiarity
%\begin{itemize}
%  \item 
%\end{itemize}
% 

% \section{Objective \& problem statement}
% 
% %% * Within first 3 pages, 1 page max
% %% * ARGUMENT conceptualized around a PROBLEM that meets the needs of a DISCOURSE COMMUNITY
% %% * Making CLAIMS with support from EVIDENCE to convince your audience
% %% * Problem rationale (1-2s): "The problem here is... (foreground)" (brainstorming: any)(intro) 
% %% * Knowledge gap (1-2s) - literature. Academic rationale. (l.review)
% %% * Context - time, place, people (intro)
% %% * Conceptual framework (optional) (l.review)
% %% * Evidence (primary: own, secondary: peer reviewd) / logic
% %% * (methods) 1-3 sentences: "The purpose of this research is to..." (must close knowledge gap!)
% %% * Expand methodology (1-2s) (optional)
% %% * Questions (findings/discussion):
% %%   * PhD 5 questions
% %%   * Aim to "open up" problem rationale, not introduce more problems.
% 
% 
% 
% Improved sonar cognition through accelerated adaptive image reconstruction and synthesis
% 
% Objective: Improving sonar cognition by accelerating and simplifying adaptive beamformers to make them real-time capable.
% 
% Problem: Adapting fast enough (vague)
% 
% Machine cognition is not human cognition. Have to play on machine strenghts. 
% 
% 
% Intelligent detector. Want to balance 
% 
% Could be ideal theoretical solution, but not practical. 
% 
% The more complicated the model, the more profound its assumptions. Breaking those breaks the model. More risky. Less stable. 
% 
% In real-world scenarios, practical system, too many uncertainties - some of which breaks with the complicated assumptions that the complicated models make. 
% 
% Mathematics is precise, elegant, the world is messy. 
% 
% Fancy algorithms not necessarily needed
% 
% What is information? Twice the amount of information? Adapt to just what is significant. Difficult to tell it apart from non-significant. 

%Questions
%\begin{itemize}
%  \item Can GPUs enable real-time MVDR?
%  \item When to does frequency domain MVDR on GPU make sence?
%  \item Is LCA a viable alternative, and if so, why?
%  \item GPUs in image analysis / classification?
%\end{itemize}
%
%CLAIMS \& EVIDENCE:
%\begin{itemize}
%\item 
%\end{itemize}
%
% Why change anything? 
% Why fix what isn't broken?
% who, what, when, where, why, how


%- 
%- Can adaptive image reconstruction 
%
%- Who benefits
%- What 
%- When to use which method?
%- Where can this be applied
%- Why bother
%- How 
%
%- How can current adaptive image reconstruction methods 
%- Can GPUs 
%
%Questions
%\begin{itemize}
%  \item Can GPUs enable real-time MVDR?
%  \item When to does frequency domain MVDR on GPU make sence?
%  \item Is LCA a viable alternative, and if so, why?
%  \item GPUs in image analysis / classification?
%\end{itemize}


\section{Outline}
%\ifIncludeWritingTips
%\begin{markdown}
%    * Describe the context within which the research takes place
%    * Summarize methodology as well
%* Conceptual framework (optional)
%    * Specify the the thesis frame
%* Research significance (limitations, scope)
%    * Why is it important? Elaborate on assumptions, limitations, scope
%* Expected outcomes (?)
%* Overview of chapters
%    * Tell reader what to expect.
%    * Form the basis for establishing arguments and key points
%* Concluding paragraph
%    * Summarize key points
%    * Links to next chapter(?)
%\end{markdown}
%\fi

\begin{figure}[!thb]
\makebox[\linewidth][c]{%
\graphicsAI[drawing,width=1.3\linewidth]{gfx/article_relation_bigger_picture.svg}}
\caption{Thesis outline. The first three papers are concerned with beamforming: the reconstruction of images from acoustic echoes. The last paper is concerned with simulation: the synthesis of image templates from 3D models and object parameters estimated from the reconstructed image. %An object classifiers receives the sonar image and its simulated counterpart and  is fed to an object classifier
}\label{fig_outline_simple}
\end{figure}

This thesis presents four journal articles, related visually in Figure \ref{fig_outline_simple}. The three first papers are concerned with making images from acoustic echoes, and the last paper uses parameters derived from such images to simulate image templates of common objects. Our practical system---an autonomous underwater vehicle (AUV)---uses the templates to classify objects in the scene and adapt accordingly. 

Chapter 2 provides some background material on beamforming. Chapter 3 summarizes the contributions of all four papers. Chapter 4 discusses findings, proposes further work and entertains ideas of what the future holds.

%\begin{lstlisting}
%Title: Sonar cognition with accelerated acustic image reconstruction and syntesis
%
%Sonar cognition(???) - introduction
%   Intelligence - Ability to acquire and apply knowledge and skills (that which learns and acts on information)
%      Filtering:  Optimize quality / significance
%      Processing: Throughput / latency
%      Storage:    Raw data or model
%      Feedback:   Immediate action/ direct causality / real-time adaptivity / short loop
%      Action:     Autonomy
%      Examples:   Mapping of genes, Self-driving cars, 2000 Titanics of data
%      
%   Information: Big data. Quality is key.
%   Algorithms: "Intelligent". *Adaptive*. Dynamic. Learning (feedback).
%   Processing: Faster. Smaller. Low-power. Dynamic optimization. 
%   Feedback: Cost associated with delayed action. 
%   Speed: Data compression/significance, Data-/task parallelity, hardware compatibility  
%   Future: Discover what we can't? Do what we won't?
%   
%Image quality
%   Standard metrics: Resolution, contrast/dynamic range (SNR)
%   Ideal metrics: Realistic. Information accuracy and quality.
%   Goal: Describe reality as best as possible.
%   
%Beamforming
%   Directionnal sensitivity
%   No need to physically move array
%   Dynamic behaviour require rapid processing
%
%Adaptive beamforming
%   Dynamic directionnal sensitivity
%
%Algorithms:
%   Beamformers: Conventional, adaptive
%   
%
%Hardware
%   Front-end: 
%      Sensor array: Convert energy to electrical signals, amplify, (de-)modulate, filter
%      Arrays: Boosts SNR. Recorded data spans space, not just time. Allow steering.
%   Processing:
%      Graphics Processing Units: Data parallell, both graphics/general purpose.
%      
%Key technical terms:
%   Array processing
%   (Adaptive) beamforming
%      Degrees of freedom, solution space
%   Graphics processors
%
%Summary of publications
%
%1. MVDR Sonar, exact, <32 channels
%   Adaptive, MV element space, reference
%   Exact solution
%   Full albeit constrained solution space
%   $O(M^3)$. All-to-all dependencies, GPU semi-optimal
%    
%2. MVDR UltraSound
%   Adaptive, MV beam space
%   Approximate solution
%   Full abeit constrained solution space 
%   >32 channels
%   $O(B^3)$. All-to-all dependencies, GPU semi-optimal
%   
%3. LCA
%   New, less known, practical
%   Adaptive, discrete solution space
%   $O(W^2)$. Perfect data parallellity, GPU optimal
%   Sonar verified
%   
%4. ATRSimulator
%   Image synthesis
%   ATR via template matching
%   High throughput avoid database
%   Improves classification
%   
%Future:
%   Software Defined Radio: Flexible, adaptive, cheap, rapid development.
%   SoCs: True heterogeneous hardware architectures
%   
%Discussion and future work
%   Summary and discussion
%   Future Work
%\end{lstlisting}

%\section{Scope}


%\section{Motivation and scope}

%
%Faster, smarter, better.
%
%big data. needed to evolve ai intelligence. e.g. deep learning.  need processing speed to handle it. especially near sensors
%
%data now enough? we have all of evolution.
%
%
%arrays allow focus and width.but requires processing.
%
%human information 2000 titanics of thumbdrives. ai used to map genes 
%Preface?
%
%
%Perhaps the most advanced trait of human intelligence is the ability of ask why. We are masters at both understanding what a problem is, as well as finding solutions to it. 
%
%What if we could create intelligent machines? 
%
%When we face a problem we do not understand, To understand what the problem is
%
%This thesis will eventially narrow down to adaptive processing sonar images, and technicalities of the implementation process.



%One of the greatest visions in science is that of artificial intelligence. If we could make machines not only do what we say, but also able to adapt their behavior to handle the unexpected, the possibilities are perhaps only limited by our imagination.
%
%One of mankind's greatest dreams is that of creating intelligent machines. Imagine things you can strike a conversation with, that aides us in our daily lives, that never grow old, impatient or boring. Yesterday science fiction, tomorrow reality. It is not a question whether the intelligent machines are coming, but when and how. Around the corner are self driving cars, virtual reality, ...
%
%To understand how we can evolve the intelligence of machines, perhaps it is best to look at how nature has done it. Our 5 basic senses are sight, smell, hearing, taste and touch. The information here is converted from the respective physical phenomena to electrical impulses through various cells. Then these electrical signals travel to a vast network of neurons that make up our brain. Since each neuron is linked to several others, the data dependency is many to many and very hard to understand.
%
%
%This requires them to sense their surroundings and To do so, we need equip it with sensors we grant them sensory input by converting energy in various physical forms into electrical signals, process these using analog and digital electronics, and imbue them with  Most of the components for doing so 
%
%Two big topics in recent years are big data and articifical intelligence. 
%
%
%
%
%The best processing unit we know of is the human mind. Its several billion neurons process information at a staggering rate, and concurrently. As described by this year's nobel price winners, memories of visual objects gets stored in a similar spacial structure in the brain.  
%
%Similar to how the process visual input from our eyes and turn them into the images we perceive, we want turn echo from sonar systems into images of the physical objects the sound was scattered by. 
%
%Want to compare acoustic array to the eyes. Similar, this images from sound should be possible. Brain parallel, and superior. Thus, processing power must be essential to the continued improvement of current sonar systems. But *how* should be additional processing power be used?
%
%One way to improve the performance of imaging devices is to make them more intelligent. 
%To improve the performance of modern imaging devices is becoming increasingly intelligent, they adapt their behaviour to best fit the situation they are in. A regular compact camera for instance, adjusts its shutter opening and speed to 
%
%The image quality of any imaging device is mostly its by its hardware components, and .
%
%There is no best algorithm/system for all tasks. Best would be to use them all as they complement each other. 
%
%Hard to determine the most vital information in an image. The better the image the more information is contains, and the more processing power and memory bandwidth is needed to handle it.
%
%
%
%\section{Aims of this thesis}
%
%\section{Motivation and scope}
%
%\section{Thesis outline}
%
%JIB
%Smarter, better, faster - On the future of machine intelligence in active sonar systems 
%
%Preface
%  Way too complex a topic to be covered in fully detail. Perhaps impossible?
%  Interesting enough to be covered, even if just partially.
%  Complexity issue - picture from essay.
%     - We observe the world from one location. The internet lets machine see everything.
%     - Mental constraint. Can machines break through it?
%  
%Introduction
%   A marvel of science, the world, and beyond
%      Intelligence / conscience. What is it?
%      HPC bridging learning and big data
%      Deep Patient, Google Brain, Nvidia self driving cars, Bloomberg money machine, image and speech analysis, ...
%   Information processing
%      Instrinsic and extrinsic information
%      Data decomposition
%      Data volume
%      Amount of information, rank, dimensions, degrees of freedom
%      
%   Machine intelligence / adaptivity / Deep learning
%      Means for intelligence?
%      Adapts to data
%      Proven powerful
%      Hard to understand, hard to control
%      Data hungry
%      Fascinating perspectives:
%      - Local dependencies work on e.g. images. Due to lack of data?
%      - First layer similar to template matching (convolitional network)
%      - If we can't understand it, perhaps we can parent it? Society's norms a giant assemble of prior knowledge.
%      - Deep neural nets interconnected, learning from each other?
%      - The brain adapts its wiring to solve tasks, does this carry over?
%      - Neuman networks - recursive
%      - Brain focused vs. diffused state, chunk nurturing vs. forming
%      - Energy efficient. But the body does it MUCH better
%      - Only make the choices that are relevant. Adaptivity vs. computational/data requirements
%        - The body has a lot of information stored in the genome. It doesn't start from scratch
%      - On images - convolutional networks
%        - Layers offer image feature representation of different scales (intuition: create extrinsic shift invariance)
%   Big data
%      Fundament of intelligence?
%      Amount of data recorded and stored immense. Sensors everywhere.
%      Phones, internet searches, smart watches, cars, surveillance cams,   
%   High performance computing
%      Engine of intelligence?
%      Combining deep learning with big data
%      Improving at a rapid pace
%      Quantum computers seem very adapt at deep learning, with node-like structure
%      FPGA might fill a gap too. Hardware generated adaptively. Efficient information exchange often more effective than processing power. Current architectures limited by bandwidth for most applications.
%      
%      
%Background
%   Acoustic waves
%      Wave equation
%   Image reconstruction
%      Conventional beamforming
%      Adaptive beamforming
%         LCA
%         MVDR
%   Image modeling
%      Images based on prior knowledge
%      Can be used as input to deep learning
%      
%
%Summary of publications
%   - Red line: HPC, adaptivity,
%   Journal:
%      2017 - JOE - Simulator
%               
%      2017 - JOE - Low-Complexity Adaptive Sonar Imaging
%      2015 - JOE - An optimized GPU implementation of the MVDR beamformer for active sonar imaging
%      2014 - TUFFC - Implementing capon beamforming on a GPU for real-time cardiac ultrasound imaging 
%      
%   Related work:
%      2014 - UAM2014 - A GPU sonar simulator for automatic target recognition
%      2013 - ICA2013 - Adapting the minimum variance beamformer to a graphics processing unit for active sonar imaging systems
%      2012 - ECUA2012 - GPU-Based Adaptive Beamforming for Active Sonar Imaging
%      2012 - IEEE-US - Implementing Capon Beamforming on the GPU for Real Time Cardiac Ultrasound Imaging
%      2011 - UAM2011 - A low complexity adaptive beamformer for active sonar imaging
%

%   
%
%\gls{ASIC}
%C-I:
%
%Introduction
%   Beamforming
%   Wave fields and Array Processing
%   Conventional Beamforming
%   Adaptive Beamforming
%      MVDR
%      Capon/MPDR
%      APES
%      Wiener
%   Medical Ultrasound Imaging
%   Medical Ultrasound Imaging
%   Digital Beamforming
%      Compact Beamformers \& Delta Sigma
%      Adaptive Beamforming in Medical Ultrasound Imaging
%         MVDR-based Beamformers and Related Methods
%         The Coherence Factor
%         Other Adaptive Methods
%Summary of publications
%   Paper I-VII
%Discussion and future work
%   Summary and discussion
%   Future Work
%
%   
%J-F:
%
%Propagating Waves
%Beamforming
%  Beampattern
%Minimum Variance Beamforming
%  Signal Cancellation
%  Robust Minimum Variance Beamforming
%  Other Adaptive Methods
%  Other High Resolution Methods
%Medical Ultrasound Imaging
%  Broadband Near Field Minimum Variance Beamforming
%  Estimation of the Spatial Covarance Matrix
%Adaptive Beamforming in Medical Ultrasound Imaging: State of the Art
%Blind Source Separation
%  Independent Component Analysis
%  Relation to Adaptive Beamforming
%Summary  of Papers
%Discussion
%Conclusion and further research
%
%Ann:
%
%Introduction
%   Aims of the thesis
%   Motivation and scope
%   Outline
%Background
%   Sonar
%      Historical Perspective
%      Sonar Imaging
%      Scattering and Reflection
%      Key challenges
%   Propagating sound waves
%   Array processing and beamforming
%      Signal model
%      Conventional beamforming
%      The beampattern
%   Adaptive Beamforming
%      Optimal MPDR and MVDR
%      Adaptive MVDR
%      Adaptive APES
%      Low Complexity LCA
%      Adaptive methods based on aperture coherence
%      Robustness of adaptive beamformers
%   Performance metrics
%Summary of publications
%Summary and Discussion
%Future work
%
%Wei:
%
%Introduction
%   Overview of fractional calculus
%   Challenges of fractional calculus
%   Definition of fractional calculus
%   Fractional calculus and anomalous phenomena
%      Non-Fourier heat conduction
%      Modeling arbitrary power law attentuation
%   Solving fractional differential equations
%      Analytical methods
%      Numerical models
%   Optimization and parallelization
%      Locality and data reuse
%      Vectorization
%      Parallelization
%   Performance modeling
%      Serial Performance model
%      Parallel Performance model
%Summary of papers
%Future work
%   
%   
%JP
%
%Background
%   General-purpose computing on GPUs
%      Comparing CPU and GPy performance
%      Programming a GPU
%   Medical Ultrasound Imaging
%      Beamforming
%      Sampling and processing complexity
%      GPUs in medical ultrasound imaging
%      Adaptive beamforming
%      Capon beamforming of focused broadband beams
%   Volume rendering
%      Adaptive volume rendering
%   Ultrasound field simulations
%   Concluding remarks
%Introduction to papers
%   Motivation
%   Aims of study
%   Summary of papers
%   Main contributions
%   Discussion and future work
%   Multimedia content
%   Software
%References
%Papers
%
%
%Signal processing is the art of combining physics, mathematics and computational engineering 
%
%Wonders happen every day. What our senses reveal our brain make real. This is the beauty of signal processing.
%
%- The wonder of how the brain process images. Great information carrier.
%
%\gls{ASIC}
%
%Electromagnetic radiation - poor range due to attenuation and scattering. Use sound instead. Inferior information carrier. Lower frequency and propagation speed.
%
%Sonar. 
%
%- Animals use sound both for imaging and communication underwater.
%
%- Quality. Depends on e.g. sonar frequency, pulse length and directivity. Range resolution given (possibly pulse compressed) signal pulse width. Less data in along-track. resolution given on matched filtered pulse 
%
%- Low PRF - long range. High PRF - short range, high along-track resolution. Propagation time large compared to  SAS, low PRF long antenna. Both long range and high along-track resolution.
%
%- Can use models for sidescan. Main difference improved along-track resolution. Existing models accurate under certain assumptions, but can not be used directly to form side-looking image. Model should include physical processes such as acoustic propagation, seabed reverberation and transducer characteristics.
%
%SAS: Range and frequency independent resolution.
%
%
%Performance:
%- AUV revisit. Better image quality, lower data collection price.



\end{document}

% Template for SiPS-2010 paper; to be used with:
%          spconf.sty  - ICASSP/ICIP LaTeX style file, and
%          IEEEbib.bst - IEEE bibliography style file.
% --------------------------------------------------------------------------
\documentclass{article}
\usepackage{spconf,amsmath,epsfig}

% Example definitions.
% --------------------
%\def\x{{\mathbf x}}
%\def\L{{\cal L}}

%%% My packages
\usepackage{amsfonts}

%%% My commands
\newcommand{\mat}[1]{\mathbf{#1}}
\renewcommand{\vec}[1]{\mathbf{#1}}

% Title.
% ------
\title{Implementing Adaptive Beamforming on the GPU for Real Time Ultrasound Imaging}
%
% Single address.
% ---------------
\name{J. P. \AA{}sen$^1$, J. I. Buskenes$^2$, C.-I. C. Nilsen$^2$, A. Austeng$^2$ and S. Holm$^{1,2}$}
\address{$^1$Mi Lab, Norwegian University of Science and Technology, Trondheim, Norway \\ $^2$Department of Informatics, University of Oslo, Oslo, Norway}
%
% For example:
% ------------
%\address{School\\
%	Department\\
%	Address}
%
% Two addresses (uncomment and modify for two-address case).
% ----------------------------------------------------------
%\twoauthors
%  {A. Author-one, B. Author-two\sthanks{Thanks to XYZ agency for funding.}}
%	{School A-B\\
%	Department A-B\\
%	Address A-B}
%  {C. Author-three, D. Author-four\sthanks{The fourth author performed the work
%	while at ...}}
%	{School C-D\\
%	Department C-D\\
%	Address C-D}
%
\begin{document}
%\ninept
%
\maketitle
%
\begin{abstract}

\end{abstract}

\section{Introduction}\label{sec:intro}
Introduction to Capon adaptive beamforming, ultrasound imaging and GPU/CUDA.

Delay and sum beamformer (DAS):
\begin{align}
z[n] = \sum_{m = 0}^{M-1}\vec{w}_m^*\vec{x}_m[n - \Delta_m[n]] = \vec{w}^H\vec{x}[n],
\end{align}
where $M$ is the number of elements.

Capon weights are found by solving the following minimization problem:
\begin{align}
&\min_{\vec{w}} E\{|z[n]|^2\} = \vec{w}^H\mat{R}\vec{w}\\
&\text{subjected to } \vec{w}^H\vec{1} = 1,
\end{align}
where $\mat{R} = E\{\vec{x}[n]\vec{x}[n]^H\}$.

The solution to this minimization problem is
\begin{align}
\vec{w}[n] = \frac{\mat{R}[n]^{-1}\vec{a}}{\vec{a}^H\mat{R}[n]^{-1}\vec{a}},
\end{align}
where $\vec{a} = \vec{1}$ when $\vec{x}$ are pre-delayed and $\vec{w} \in \mathbb{C}^M$.

Estimation of $\mat{R}[n]$: $\mat{\hat{R}}[n] = \vec{x}[n]\vec{x}[n]^H$.

In order to get a well conditioned $\mat{\hat{R}}$, avoid signal cancellation and to get DAS-like speckle, $\mat{\hat{R}}$ has to be averaged over $L\le M/2$ subarrays and $N_{avg}$ time samples.
\begin{align}
\mat{\hat{R}} = \frac{1}{(2N_{avg} + 1)K}\sum_{n'=n-N_{avg}}^{n+N_{avg}} \sum_{l=0}^{K} \vec{x}_l[n']\vec{x}_l[n']^H,
\end{align}
where $K = M-L+1$, $Y_{avg}\sim \tau/T_s$ ($\tau$ is the imaging pulse length and $T_s$ is the sampling period.) and $\vec{x}_l$ is the $l^\text{th}$ subarray $[x_l[n], \dotso x_{l+L}[n]]$.

Building and calculating the inverse of $\hat{R}$ is computational demanding, limiting the methods accessibility. Following the innovation in GPU computing, this has now started to change.

\section{Method}\label{sec:meth}
\subsection{Building the Sample Covariance Matrix}
Building $\hat{\mat{R}}$ using a sliding window approach across $K$ and $N_{avg}$. We have a limited amount of fast, near-core memory on the GPU. On NVIDIA architecture this memory is known as shared memory, and is restricted to 48KB per compute block. The maximum block size of 32x32, further restrict how large arrays we can handle inside a single compute block. We therefore restrict $L$ to a maxima of $32$ elements. Since $L \le M/2$, $M \le 64$. We can not afford to hold the full $\hat{\mat{R}}$ ($M = L$) when $\hat{\mat{R}} > 32$, because of limitied amount of memory. ... Explain in detail how \texttt{buildR} is implemented in a kernel. ... Give more details on how much shared memory we have, and how it can be divided between compute blocks occupying one stream multiprocessor (SM).

\subsection{Solving Multiple Small Linear Systems}
Most focus on large matrices in the literature.
We have used an unreleased GPU implementation of Gauss Jordan (GJ) elimination (by NVIDIA) to solve $\mat{R}\vec{b} = \vec{1}$, where $\vec{b} = \mat{R}^{-1}\vec{1}$. Can include details on GPU implementation of $\mat{U}^H\mat{D}\mat{U}$, but this has not proved to be faster than NVIDIA's GJ implementation. However, the complexity for solving with $\mat{U}^H\mat{D}\mat{U}$ decomposition is supposed to be $1/3$ of GJ.

\section{Results}\label{sec:res}

Present images before and after Capon weights has been applied. Comment on the effect of selecting different parameters. Present graph showing running times for Cython-Capon v.s. CUDA-Capon for different choices of parameters. 

(A prerequisite for this is to get the GPU-processing up and running on simulated data before Easter!! Maybe also a Matlab plug-in.)

\section{Discussion}\label{sec:dis}
Discuss results presented in Section \ref{sec:res}.

\section{Conclusion}\label{sec:con}
...


%\begin{abstract}
%The abstract should appear at the top of the left-hand column of text, about
%0.5 inch (12 mm) below the title area and no more than 3.125 inches (80 mm) in
%length.  Leave a 0.5 inch (12 mm) space between the end of the abstract and the
%beginning of the main text.  The abstract should contain about 100 to 150
%words, and should be identical to the abstract text submitted electronically
%along with the paper cover sheet.  All manuscripts must be in English, printed
%in black ink.
%\end{abstract}
%%
%\begin{keywords}
%One, two, three, four, five
%\end{keywords}
%%
%\section{Introduction}
%\label{sec:intro}
%
%These guidelines include complete descriptions of the fonts, spacing, and
%related information for producing your proceedings manuscripts. Please follow
%them and if you have any questions, direct them to Conference Management
%Services, Inc.: Phone +1-979-846-6800 or Fax +1-979-846-6900 or email
%to papers@sips2010.org.
%
%\section{Formatting your paper}
%\label{sec:format}
%
%All printed material, including text, illustrations, and charts, must be kept
%within a print area of 7 inches (178 mm) wide by 9 inches (229 mm) high. Do
%not write or print anything outside the print area. The top margin must be 1
%inch (25 mm), except for the title page, and the left margin must be 0.75 inch
%(19 mm).  All {\it text} must be in a two-column format. Columns are to be 3.39
%inches (86 mm) wide, with a 0.24 inch (6 mm) space between them. Text must be
%fully justified.
%
%\section{PAGE TITLE SECTION}
%\label{sec:pagestyle}
%
%The paper title (on the first page) should begin 1.38 inches (35 mm) from the
%top edge of the page, centered, completely capitalized, and in Times 14-point,
%boldface type.  The authors' name(s) and affiliation(s) appear below the title
%in capital and lower case letters.  Papers with multiple authors and
%affiliations may require two or more lines for this information.
%
%\section{TYPE-STYLE AND FONTS}
%\label{sec:typestyle}
%
%To achieve the best rendering both in the proceedings and from the CD-ROM, we
%strongly encourage you to use Times-Roman font.  In addition, this will give
%the proceedings a more uniform look.  Use a font that is no smaller than nine
%point type throughout the paper, including figure captions.
%
%In nine point type font, capital letters are 2 mm high.  If you use the
%smallest point size, there should be no more than 3.2 lines/cm (8 lines/inch)
%vertically.  This is a minimum spacing; 2.75 lines/cm (7 lines/inch) will make
%the paper much more readable.  Larger type sizes require correspondingly larger
%vertical spacing.  Please do not double-space your paper.  True-Type 1 fonts
%are preferred.
%
%The first paragraph in each section should not be indented, but all the
%following paragraphs within the section should be indented as these paragraphs
%demonstrate.
%
%\section{MAJOR HEADINGS}
%\label{sec:majhead}
%
%Major headings, for example, "1. Introduction", should appear in all capital
%letters, bold face if possible, centered in the column, with one blank line
%before, and one blank line after. Use a period (".") after the heading number,
%not a colon.
%
%\subsection{Subheadings}
%\label{ssec:subhead}
%
%Subheadings should appear in lower case (initial word capitalized) in
%boldface.  They should start at the left margin on a separate line.
% 
%\subsubsection{Sub-subheadings}
%\label{sssec:subsubhead}
%
%Sub-subheadings, as in this paragraph, are discouraged. However, if you
%must use them, they should appear in lower case (initial word
%capitalized) and start at the left margin on a separate line, with paragraph
%text beginning on the following line.  They should be in italics.
%
%\section{PRINTING YOUR PAPER}
%\label{sec:print}
%
%Print your properly formatted text on high-quality, 8.5 x 11-inch white printer
%paper. A4 paper is also acceptable, but please leave the extra 0.5 inch (12 mm)
%empty at the BOTTOM of the page and follow the top and left margins as
%specified.  If the last page of your paper is only partially filled, arrange
%the columns so that they are evenly balanced if possible, rather than having
%one long column.
%
%In LaTeX, to start a new column (but not a new page) and help balance the
%last-page column lengths, you can use the command ``$\backslash$pagebreak'' as
%demonstrated on this page (see the LaTeX source below).
%
%\section{PAGE NUMBERING}
%\label{sec:page}
%
%Please do {\bf not} paginate your paper.  Page numbers, session numbers, and
%conference identification will be inserted when the paper is included in the
%proceedings.
%
%\section{ILLUSTRATIONS, GRAPHS, AND PHOTOGRAPHS}
%\label{sec:illust}
%
%Illustrations must appear within the designated margins.  They may span the two
%columns.  If possible, position illustrations at the top of columns, rather
%than in the middle or at the bottom.  Caption and number every illustration.
%All halftone illustrations must be clear black and white prints.  Do not use
%any colors in illustrations.
%
%Since there are many ways, often incompatible, of including images (e.g., with
%experimental results) in a LaTeX document, below is an example of how to do
%this \cite{Lamp86}.
%
%% Below is an example of how to insert images. Delete the ``\vspace'' line,
%% uncomment the preceding line ``\centerline...'' and replace ``imageX.ps''
%% with a suitable PostScript file name.
%% -------------------------------------------------------------------------
%\begin{figure}[htb]
%
%\begin{minipage}[b]{1.0\linewidth}
%  \centering
%% \centerline{\epsfig{figure=image1.ps,width=8.5cm}}
%  \vspace{2.0cm}
%  \centerline{(a) Result 1}\medskip
%\end{minipage}
%%
%\begin{minipage}[b]{.48\linewidth}
%  \centering
%% \centerline{\epsfig{figure=image3.ps,width=4.0cm}}
%  \vspace{1.5cm}
%  \centerline{(b) Results 3}\medskip
%\end{minipage}
%\hfill
%\begin{minipage}[b]{0.48\linewidth}
%  \centering
%% \centerline{\epsfig{figure=image4.ps,width=4.0cm}}
%  \vspace{1.5cm}
%  \centerline{(c) Result 4}\medskip
%\end{minipage}
%%
%\caption{Example of placing a figure with experimental results.}
%\label{fig:res}
%%
%\end{figure}
%
%% To start a new column (but not a new page) and help balance the last-page
%% column length use \vfill\pagebreak.
%% -------------------------------------------------------------------------
%\vfill
%\pagebreak
%
%
%\section{FOOTNOTES}
%\label{sec:foot}
%
%Use footnotes sparingly (or not at all!) and place them at the bottom of the
%column on the page on which they are referenced. Use Times 9-point type,
%single-spaced. To help your readers, avoid using footnotes altogether and
%include necessary peripheral observations in the text (within parentheses, if
%you prefer, as in this sentence).
%
%
%\section{COPYRIGHT FORMS}
%\label{sec:copyright}
%
%You must include your fully completed, signed IEEE copyright release form when
%you submit your paper. We {\bf must} have this form before your paper can be
%published in the proceedings.  The copyright form is available as a Word file,
%a PDF file, and an HTML file. You can also use the form sent with your author
%kit.
%
%\section{REFERENCES}
%\label{sec:ref}
%
%List and number all bibliographical references at the end of the paper.  The references can be numbered in alphabetic order or in order of appearance in the document.  When referring to them in the text, type the corresponding reference number in square brackets as shown at the end of this sentence \cite{C2}.
%
%% References should be produced using the bibtex program from suitable
%% BiBTeX files (here: strings, refs, manuals). The IEEEbib.bst bibliography
%% style file from IEEE produces unsorted bibliography list.
%% -------------------------------------------------------------------------
%\bibliographystyle{IEEEbib}
%\bibliography{strings,refs}

\end{document}

\section{Motivation}
Rapid development in computer game technology and accompanying programming languages have recently provided researchers with small personal super computers, comprised in a single graphics card (\nom{GPU}{Graphics Processing Unit}). The latest graphics cards\footnote{As of January 2014} from both Nvidia and AMD provides close to 6 T\nom{FLOPS}{Floating point operations per second} single precision and 1.5 TFLOPS double precision. For double precision this is more than the worlds most powerful supercomputer provided in 1998.

This immense rise in computational power and programming flexibility gives new possibilities when designing ultrasound imaging systems. Algorithms which previously had to be implemented in hardware for performance reasons can now be moved to software. It also means that algorithms, either too complicated to implement in hardware, or though to be too computationally heavy for real-time use, becomes realizable. It is clear that high performance programmable processors, like the GPU, have and will make future ultrasound imaging systems more flexible, cheaper to produce, and equipped with even more cutting-edge processing.

\section{Aims of study}
The aim of this thesis has been to investigate the possibility of utilizing GPUs for advanced processing in an ultrasound imaging system. 

%\subsubsection{Accelerate the Capon beamformer to facilitate real time imaging}

%\subsubsection{}

\section{Summary of papers}

\subsection{Paper I}
\textbf{An Optimised GPU Implementation of the MVDR Beamformer for Active Sonar Imaging}\\
J.\:I.\:Buskenes, \textbf{J.\:P.\:\AA{}sen}, C.-I.\:C.\:Nilsen and A.\:Austeng\\
{\it IEEE Transactions on Oceanic Engineering, submitted.}\\\\
Summary of paper...

\subsection{Paper II}
\textbf{Implementing Capon Beamforming on a GPU for Real-Time Cardiac Ultrasound Imaging}\\
\textbf{J.\:P.\:\AA{}sen}, J.\:I.\:Buskenes, C.-I.\:C\:Nilsen, A.\:Austeng and S.\:Holm\\
{\it IEEE Transactions on Ultrasonics, Ferroelectrics, and Frequency Control, vol. 61, no. 1, pp. \todo{add pages}, Jan. 2014.}\\\\
Summary of paper...

\subsection{Paper III}
\textbf{Capon Beamforming and Moving Objects - An Analysis of Lateral Shift-Invariance}\\
\textbf{J.\:P.\:\AA{}sen}, A.\:Austeng and S.\:Holm\\
{\it IEEE Transactions on Ultrasonics, Ferroelectrics, and Frequency Control, submitted.}
Summary of paper...

\subsection{Paper IV}
\textbf{Adaptive Volume Rendering of Cardiac 3D Ultrasound Images - Utilizing Blood Pool Statistics}\\
\textbf{J.\:P.\:\AA{}sen}, E.\:Steen, G.\:Kiss, A.\:Thorstensen and S.\:I.\:Rabben\\
{\it Proc. SPIE Medical Imaging 2012, \todo{add pages}.}\\\\
Summary of paper...

\subsection{Paper V}
\textbf{Huygens on Speed: Interactive Simulation of Ultrasound Pressure Fields}\\
\textbf{J.\:P.\:\AA{}sen} and S.\:Holm\\
{\it Proc. IEEE Ultrasonics Symposium 2012 \todo{add pages}.}\\\\
Summary of paper...

\section{Discussion}
The main contributions of this thesis are:
\begin{enumerate}
\item a GPU implementation of the Capon beamformer
\item a GPU implementation of the beamspace Capon beamformer
\item real-time beamspace Capon beamforming for cardiac ultrasound imaging
\item the first investigation of Capon beamforming applied on multiple frames in medical ultrasound imaging (simulated, \textit{in-vitro}, and \textit{in-vivo})
\item an investigation of shift-invariance for the Capon beamformer
\item a method for improved shift-invariance of the Capon beamformer
\item a method for reduced blood-pool noise in volume rendering of cardiac ultrasound volumes
\item a GPU implementation of the adaptive volume rendering method
\item a GPU implementation of simple paint-like simulation tool
\item several discussions on how to utilize the GPU for computationally intense algorithms  
\end{enumerate}

Discuss all papers.

Paper1

Paper2

Paper3

Paper4

Paper5

\section{Conclusion}

Concluding remarks...

\section{Future work}

All five papers...

\endinput
% Thesis introduction
% Author: Tore.
%

General introduction to ultrasound, GPUs etc...

\section{General-purpose computing on graphics processing units}

Architecture. See Paper II.

About theoretical FLOPS: Note that these are theoretical numbers and the actual throughput typically is algorithm dependent.

\section {Ultrasound imaging}

Small section about heart anatomy.

Typical probes, scan sequences, resolution and sampling.
							
\subsection{Basic beamforming}

Delay and sum. Apodization. (See intro to Paper II or III).
Time v.s. phase delays.

\subsection{Adaptive Beamforming}\label{sec:adaptbf}

Intro to adaptive beamforming. List other variants (LCA, beamspace, Eigen space etc.). 

Beamspace data is typically refers to the polar grid that cardiac ultrasound data is located in prior to scan conversion. In combination with the Capon beamformer, beamspace refers to the K-space representation of the impinging signals (hence the \nom{FFT}{Fast fourier transform} of the channel data). 

Not phase aberration correction.

Add section about the computationally complexity. How many flops are required per rx-beam etc...

Add section about how to present data (max v.s mean etc.)
						
\subsection{Shift invariance}

\section{Volume rendering}

Get section from master theses. Ray casting and opacity functions.

\subsection{Adaptive volume rendering}

Visibility driven visualization.

\section{Field Simulations}

Small chapter about different simulation tools (See hos paper).
			
\endinput


%%% My commands
%\renewcommand{\vec}[1]{\mathbf{#1}}

\newcommand{\img}{Paper3/img/}
\epstopdfsetup{outdir=Paper3/img}

\newcommand\multimedia[1]{\textbf{{\color{red}[#1]}}}
%\newcommand\comment[1]{\textit{{\color{red}(#1)}}}

\begin{abstract}
%\boldmath
If an ultrasound imaging system provides a presentation of a moving object which is sensitive to small spatial shifts, the system is said to be locally spatially shift-variant. This can for instance happen if the axial or lateral sampling is insufficient. The Capon beamformer has been shown to provide increased lateral resolution in ultrasound images. Increased lateral resolution should demand denser lateral sampling. However, in previous literature on Capon beamforming for medical ultrasound imaging only single frame scenarios have been simulated. Temporal behaviour and effects caused by the increased resolution and lack of oversampling have therefore been neglected.

In this paper we analyse the local lateral shift-invariance of the Capon beamformer when imaging moving objects. We show that insufficient lateral sampling makes an imaging system based on the Capon beamformer laterally shift-variant. Different methods for oversampling on transmit and receive are then discussed and investigated in order to improve on it. It is shown that lateral shift-invariance can be improved by oversampling based on phase rotation on receive. This without affecting the acquisition frame rate and with a minor change in processing complexity.
\end{abstract}

\section{Introduction}

In the last decade several authors have applied the Capon beamformer [also known as the minimum variance beamformer] for medical ultrasound imaging \cite{Synnevag2007, Vignon2008, Viola}. The method has been shown to improve lateral resolution and edge sharpness in ultrasound images \cite{Synnevag2007, Synnevag2009, Chen2011}. This is accomplished by adaptively adjusting the array weights based on element-to-element covariance in order to minimize interference. The method is, however, also highly sensitive to model errors \cite{Mestre2006, Widrow1982, Wax1996, Wax1996a}. This has led to the development of several robust versions of the Capon beamformer with e.g. spatial smoothing \cite{Shan1985} or diagonal loading \cite{JianLi2003} of the covariance estimate, or by applying steering vector uncertainty sets \cite{Lorenz2005, Rubsamen2013}. Spatial smoothing and diagonal loading, together with temporal smoothing of the covariance estimate, have been applied in earlier work on Capon beamforming for medical ultrasound imaging \cite{Synnevag2009}. The first two trade resolution for robustness and temporal smoothing helps to preserve speckle statistics for high resolution parameters \cite{Synnevag2007a}. Steering vector uncertainty sets, however, have not been applied for medical ultrasound imaging.

% but at an increased computational expense \cite{So2011}.

%In recent work we introduced an implementation of the Capon beamformer capable of processing ultrasound images at interactive frame rates \cite{Asen}. In the same paper we discovered that oversampling had to be applied to make the imaging system locally spatially shift-invariant. %It is therefore now interesting how a moving object appears after Capon beamforming has been applied. %Earlier work on Capon beamforming for medical ultrasound imaging has investigated the method's capabilities in single frames only, and temporal behaviour and effects have therefore been neglected. 
%

A steering vector uncertainty set reduces signal-nulling which appears if the signal of interest is present during the covariance estimation process and if this estimate is later used with a steering vector that does not match the signal propagation vector exactly. This self-nulling is actually how the Capon beamformer obtains its super resolution capabilities, but the signal can also get lost in the process. In \cite{JianLi2003} it was shown that most methods based on steering vector uncertainty sets are equivalent to diagonal loading where the loading factor depends on the steering vector uncertainty. In that way they make up for a given uncertainty by making the Capon beam wider. Hence, the maximal achievable resolution is decreased if the initial steering vector spacing was too coarse. In medical ultrasound imaging this initial set of steering vectors is governed by the number of receive beams. In this paper we will make use of an additional set of steering vectors to reduce the self-nulling effect by means of oversampling. The maximal achievable resolution is then maintained without the presence of severe self-nulling.

In \cite{Cox1973}, Cox presents in great detail how the amount of mismatch between the selected steering vector and the signal propagation vector impacts the beamformer output in simple situations. He also states the following important fact about the Capon beamformer [the $\vec{k}_3$ processor]: 
\begin{quote}
"\textit{... a $\vec{k}_3$ processor requires more closely spaced beams than a ... $\vec{k}_1$-processor in order to avoid serious signal suppression effects being introduced on signals arriving from directions between the beams}", 
\end{quote}
where $\vec{k}_1$ is the delay-and-sum (DAS) beamformer. In Fig.\,\ref{fig:das_capon_beams} we have plotted the two-way response for DAS and the Capon beamformer using (26) and (32) in \cite{Cox1973}. A 64-element linear array with $\lambda/2$ element spacing, where $\lambda$ is the wavelength, is scanned over a sector equal to four times the array resolution using narrow band excitation. Located in the farfield of the sector is an object moving parallel to the array surface. The plot shows the high precision scanned output when the object is located at plus and minus half the required beam spacing for DAS [See (\ref{eq:resolution})]. %No spatially white noise is present. Hence, the interference-noise matrix ($\sigma_0^2\mat{Q}$ in \cite{Cox1973}) is the identity. 
The effect of typical subarray averaging and diagonal loading have been added to the Capon beamformer output [subarray length of 32 and a diagonal loading factor of 0.01]. We see that a DAS beamformer, using the required beam spacing, could yield a $1.9$ dB scalloping loss \cite{Harris1978} if the object is moving parallel to the array. However, if the Capon beamformer is used with the same beam density the loss could be as large as $27$ dB. This imaging system is then highly sensitive to small spatial shifts, either by the object or the array itself. A $1.9$ dB scalloping loss for the Capon beamforming is obtained for this specific example when the beam spacing or object distance is decreased with a factor of 25. 

\begin{figure}[!t]
\centerline{
\includegraphics[width=0.8\linewidth]{\img fig1_2way.eps}
}
\caption{Estimated drop in output power mid-way between receive beams for delay-and-sum and Capon beamforming [with subarray length of 32 and a diagonal loading factor of 0.01] applied in a farfield-narrowband setting. %The scanned output from both beamformers is plotted for two different time instances. For the first instances an object is located at an angle equal to half the required beam spacing for delay-and-sum, and for the latter the object is moved an angular distance corresponding to the system resolution. If the beam density was reduced to one beam per system resolution the maximum scalloping loss is found at zero degrees.
}
\label{fig:das_capon_beams}
\end{figure}

In previous literature on Capon beamforming for medical ultrasound imaging only single frame scenarios have been simulated. The scalloping loss effect has therefore been passed over in silence by either oversampling on transmit, by carefully positioning point scatterers exactly on transmit-receive beams, or by ignorance. In our first work where Capon beamforming was applied on a loop of images containing moving objects \cite{Asen2012, Asen} we commented on the effect, but no solution was presented other than oversampling on transmit when simulating. The same was done by Jensen and Austeng \cite{Jensen2012}.

In this paper we investigate the issue further. We show that when a moving point scatterer is imaged with the Capon beamformer with the same beam spacing which is sufficient for DAS, the predicted scalloping loss is observed both in simulations and \textit{in vitro} data. The question is then, what can we do in order to improve on the situation without affecting the acquisition frame rate and processing throughout. %We try two answer this by studying the scalloping loss of the Capon beamforming and investigating different methods aimed at reducing it. We also discuss how the different methods will affect the acquisition frame rate and the processing complexity. 

The next section gives an introduction to Capon beamforming together with a short introduction to lateral sampling and shift-invariance. In Section \ref{sec:capon_LLSI} we study the scalloping loss yield by the Capon beamformer and its local lateral shift-invariance property.  Different methods for reduced scalloping loss and by that improved local lateral shift-invariance are presented in Section \ref{sec:methods}. Then, in Section \ref{sec:res_paper3} we present the results of applying the most promising method on \textit{in vitro} data. Finally we discuss our findings in Section \ref{sec:dis_paper3} and give our conclusions in Section \ref{sec:con_paper3}.

%\comment{Next sections summary.}

% needed in second column of first page if using \IEEEpubid
%\IEEEpubidadjcol

\section{Background}
\subsection{Standard Beamforming}
The standard form of time domain array beamforming is the DAS beamformer where the output $z$ is calculated as:
\begin{align}\label{eq:das}
z[n] = \sum_{m = 0}^{M-1}w_m^*x_m[n - \Delta_m[n]] = \vec{w}^H\vec{x}[n],
\end{align}
where $M$ is the number of elements or channels in the array, $\Delta_m[n]$ is the per-element focusing and steering delay, and $\vec{w}$ is a weight vector or window function. The weight vector is typically selected in order to trade resolution for a lower side lobe level. Hence, the weights are typically real and their K-space response is then symmetric.

\subsection{Capon Beamforming}
The Capon beamformer produces a weight vector, like the one in (\ref{eq:das}), based on the impinging signal and an optimization criterion. The problem is as follow \cite{Capon1969}:
\begin{align}
&\min_{\vec{w}} E\{|z[n]|^2\} \rightarrow \min_{\vec{w}} \vec{w}^H \mat{\hat{R}} \vec{w} \label{eq:capon_optimization_criteria} \\
&\text{subjected to } \vec{w}^H\vec{a} = 1,
\end{align}
where $\mat{\hat{R}}$ is a sample covariance matrix. Hence, the resulting weight vector minimizes the output power while maintaining unit gain in the steering direction $\vec{a}$. %The covariance matrix is usually unknown and has to be estimated from the element data. The covariance matrix in (\ref{eq:capon_optimization_criteria}) is therefore substituted with the sample covariance matrix.

The sample covariance matrix can be estimated for an active broadband system in the following way \cite{Synnevag2009}:
\begin{align}
\mat{\breve{R}}[n] = \frac{1}{N_LN_K}\sum_{n'=n-K}^{n+K} \sum_{l=0}^{N_L-1} \vec{x}_l[n']\vec{x}_l[n']^H,
\end{align}
where $K$ is selected proportional to the pulse length, $N_K = 2K + 1$, $N_L = M-L+1$, and $\vec{x}_l$ is a $L$ long subarray $[x_l[n], \dotso x_{l+L}[n]]$. Finally the matrix is loaded with a diagonal factor for numerical stability and increased robustness, 
\begin{align}
\epsilon &= d*\text{trace}\{\mat{\breve{R}}\}/L\\
\mat{\hat{R}} &= \mat{\breve{R}} + \epsilon\mat{I}.
\end{align} 

The solution to the minimization problem in (\ref{eq:capon_optimization_criteria}) is
\begin{align}\label{eq:capon_weights}
\vec{w}[n] = \frac{\mat{\hat{R}}[n]^{-1}\vec{a}}{\vec{a}^H\mat{\hat{R}}[n]^{-1}\vec{a}} \in \mathbb{C}^L.
\end{align}
Note that one matrix has to be constructed and inverted for each data vector $\vec{x}[n]$ received by the system. %This is indeed computational demanding, but following the innovation in GPU computing, and by applying a novel method in order to reduce the size of the matrix inversion problem \cite{Nilsen2009}, real time processing is now possible \cite{Asen}. 
From (\ref{eq:capon_weights}) it is also clear that complex data [in-phase and quadrature data] can lead to complex weights. The weights can therefore have an asymmetric K-space response with a shifted main lobe and zeros at arbitrary positions. %The beam can therefore be micro-steered by phase rotation.

%Nilsen et. al has proposed to apply beamspace processing to further reduce the computations required. With beamspace we mean transforming the channel data from element space to a fan of beams covering the imaging sector, $\vec{x}_{BS} = \mat{B}x$.  The steering vectors in $\mat{B}$, the so-called Butler matrix, determines each beams direction in space. We can therefore easily remove beams where no interference is present by removing rows in $\mat{B}$. For a focused system like medical ultrasound, where the received signal is concentrated in a narrow band around broadside, there is a small percentage of the beams that contains almost all of the energy. Nilsen et. al has shown that as little as three beams still produced results comparable with applying the full Capon estimation in element space.

\subsection{Lateral Sampling and Shift-Invariance}

The required lateral sampling in an ultrasound image for a given center frequency, $f_c$, and bandwidth, $B$, is given by the aperture size, $D$, and the wavelength of the center frequency, $\lambda_c = c/f_c$, where $c$ is the speed of sound. The required Nyquist beam spacing is then:
\begin{align}
\delta\theta < \frac{\lambda_c}{2qD}, \label{eq:resolution}
\end{align}
where the factor two is the effect of a two-way acquisition \cite{Hergum2009}, and where $q$ is an oversampling factor which can be applied to reduce folding of frequencies between $f_c$ and $f_c + B/2$. In this paper we will use the $q$-factor to perform oversampling prior to Capon beamforming.

Equation (\ref{eq:resolution}) was used to calculate the object positions in Fig.\,\ref{fig:das_capon_beams}. Therefore, if (\ref{eq:resolution}) is not fulfilled, the resulting scalloping loss for the DAS beamformer could become visible in the image. The system output will then be sensitive to small spatial shifts. We say that the system is locally spatially shift-variant. Since Capon beamforming improves the lateral resolution we will focus on sensitivity to small lateral shifts, or in other words the local lateral shift-invariance property of the imaging system.%This invariance to local movements is what makes ultrasound imaging feasible in general, but it is specially important for quantitative techniques, like speckle tracking of tissue and blood, that correlate time-varying features. %In this paper we analyze the lateral shift-invariance property of the Capon beamformer by studying the beamformer output when applied on simulated and \textit{in vitro} images of moving objects.


In an ultrasound image with 50 dB dynamic range mapped to 256 gray levels, a 1 dB loss corresponds to 5 gray levels. This is approximately equal to the visibility threshold [Weber fraction of 2\%] for gray scale images. A loss larger than 1 dB could therefore end up being visible to the observer. %However, what eventually matters is whether the loss is visible in the image or not.%In our further analysis a system would be regarded as locally laterally shift-invariant if the scalloping loss is less than 1 dB.  %However, depending on viewing light conditions and e.g. the selected gray level mapping, a larger loss might be acceptable.



In order to visualize the local lateral shift-invariance of a given imaging system, we will make use of lateral shift-variance plots (LSV-plots) \cite{Hergum2007}. A LSV-plot is constructed for a given imaging system by imaging a point scatterer moving laterally at constant speed. For each lateral position the root-mean-square (RMS) of the image data is calculated in range, and the resulting beam profiles are stacked in to an image. %For a given scan sequence there are twice as many point positions than there are receive lines. 
The LSV-plots in this paper are displayed as a contour plot with contours on -1, -2, -3, -6, -12 and -24 dB. The maximum for each beam profile is also marked with a black dot, and the amplitude variation among these points is presented in a subplot next to the LSV-plot. The LSV-plots based on simulations have been up-interpolated to 128 samples in each direction using cubic interpolation if the number of receive beams is less than 128.

To further quantify the shift-invariance we have calculated the mean absolute error (MAE) between the maximum point position in each beam profile and a straight line ($f(x)=x$). We will refer to this measure as P-MAE. The MAE of the maximum points' amplitude versus the data maximum is also measured. We refer to this measure as A-MAE. The first measure quantifies the amount of geometric distortion \cite{Hergum2007}, and the latter quantifies the peak gain variation or in other words the scalloping loss. The maximum amplitude deviation, or maximum scalloping loss, along this point trace is also calculated. We present all these numbers together with the LSV-plot.

A given imaging system will then be locally laterally shift-invariant if the contours are diagonal and if the amplitude variation among the maximum points is less than 1 dB. Then there is minimal geometric distortion (low P-MAE) and the scalloping loss will not be visible for the observer (low A-MAE). %Since both effects can occur at the same time it is important to have a measure not effected by geometric distortion.

Figure \ref{fig:das_shiftinvariant} presents such a LSV-plot for an imaging system using the DAS beamformer with a beam spacing equal to (\ref{eq:resolution}). Dotted vertical lines indicate position of transmit-receive beam pairs. The point scatterer response has been simulated using Field II \cite{Jensen1992, Jensen1996a} using a 96 element phased array with a 22 mm azimuth aperture, which was excited with a 2.5 MHz and 1.5 cycles long pulse. Transmit focus in azimuth and elevation were configured to be equal and the point scatterer was moved through this point. The transmit focus is where the ultrasound beam is at its narrowest, hence shift variance effects will appear at this location first. The beam density was calculated  based on (\ref{eq:resolution}) with a speed of sound equal to 1540. We see that the contours in Fig.\,\ref{fig:das_shiftinvariant} are approximately diagonal (low P-MAE) and that the amplitude variation among the maximum points is minimal (low A-MAE). We therefore conclude that the system is locally laterally shift-invariant. %In this, and the following LSV-plots based on simulations, the scan sequence always has a beam at zero degrees and the point scatterer is located in the center of this beam at one time instance.

%The scalloping loss in Fig.\,\ref{fig:das_shiftinvariant} is smaller than what was predicted by Fig.\,\ref{fig:das_capon_beams}. To understand why note that Fig.\,\ref{fig:das_capon_beams} shows the scanned response for a narrowband system while the imaging system in Fig.\,\ref{fig:das_shiftinvariant} uses a broadband pulse. The beam in Fig.\,\ref{fig:das_shiftinvariant} will therefore be more similar to a squared gaussian function than the squared $sinc$ functions depicted in Fig.\,\ref{fig:das_capon_beams}. The imaging system in Fig.\,\ref{fig:das_shiftinvariant} also uses an expanding cosine window on receive.

%The scalloping loss in Fig.\,\ref{fig:das_shiftinvariant} is smaller than what was predicted by Fig.\,\ref{fig:das_capon_beams}. To understand why note that Fig.\,\ref{fig:das_capon_beams} shows the one-way response of a passive array, and the system resolution has therefore been calculated based on the receive aperture only. Whereas the beamspacing in (\ref{eq:resolution}) is derived from the fact that the two-way lateral response of an active rectangular aperture is a $D_{tx}+D_{rx}$ wide triangle in K-space \cite{Hergum2009}. Even though the lateral bandwidth is increased on receive, the distance to the first zero (The Rayleigh criteria) remains the same when multiplying two sinc functions. Sampling according to (\ref{eq:resolution}) is therefore more than what is strictly needed in practise. The loss in Fig.\,\ref{fig:das_shiftinvariant} is because of this smaller than what was predicted.

In Fig.\,\ref{fig:das_shiftvariant} the beam density is reduced by a factor of two compared to Fig.\,\ref{fig:das_shiftinvariant}. The geometric distortion is now increased to a level where the contours are not diagonal. The maximum scalloping loss is almost 4 dB. Hence, the imaging system is no longer locally laterally shift-invariant. We observe that the maximum peak is both geometrically distorted and attenuated when the point is located between two beams and the distance between the beams is too high.

\begin{figure*}[!t]
\centerline{
\begin{tabular}{c}
\subfloat[]{\includegraphics[width=0.7\linewidth]{\img fig2a.eps}\label{fig:das_shiftinvariant}}\\
\subfloat[]{\includegraphics[width=0.7\linewidth]{\img fig2b.eps}\label{fig:das_shiftvariant}}
\end{tabular}
}
\caption{LSV-plots of delay-and-sum beamforming. If contours are diagonal and the amplitude variation among the peak points (black dots) is small, the imaging system is said to be locally laterally shift-invariant. Dotted vertical lines are plotted where transmit-receive beams are located. a) Sampling density according to (\ref{eq:resolution}). The system is locally laterally shift-invariant. b) Undersampling by a factor of two. The system is no longer locally laterally shift-invariant.}
\label{fig:das}
\end{figure*}

\section{Local Lateral Shift-Invariance of the Capon Beamformer}\label{sec:capon_LLSI}
We will now investigate the local lateral shift-invariance of the Capon beamformer with high resolution parameters \cite{Synnevag2009, Asen}. In Fig.\,\ref{fig:capon1} the same point scatterer response as presented in Fig.\,\ref{fig:das_shiftinvariant} has been weighted with Capon weights calculated using (\ref{eq:capon_weights}) with $L = M/2 = 48$, $K=1$ and $d=1/100$. We see how the signal cancellation caused by the difference between the signal propagation vector and the assumed steering vector increases as the point moves away from a given transmit-receive beam. As a consequence the scalloping loss is tremendous, and not far from the predicted value in Fig.\,\ref{fig:das_capon_beams}. %The difference between the predicted and the simulated loss might be caused by several factors. The most important one being that Fig.\,\ref{fig:das_capon_beams} shows a single frequency whereas the RMS operator used in Fig.\,\ref{fig:capon1} average a broadband signal in range. For a slice through the center of the pulse the scalloping loss is below -30dB.

The diagonal factor $d$ and the effective adaptive aperture size $L$ can be used to control the mainlobe width of the Capon beamformer. In the extreme cases, $d\gg1$ and $L=1$, the Capon beamformer becomes either a triangle or uniform-weighted DAS beamformer respectively, and a beam spacing according to (\ref{eq:resolution}) would then be sufficient. In Fig.\,\ref{fig:capon2} we show how the scalloping loss is reduced when the amount of diagonal loading is increased. 

We observe that Fig.\,\ref{fig:capon2} now contains a scaled version of the pattern presented in Fig.\,\ref{fig:das_shiftvariant}. From Fig.\,\ref{fig:capon} it should be clear that Capon beamforming either requires increased lateral sampling, or the resolution has to be reduced to the same level as DAS in order to make the imaging system locally laterally shift-invariant. It would be unwise to do all the computation involved with Capon beamforming with a high $d$ value and end up with a resolution equal to DAS. However, we see that diagonal loading can be a useful tool if resolution has to be traded for the need of oversampling. The same is true for the subarray length parameter.

\begin{figure*}[!t]
\centerline{
\begin{tabular}{c}
\subfloat[]{
\includegraphics[width=0.7\linewidth]{\img fig3a.eps}\label{fig:capon1}
}\\
\subfloat[]{
\includegraphics[width=0.7\linewidth]{\img fig3b.eps}\label{fig:capon2}
}
\end{tabular}
}
\caption{LSV-plots of Capon beamforming with $L=M/2=48$ and $K=1$, and with sampling corresponding to (\ref{eq:resolution}). a) Capon beamforming with $d=1/100$ as diagonal loading. The system is not locally laterally shift-invariant. b) The diagonal loading factor $d$ is increased to 1. The scalloping loss is reduced, but the system is still not locally laterally shift-invariant.}
\label{fig:capon}
\end{figure*}

\section{Oversampling Methods}\label{sec:methods}
In this section we will discuss different methods aimed at improving the local lateral shift-invariance of the Capon beamformer. From the previous section it should be clear that we somehow need to increase the lateral sampling to make the Capon imaging system locally laterally shift-invariant. In this section we will investigate how this oversampling should be conducted, how large it needs to be and how the proposed methods will affect the acquisition frame rate and the processing complexity.

\subsection{Oversampling on Transmit}
A straight forward way of obtaining sufficient sampling in an active imaging system with an equal number of transmit (tx) and receive (rx) beams is to increasing the number of tx-beams until the system is locally laterally shift-invariant. In Fig.\,\ref{fig:capon_oversampling_a} and Fig.\,\ref{fig:capon_oversampling_b} we have increased the tx-beam density by a $q$-factor of four and sixteen respectively. As expected the geometric distortion and scalloping loss decreases with increased lateral sampling. Note that Fig.\,\ref{fig:capon_oversampling_b} is zoomed compared to Fig.\,\ref{fig:capon} in order to better see the beam-to-beam variation.

From Fig.\,\ref{fig:capon_oversampling_b} we see that sixteen times oversampling reduces the scalloping loss significantly. Even though high frequency variations are reduced, the maximum is still larger than 1 dB. An even higher level of oversampling is therefore required in order to have a completely local lateral shift-invariant Capon beamformer when a moving point scatterer is simulated without noise and with the same parameters as in Fig.\,\ref{fig:capon}. %$1/100$ in diagonal loading. 
Note that a diagonal loading factor of $1/100$ will introduce a white noise component in the sample covariance matrix $20$ dB lower than the signal present. Hence, the covariance has been estimated from data which appeared to have an SNR of 20 dB.

Even though increasing the number of transmits improves the lateral shift-invariance it will also reduce the acquisition frame rate with a factor equal to the oversampling factor. The processing complexity will also increase equally. For these two reasons, oversampling on transmit is something which is desirable to avoid in real time ultrasound applications, and therefore we will not discuss this method further.

\begin{figure*}[!t]
\centerline{
\begin{tabular}{c}
\subfloat[]{
\includegraphics[width=0.7\linewidth]{\img fig4a.eps}\label{fig:capon_oversampling_a}
}\\
\subfloat[]{
\includegraphics[width=0.7\linewidth]{\img fig4b.eps}\label{fig:capon_oversampling_b}
}
\end{tabular}
}
\caption{Capon beamforming as described in Fig.\,\ref{fig:capon} with oversampling on transmit. The figure is zoomed compared with Fig.\,\ref{fig:capon} to better see the beam-to-beam variation. a) Four times oversampling ($q=4$). b) Sixteen times oversampling ($q=16$). The maximum scalloping loss is reduced, but it is still too high for the system to be locally laterally shift invariant.}
\label{fig:capon_oversampling}
\end{figure*}

\subsection{Oversampling by Parallel Receive Beamforming}
Parallel receive beamforming (PRB) means that multiple narrow receive beams (rx-beams) are constructed from one broad tx-beam. Then, by reducing the number of tx-beams and relying on multiple rx-beams in order to get sufficient sampling on receive, it is possible to achieve high frame rate imaging. However, PRB also introduces geometrical distortion which to some degree can be compensated for by using e.g. synthetic transmit beamforming (STB) \cite{Hergum2007, Denarie2013}. In \cite{Rabinovich2013} it is shown that STB works well with Capon beamforming, something which is also verified in \cite{Asen}. In the following discussion we consider corrected PRB as being equal to oversampling on transmit when it comes to image quality. The LSV-plots for PRB are therefore not included.

In cardiac ultrasound imaging, PRB is typically used in order to keep the number of rx-beams high while the number of tx-beams are reduced. When using PRB for oversampling we would like to maintain the same number of tx-beams while the number of rx-beams are increased. The acquisition frame rate will then not be affected. On the other hand, more rx-beams will increase the number of covariance matrices and thereby the number of matrix-inversions with a factor equal to the number of parallel receive lines. The processing times reported in \cite{Asen} will then also increase with the same factor. Since more than sixteen times oversampling is required in order to achieve lateral-shift-invariant and high-resolution imaging of a moving point scatterer, none of the implementations described in \cite{Asen} would be real time any more. We therefore need a different way of oversampling our imaging sector.

\subsection{Oversampling by Phase Rotation}
The Capon beamformer formula in (\ref{eq:capon_weights}) gives a direct opportunity for oversampling using phase rotation. The steering vector $\vec{a}$ has in previous work been set to $\vec{1}$ since ultrasound data have to be pre-delayed. However, the steering vector can also, as in narrow band applications, be varied over a set of pre-defined vectors. In a narrow band setting the beam can make a sweep across the whole imaging sector through phase steering. This is also possible in broadband applications as long as the phase rotation is less than one pulse length (See coarse-fine beamforming in \cite{Thomenius} and its references). The maximal steering angle is approximately given by
\begin{align}
\theta_{max} = 2\alpha \delta\theta,
\end{align}
where $\alpha$ is the pulse length in $\lambda$'s, and $2$ and $\delta\theta$ is the oversampling factor and beam spacing from (\ref{eq:resolution}). For our simulations $\theta_{max} = 3\delta\theta$.
It is therefore possible to make a sweep based on phase rotation between a given beam and half-way to its two nearest neighbours without introducing large errors. Phase rotation is actually what the Capon beamformer uses internally to micro-steer away from bright point scatterers and to enhance edges.

For a uniform linear array located along the x-axis the steering vector $\vec{a}_\theta$ should be calculated as 
\begin{align}
\vec{a}_\theta = 
\begin{bmatrix}
e^{-j\frac{2\pi}{\lambda_c}x_0\sin(\theta)} \\
e^{-j\frac{2\pi}{\lambda_c}x_1\sin(\theta)} \\
\vdots \\ 
e^{-j\frac{2\pi}{\lambda_c}x_{L-1}\sin(\theta)}
\end{bmatrix}
\end{align}
where $x_i$ is the element position and $\theta$ is swept from $-\delta\theta/2$ to $\delta\theta/2$. Note that the steering vector must be calculated for an $L$ element symmetric subarray, even though the full array has $M$ elements. This subarray should have the same element pitch, $|x_i - x_{i-1}|$, as the full array. %The steering error introduced by this steering vector for a frequency band $B$ is less than $B/2 \times \delta\theta/2$.

The steering angles can be spread out in many different ways. If the number of steering angles is $N_\theta$ we have selected a uniform distribution of $\theta$  with $N_\theta/2$ angles on each side of the original steering direction when $N_\theta$ is even.%, and $(N_\theta-1)/2$ [including the original steering direction] if $N_\theta$ is odd. 

Figure \ref{fig:capon_oversampling_pr} shows the phase rotation method applied on the same scan sequence as in Fig.\,\ref{fig:capon} with four and sixteen times oversampling ($q$-factor) on receive respectively. Note that the figure is zoomed compared with Fig.\,\ref{fig:capon}. As with oversampling on transmit there is some variation when the point moves in and out of focus. The maximum scalloping loss is therefore large. The extreme local variations are however reduced significantly. Comparing Fig.\,\ref{fig:capon_oversampling} with Fig.\,\ref{fig:capon_oversampling_pr}, the errors introduced by the phase rotation method are best seen in sidelobes.  

Since the phase rotation method operates on the same channel data as in Fig.\,\ref{fig:capon1}, there will be no reduction in the acquisition frame rate by applying this method. In addition, since the same number of covariance matrices are constructed and inverted, there is no extra cost involved here as well. The only additional cost will be the multiplication of the extra steering vectors with the inverted covariance matrix. If the equation $\mat{\hat{R}}\vec{a} = \vec{b}$, as in \cite{Asen}, is solved using the Gauss-Jordan algorithm, we can solve the problem for all steering vectors at once by row reducing the matrix $[\mat{\hat{R}},  \vec{a}_{\theta_1}, \vec{a}_{\theta_2}, \cdots \vec{a}_{\theta_{N_\theta}}]$ to the following form $[\mat{I},  \vec{b}_{\theta_1}, \vec{b}_{\theta_2}, \cdots \vec{b}_{\theta_{N_\theta}}]$. This is an $O(L^3)$ operation as long as $N_\theta$ is not much greater than $L$.

\begin{figure*}[!t]
	\centerline{
		\begin{tabular}{c}
		\subfloat[]{
			\includegraphics[width=0.7\linewidth]{\img fig5a.eps}
		}\\
		\subfloat[]{
			\includegraphics[width=0.7\linewidth]{\img fig5b.eps}
		}
		\end{tabular}
	}
	\caption{Capon beamforming as described in Fig.\,\ref{fig:capon} with oversampling based on phase rotation. The figure is zoomed compared with Fig.\,\ref{fig:capon} to better see the beam-to-beam variation. a) Four times oversampling on receive ($q=4$). b) Sixteen times oversampling on receive ($q=16$).}
	\label{fig:capon_oversampling_pr}
\end{figure*}

\section{In Vitro Phantom Data}\label{sec:res_paper3}
In Fig.\,\ref{fig:LSV_invitro} we present the result of applying the Capon beamformer on \textit{in vitro} data with and without oversampling by phase rotation. The data was acquired using a modified Vivid E9 high-end cardiac ultrasound scanner by GE Vingmed Ultrasound, Horten, Norway, equipped with a M5S-D phased transducer operating in harmonic mode at 3.34 MHz. The transmit beam density was set to match (\ref{eq:resolution}) based on this frequency, and 111 tx-beams where therefore distributed on a 65\degree sector. Transmit focus is located at 8 cm range. Note that in harmonic imaging the generated transmit beam at the harmonic frequency will be wider and have lower sidelobes than the beam resulting from transmitting at the harmonic frequency \cite{Fedewa2004}. The lateral sampling is therefore increased compared with our simulations. A tissue mimicking phantom containing wire targets and anechoic cysts was imaged, and the data was acquired while the probe was moved across the phantom manually. Fig.\,\ref{fig:phantom_das_image} shows the first frame out of 14 in the acquired loop processed with DAS beamforming. %Movies of the whole loop processed with DAS and Capon beamforming with and without ten times oversampling by phase rotation can be found in the supplementary material \multimedia{Videos}.

One point scatterer in Fig.\,\ref{fig:phantom_das_image} has been selected [white box], and then used to generate LSV-plots. The point scatterer is located close to the transmit focus. Figure \ref{fig:phantom_das_LSV} shows the resulting LSV-plot when using the DAS beamformer. The scalloping loss is less than 1 dB, and the system is by that locally laterally shift-invariant. As a consequence point scatterers have a consistent appearance when subjected to motion \multimedia{Media Movie 1}.  

Figure \ref{fig:phantom_capon_LSV} shows the same point scatterer response processed with the Capon beamformer without oversampling. For the frame where the point scatterer is at -6 degrees the receive beam direction is very close to the signal propagation vector. Hence, we see a high amplitude value at this location. In almost all the other frames the signal loss is massive (at most 10 dB). The result is blinking point scatterers when subjected to motion \multimedia{Media Movie 2}. 

Figure \ref{fig:phantom_capon_LSV_16PR} presents the result of applying sixteen times oversampling by phase rotation. We observe how the amplitude of the point scatterer is better preserved in all frames, and the scalloping loss is not far from our 1 dB threshold. Note that the oscillation along the diagonal does not indicate shift-variance. It is a result of lack of control when the probe was manually dragged across the phantom. In order to get a diagonal without peaks the probe has to be dragged across the phantom with a speed which matches the acquisition frame rate. In addition, due to the increase in lateral resolution, the Capon beamformer requires smaller frame-to-frame movements than DAS. Looking at the video of Capon beamforming with sixteen times oversampling by phase rotation \multimedia{Media Movie 3} the blinking is now highly reduced. The same can be seen from Fig.\,\ref{fig:img_invitro} where point scatterers appear brighter in the image with oversampling by phase rotation.%The system is however not fully shift invariant. To achieve this a higher level of oversampling, %We therefore conclude that the system is now locally laterally shift invariant.

%Acquire channel data from a phantom containing point wire targets and cysts while moving the probe laterally. Confirm the improved shift-invariance when using phase rotation. See if improved sampling also increase cyst contrast.
%Make plot of contrast between two areas in a simulation for different values of $L$. Make the same plot for different number of rx beams. Do the same for an in-vivo image. The rational being that contrast is also improved by denser rx-sampling not just shift-invariance. 

%\begin{figure}\caption{Plot of contrast vs oversampling factor (phase rotation) in in-vivo data.}
%\end{figure}

%\subsection{In vitro Images}
%Acquire channel data from a phantom containing point wire targets and cysts while moving the probe laterally. Confirm the improved shift-invariance when using phase rotation. 

\begin{figure*}[!t]
\centerline{
\begin{tabular}{cc}
\subfloat[]{\includegraphics[width=0.5\linewidth]{\img fig6a.png}\label{fig:phantom_das_image}}
\subfloat[]{\includegraphics[width=0.5\linewidth]{\img fig6b.eps}\label{fig:phantom_das_LSV}}\\
\subfloat[]{\includegraphics[width=0.5\linewidth]{\img fig6c.eps}\label{fig:phantom_capon_LSV}}
\subfloat[]{\includegraphics[width=0.5\linewidth]{\img fig6d.eps}\label{fig:phantom_capon_LSV_16PR}}
\end{tabular}
}
\caption{LSV-plots generated from \textit{in vitro} phantom data where the probe has been manually dragged across the phantom. See the supplementary material for videos of the whole loop processed with the different beamformers. Capon beamforming has been performed with the same settings as in Fig.\,\ref{fig:capon1}. P-MAE is not shown due to a lack of probe movement control. a) First frame in the recorded loop processed with delay-and-sum beamforming  \multimedia{Media Movie 1}. A white box indicates the point scatterer used to construct the LSV-plots.  b) LSV-plot for delay-and-sum beamforming. c) LSV-plot for Capon beamforming without oversampling. d) LSV-plot for Capon beamforming with sixteen times oversampling using phase rotation.}
\label{fig:LSV_invitro}
\end{figure*}

\begin{figure*}[!t]
\centerline{
\begin{tabular}{c}
\subfloat[]{\includegraphics[width=0.7\linewidth]{\img fig7a.png}\label{fig:phantom_capon_image}}\\
\subfloat[]{\includegraphics[width=0.7\linewidth]{\img fig7b.png}\label{fig:phantom_capon_16PR_image}}
\end{tabular}
}
\caption{First frame from Fig.\,\ref{fig:phantom_das_image} processed with Capon beamforming. a) Without oversampling \multimedia{Media Movie 2}. b) With sixteen times oversampling using phase rotation \multimedia{Media Movie 3}. Notice how the point scatterers appear brighter in the image with oversampling.}
\label{fig:img_invitro}
\end{figure*}

%\subsection{In-Vivo Images}
%Investigate if phase rotation improves contrast in in-vivo images of the heart or liver.

%\begin{figure*}[!t]
%\centerline{
%\subfloat[]{\includegraphics[width=0.5\linewidth]{\img phantom1_capon_processing_K=0_L=24_full.png}\label{fig:in-vivo_capon}}
%\hfill{}
%\subfloat[]{\includegraphics[width=0.5\linewidth]{\img phantom1_capon_processing_K=0_L=24_8pr_full.png}\label{fig:in-vivo_capon_8pr}}
%\hfill{}
%\subfloat[]{\includegraphics[width=0.3\linewidth]{\img das_undersampled2_2x.eps}\label{fig:das_shiftvariant}}
%}
%\caption{\comment{For the moment phantom data. Need to acquire some high quality in-vivo data.}. In-vivo data. a) Capon image. b) Capon image with oversampling using 8PR.}
%\label{fig:das}
%\end{figure*}

\section{Discussion}\label{sec:dis_paper3}
%Note that in this paper we have investigated a single array configuration and a limited parameter range for the Capon beamformer. The goal was not to make a reference for which oversampling factor to use in different scenarios, but rather do a detailed investigation of the scalloping loss effect. We hope with this investigation that future readers will be able to find out them self what oversampling factor to use for their system. 

Faced with an imaging scene containing bright point scatterers or sharp edges, one could wonder how large the oversampling factor needs to be in order to avoid visible scalloping losses. In his seminal paper, Cox \cite{Cox1973} derives the required beam density for a given output SNR. For instance 47 dB requires a beam density for the Capon beamformer ten times the one given by (\ref{eq:resolution}) [with $q=1/2$]. Since resolution for the Capon beamformer depends on SNR we would recommend to first use the dynamic range in the image as guidance. The amount of diagonal loading will also dictate the maximal achievable resolution since $10\log{d}$ can be interpreted as the maximal input SNR seen by the Capon beamformer. Fig.\,\ref{fig:das_capon_beams} shows that 25 times oversampling is required for a $d$-value of $1/100$. When \textit{in vitro} data contains additional noise this oversampling requirement is reduced.%Cox measures resolution as the 3 dB saddle point between two impinging monochromatic plane waves. For ultrasound data only a 50 \% increase in resolution was found in \cite[Fig.~7]{Asen} for a 30 dB output SNR when measured in a similar way as Cox [6 dB saddle point]. However, sixteen times oversampling was required to make the imaging system shift invariant in the same paper. In this paper more than ten times oversampling was found to be required for a 40 dB output SNR (after array gain) in simulations, however for the \textit{in vitro} data ten times oversampling was enough for the system to be locally laterally shift-invariant. This might be a result of improved initial sampling (as discussed in Section \ref{sec:res}) and a higher initial noise level before diagonal loading was applied. 

Relating the presented phase rotation method to earlier work we find that it is similar to a steering vector uncertainty set. However, where these set-based methods typically has a criterion for selecting the best output after applying all the steering vectors contained in the set, oversampling by phase rotation outputs all. It can also be thought of as a narrow band approach to parallel receive beamforming where Capon's method is applied on all receive beams.

Oversampling by phase rotation can therefore introduce similar block artifacts as parallel receive beamforming (PRB). To remove these artifacts we could distribute our steering vectors in an overlapping scheme [like rx-beams are spread out for STB]. We then get a phase rotation version of STB where e.g. eight phase rotated beams are averaged into four. If the geometric distortion, or stripes, seen at the bottom of the sector in the video with sixteen times oversampling \multimedia{Media Movie 3} is removed by such technique is left for future work.

When oversampling is applied it becomes evident that the high resolution settings used in this paper might be too extreme, both with respect to the level of required oversampling and the point scatterer's appearance. Traditionally in medical ultrasound imaging the point spread function is always wider laterally than in range. With the Capon beamformer this is now changed. A more symmetric point spread function while preserving speckle statistics is obtained with the parameters $L=M/4$ and $K=0$ \cite{Synnevag2007a}. The level of required oversampling and the computational complexity are also reduced. %Another option is to perform pulse inversion if a spiky symmetric point spread function is desirable.

Note that oversampling is not only essential to keep the amplitude level of moving point scatterers. It should also help to improve contrast and edge definitions. If oversampling is not applied homogeneous speckle regions could get a too low average value and the edge delineation of a moving object will fluctuate in the same way as a moving point scatterer would have sparkled.

Finally we want to stress the fact that this demand for oversampling does not only affect the Capon [or minimum variance beamformer]. Other methods claiming super resolution, like the Eigenspace method, low complexity beamformer etc. \cite{Synnevag2011, Nilsen2009, Mehdizadeh2012a, Kim}, would also require oversampling when improved lateral resolution is obtained. The same applies for high resolution spectral estimation \cite{Ekroll2012}.

%When signal detection (absolute value and square root) is performed before an ultrasound image is displayed the lateral bandwidth is increased. In order to not introduce aliasing in the image, IQ-interpolation is typically used on ultrasound systems to double the lateral beam density before detection. For the same reason we need to increase the lateral sampling when Capon beamforming is applied, but now on a channel data level. The Capon beamformer improves the lateral resolution, hence the lateral bandwidth is increased, and higher lateral sampling is required. 

%One way of reducing the number of matrix inversions in PRB is to estimate the covariance matrix from multiple beams \cite{Jensen2012}, generating one covariance matrix per range.  Tailoring the method of Jensen and Austeng to cardiac ultrasound imaging it makes more sense, due to motion, to construct one covariance matrix per tx-beam using multiple parallel rx-beams. With this solution the number of matrix inversions will not changes, but the time taken to construct the matrices will increase. We will not investigate this method further in this paper.

%\subsection{Impact on Real Time Performance}
%Discuss, and show with benchmarks how the different methods for making Capon beamforming shift-invariant impact the total execution time.

\section{Conclusion}\label{sec:con_paper3}
%In this paper we have investigated the lateral shift-invariance property of the Capon beamformer when imaging moving objects. We have shown that the Capon beamformer is less sensitive to small lateral shifts if oversampling based on phase rotation is applied. This without reducing the acquisition frame rate and with only a minor increase in processing complexity. %In Section \ref{sec:capon_LLSI} it was shown that a Capon beamformer with high resolution parameters is not locally laterally shift-invariant when subjected to motion. The root cause was found to be too coarse lateral sampling. 
%In Section \ref{sec:methods} we presented methods for improved shift-invariance based on oversampling. The most promising method, with respect to improved shift-invariance, reduction in acquisition frame rate and processing complexity combined, was based on phase rotation. 
%In Section \ref{sec:res} we presented results showing that this method also works on \textit{in vitro} data when the ultrasound probe is subjected to motion. 

In this paper we have investigated the lateral shift-invariance property of the Capon beamformer when imaging moving objects. It was shown that Capon beamforming with the same beam density as required for delay-and-sum beforming results in blinking point scatterers when the system is subjected to motion.  We then discussed and investigated different methods based on oversampling on transmit and receive in order to reduced this effect. Finally we showed that the local lateral shift-invariance could be improved by oversampling based on phase rotation on receive. This without affecting the acquisition frame rate and with a minor change in processing complexity. In simulations, this method reduced the maximum scalloping loss by 16 dB. On \textit{in vitro} data the reduction was around 7 dB and the point scatterer blinking was highly reduced. 

\section*{Acknowledgment}


The authors would like to thank Anders S\o{}rnes at GE Vingmed Ultrasound, Horten, Norway for the help with acquiring \textit{in vitro} data.


%%%%%%%%%%%%%%%%                                ~~~~~~~~~~~~~~~~~~~~~~~~~~~~~~~~~~~~~~~~~~~~~~~~~~
% CONDITIONALS %
%%%%%%%%%%%%%%%%                                ~~~~~~~~~~~~~~~~~~~~~~~~~~~~~~~~~~~~~~~~~~~~~~~~~~

\newif\ifPeerReview\PeerReviewfalse             % Whether to create the PeerReview version or
                                                % Journal version
\newif\ifFlatArchive\FlatArchivefalse           % Whether archive is flat (messy) or contain 
                                                % subfolders for graphics etc.
\newif\ifFloatAtEnd\FloatAtEndfalse             % Available in PeerReview mode:
                                                % Place floats at end of document?
\newif\ifTODO\TODOtrue                        % Use todo notes?

%%%%%%%%%%%%                                    ~~~~~~~~~~~~~~~~~~~~~~~~~~~~~~~~~~~~~~~~~~~~~~~~~~
% IEEEtran %
%%%%%%%%%%%%                                    ~~~~~~~~~~~~~~~~~~~~~~~~~~~~~~~~~~~~~~~~~~~~~~~~~~

\ifPeerReview
\documentclass[12pt,journal,captionsoff,onecolumn]{IEEEtran}
\newcommand\CLASSINPUTbaselinestretch{1.66}     % http://theoval.cmp.uea.ac.uk/~nlct/latex/thesis/node17.html
\else
\documentclass[journal]{IEEEtran}
\fi

% \RequirePackage[latin1]{inputenc}%              % Set input encoding (optionally latin1)
% \RequirePackage[T1]{fontenc}%                 % Set font encoding
% \usepackage[norsk]{babel}

%%%%%%%%%%%%%%%%%%%%%%%%%%%%%%                  ~~~~~~~~~~~~~~~~~~~~~~~~~~~~~~~~~~~~~~~~~~~~~~~~~~
% IEEE ''APPROVED'' PACKAGES %
%%%%%%%%%%%%%%%%%%%%%%%%%%%%%%                  ~~~~~~~~~~~~~~~~~~~~~~~~~~~~~~~~~~~~~~~~~~~~~~~~~~


\ifCLASSINFOpdf
   \usepackage[dvips]{graphicx}                 % Might not work. Use 'latex' instead of 
   \ifFlatArchive\else                          % 'pdflatex'
      \graphicspath{./gfx/}
   \fi
\else
   \usepackage[dvips]{graphicx}
   \ifFlatArchive\else
      \graphicspath{./gfx/}
   \fi
\fi

\RequirePackage[table,dvipsnames,svgnames]{xcolor}

\usepackage[cmex10]{amsmath}                    % cmex10 option to be IEEE explore compliant
\interdisplaylinepenalty=2500                   % Allows multiline equations to be broken

% \RequirePackage{amssymb}

\RequirePackage{array}

\ifCLASSOPTIONcompsoc
   \usepackage[caption=false,font=normalsize,labelfont=sf,textfont=sf]{subfig}
\else
   \usepackage[caption=false,font=footnotesize]{subfig}
\fi
\ifCLASSOPTIONcaptionsoff                       % IEEE promoted hack to turn off captions from the 
   \let\MYorigsubfloat\subfloat                 % subfloat package should the captionsoff option
   \renewcommand{\subfloat}[2][\relax]{\MYorigsubfloat[]{#2}} % be specified.
\fi

\ifFloatAtEnd
\ifCLASSOPTIONcaptionsoff                       % Places float at the end of the document when the
  \usepackage[nomarkers]{endfloat}              % captionsoff options is specified to IEEEtrans.cls
  \let\MYoriglatexcaption\caption               % (PeerReview mode)
  \renewcommand{\caption}[2][\relax]{\MYoriglatexcaption[#2]{#2}}
\fi
\fi

\usepackage{fixltx2e}                           % Fix some twocolumn float problems

%\usepackage{stfloats}                          % Allows: \begin{figure*}[!b]
                                                % (double column figures on top/bottom)

\usepackage{url}                                % Support for handling and breaking URLs

% NOTE: PDF hyperlink and bookmark features are not required in IEEE
%       papers and their use requires extra complexity and work.
\newcommand\MYhyperrefoptions{bookmarks=true,bookmarksnumbered=true,
pdfpagemode={UseOutlines},plainpages=false,pdfpagelabels=true,
colorlinks=true,linkcolor={black},citecolor={black},urlcolor={black},
pdftitle={Low Complexity Adaptive Beamformer for Active Sonar Imaging},
pdfsubject={},
pdfauthor={Jo Inge Buskenes},
pdfkeywords={adaptive beamforming, beamforming, complexity, sonar, active}}%
\ifCLASSINFOpdf
   \usepackage[\MYhyperrefoptions,pdftex]{hyperref}
\else
   \usepackage[\MYhyperrefoptions,breaklinks=true,dvips]{hyperref}
   \usepackage{breakurl}                        % Allows 'dvips' driver to break links
\fi

%%%%%%%%%%%%%%%%%%%%%%%                         ~~~~~~~~~~~~~~~~~~~~~~~~~~~~~~~~~~~~~~~~~~~~~~~~~~
% ADDITIONAL PACKAGES %       
%%%%%%%%%%%%%%%%%%%%%%%                         ~~~~~~~~~~~~~~~~~~~~~~~~~~~~~~~~~~~~~~~~~~~~~~~~~~

\usepackage[maxfloats=25]{morefloats}
\newcounter{todoidx}
% \setcounter{todoidx}

\ifTODO
   \definecolor{todobackground}{rgb}{0.95,0.95,0.95}
   \setlength\marginparsep{1pt}
   \setlength\marginparwidth{35pt}
   \newlength\marginparwidthsmall
   \setlength\marginparwidthsmall{\marginparwidth}
   \addtolength\marginparwidthsmall{-7pt}
   \newcommand\todo[1]{%
      \addtocounter{todoidx}{1}%
      {\color{Red}\fbox{\bf\thetodoidx{}}}%
      \marginpar{%
         {\vspace*{-10pt}\color{Red}\fbox{\bf\thetodoidx{}}}\\%
         \fcolorbox{red}{todobackground}{\parbox{\marginparwidthsmall}{\scriptsize #1}}}}

   \newcommand\todopar[1]{\fcolorbox{red}{white}{\parbox{0.97\linewidth}{#1}}}
\else
%    \usepackage[disable]{./todonotes} 
   \newcommand\todo[1]{}
\fi

\newenvironment{narrow}[2]{%
\begin{list}{}{%
\setlength{\topsep}{0pt}%
\setlength{\leftmargin}{#1}%
\setlength{\rightmargin}{#2}%
\setlength{\listparindent}{\parindent}%
\setlength{\itemindent}{\parindent}%
\setlength{\parsep}{\parskip}}%
\item[]}{\end{list}}

\usepackage{float}

%%%%%%%%%%                                      ~~~~~~~~~~~~~~~~~~~~~~~~~~~~~~~~~~~~~~~~~~~~~~~~~~
% MACROS %       
%%%%%%%%%%                                      ~~~~~~~~~~~~~~~~~~~~~~~~~~~~~~~~~~~~~~~~~~~~~~~~~~


\newcommand\Grey[1]{{\color{Grey}#1}}
\newcommand\Red[1]{{\color{Red}#1}}
\newcommand\Blue[1]{{\color{Blue}#1}}
\newcommand\DarkBlue[1]{{\color{DarkBlue}#1}}
\newcommand\LightBlue[1]{{\color{LightBlue}#1}}
\newcommand\Brown[1]{{\color{Brown}#1}}
\newcommand\Green[1]{{\color{Green}#1}}
\newcommand\SeaGreen[1]{{\color{SeaGreen}#1}}
\newcommand\Yellow[1]{{\color{yellow}#1}}
\newcommand\Orange[1]{{\color{orange}#1}}

\newcommand\nn{\nonumber\\}

\newcommand\nmat[1]{\begin{matrix}#1\end{matrix}}
\newcommand\bmat[1]{\begin{bmatrix}#1\end{bmatrix}}
\newcommand\case[1]{\begin{cases}#1\end{cases}}
\newcommand\textbox[2]{\footnotesize\text{\parbox{#1}{\centering\emph{#2}}}}

\newcommand\rand{\text{rand}}
\newcommand\randn{\text{randn}}
\newcommand\rect{\text{rect}}
\newcommand\sinc{\text{sinc}}
\newcommand\tr{\text{tr}}
\newcommand\adj{\text{adj}}

% \newcommand\max{\text{max}}
\newcommand\argmin{\text{argmin}}

\newcommand\qqquad{\quad\qquad}
\newcommand\qqqquad{\qquad\qquad}

\renewcommand\l[1]{\left#1}
\renewcommand\r[1]{\right#1}

% {\text{\parbox{1.5cm}{\centering volume hyper- sphere}}}

%Keyword colouring:
\newcommand\kw[1]{#1}
\newcommand\parm[1]{#1}%\color{Black}#1\color{Black}}

\newcommand\of[1]{\scriptstyle(\parm{#1})\displaystyle}
\newcommand\df[1]{\scriptstyle[\parm{#1}]\displaystyle}
\newcommand\var[3]{#1_\text{#2}\of{#3}}

\newcommand\diag{\text{diag}}

% \raisebox{lift}[extend-above-baseline][extend-below-baseline]{text}
\newcommand\mt[1]{\text{\emph{#1}}} %mt = mathtext
\newcommand\mathnorm{\textstyle}
\newcommand\mathbig[1]{\displaystyle#1\mathnorm}
\newcommand\mathsmall[1]{\scriptstyle#1\mathnorm}
\newcommand\mathtiny[1]{\scriptscriptstyle#1\mathnorm}
\newcommand\sfrac[2]{\scriptstyle\raisebox{0.25pt}[0pt][0pt]{$\frac{#1}{#2}$}\mathnorm}
\newcommand\nfrac[2]{\textstyle\frac{#1}{#2}\displaystyle}

\newcommand\sumu[1]{\sum\limits^{#1}\,}
\newcommand\suml[1]{\sum\limits_{#1}\,}
\newcommand\sumb[2]{\sum\limits_{#1}^{#2}\,}

\newcommand\produ[1]{\prod\limits^{#1}\,}
\newcommand\prodl[1]{\prod\limits_{#1}\,}
\newcommand\prodb[2]{\prod\limits_{#1}^{#2}\,}

%Math macros:
\newcommand\diff[2]{\frac{\kw{d}\,\textstyle #1\scriptstyle}{\kw{d\parm{#2}}}\displaystyle}
\newcommand\ddiff[2]{\frac{\kw{d^2}\,\displaystyle #1\scriptstyle}{\kw{d\parm{#2}}^2}\displaystyle}

\renewcommand\d[1]{\scriptstyle\kw{\,d\parm{#1}}\displaystyle}

% These commands are mutually exclusive. Remember to "renew" in v2.
\newcommand\intb[4]{\int\limits_{#3}^{#4} #1 \d{#2}} % \int{exp}{var}{from}{to}
\newcommand\intl[3]{\int\limits_{#3} #1 \d{#2}} % \int{exp}{var}{for all}
\newcommand\intu[2]{\int #1 \d{#2}} % \int{exp}{var}{for all}

\newcommand\T{^{\scriptscriptstyle T}}
\renewcommand\H{^{\scriptscriptstyle H}}

\renewcommand\vec[1]{\boldsymbol{#1}}
\newcommand\mat[1]{\boldsymbol{#1}}


\renewcommand*\a{\vec a}
\renewcommand*\i{\vec i}
\renewcommand*\k{\vec k}
\newcommand*\n{\vec n}
\newcommand*\p{\vec p}
\newcommand*\s{\vec s}
\newcommand*\w{\vec w}
\newcommand*\x{\vec x}
\newcommand*\y{\vec y}

\newcommand*\A{\mat A}
\newcommand*\B{\mat B}
\newcommand*\C{\mat C}
\newcommand*\E{\mat E}
% \renewcommand*\H{\mat H}
\renewcommand*\P{\mat P}
\newcommand*\eP{\mat{\hat P}}
\newcommand*\R{\mat R}
\newcommand*\Ri{\R^{-1}}
\newcommand*\eR{\mat{\hat R}}
\newcommand*\eRi{\hat{\mat R}\,\!^{-1}}
\newcommand*\Navg{N_\text{avg}}
\newcommand*\W{\mat W}
\newcommand*\X{\mat X}
\newcommand*\Xd{\X_{\!\Delta}}
\newcommand*\Y{\mat Y}

\renewcommand*\L{\mat \Lambda}
\newcommand*\U{\mat U}
% \renewcommand*\t{\mathtiny{^T}}
% \newcommand*\h{\mathtiny{^H}}
\renewcommand*\t{^T}
\newcommand*\h{^H}

\newcommand\D{\vec\nabla} %Del: Vector differential operator - nabla
\newcommand\Dx{\vec\nabla\times}
\newcommand\Dd{\vec\nabla\cdot}

\usepackage{tikz}
\usetikzlibrary{shapes,snakes}
\usepackage{amsmath,amssymb}

\newenvironment{outline}
{\begin{itemize}}
{\end{itemize}}

%    \definecolor{todobackground}{rgb}{0.95,0.95,0.95}
%    \setlength\marginparsep{3pt}
%    \setlength\marginparwidth{42pt}
%    \newlength\marginparwidthsmall
%    \setlength\marginparwidthsmall{\marginparwidth}
%    \addtolength\marginparwidthsmall{-7pt}
%    \newcommand\todo[1]{%
%       \addtocounter{todoidx}{1}%
%       {\color{Red}\fbox{\bf\thetodoidx{}}}%
%       \marginpar{%
%          {\vspace*{-10pt}\color{Red}\fbox{\bf\thetodoidx{}}}\\%
%          \fcolorbox{red}{todobackground}{\parbox{\marginparwidthsmall}{#1}}}}
% 

% correct bad hyphenation here
% \hyphenation{op-tical net-works semi-conduc-tor}


%%%%%%%%%%%%%%%%%%                              ~~~~~~~~~~~~~~~~~~~~~~~~~~~~~~~~~~~~~~~~~~~~~~~~~~
% DOCUMENT START %
%%%%%%%%%%%%%%%%%%                              ~~~~~~~~~~~~~~~~~~~~~~~~~~~~~~~~~~~~~~~~~~~~~~~~~~

% \usepackage{yfonts}


\begin{document}

\title{Outline hints}

\author{Authors%
\IEEEcompsocitemizethanks{\IEEEcompsocthanksitem All authors are with the Department
of Informatics, University of Oslo, Norway.}% <-this % stops a space

% \thanks{Manuscript received April 19, 2005; revised January 11, 2007.}
}

% The paper headers
\markboth{Some journal}%
{Outline hints}

% Publishers ID mark:
%\IEEEpubid{0000--0000/00\$00.00~\copyright~2007 IEEE}

% use for special paper notices
%\IEEEspecialpapernotice{(Invited Paper)}

% for Computer Society papers, we must declare the abstract and index terms
% PRIOR to the title within the \IEEEcompsoctitleabstractindextext IEEEtran
% command as these need to go into the title area created by \maketitle.
\IEEEcompsoctitleabstractindextext{%
\begin{abstract}

\begin{itemize}
\item Todo.
\end{itemize}

\end{abstract}

% Keywords (normally not used for peer reviews)
\ifPeerReview\else
\begin{IEEEkeywords}
Outline hints
\end{IEEEkeywords}
\fi}
% \fi

% make the title area
\maketitle

% This command fixes abstract positioning for compsoc articles:
\IEEEdisplaynotcompsoctitleabstractindextext

% (Optional) Add some extra info on cover page of peer review papers:
% \ifCLASSOPTIONpeerreview
% \begin{center} \bfseries EDICS Category: 3-BBND \end{center}
% \fi

% Insert page break and insert second title (peer review mode)
\IEEEpeerreviewmaketitle

\section{Prerequisites}

\begin{itemize}
\item Research articles. Take note of journals aim for these. Perhaps take impact factor into account.
\item For a given journal, make note of typical article and section lengths.
\item Notice language and outline of such articles as well.
\end{itemize}

IMRAD style:

\section{Abstract}

\begin{itemize}
\item Past tense.
\end{itemize}


\section{Introduction}

\begin{itemize}
\item Establish audience, and outline what they need to know.
\item Purpose: What the audience need to know in order to understand the findings of the paper.
\item Reason: Why is the research in this field necessary, and where does this individual contribution fit in?
\item Active voice.
\item Present tense, as focus will be put on the problem at hand, and respect should be paid to established knowledge.
\end{itemize}

\begin{enumerate}
\item Nature and scope of the problem, with all possible clarity. Motivate!!! Why the subject was chosen, why it is important.
\item Brief review of present litterature for orientation purposes. 
\item Method of investigation? Optionally, why was it chosen? Mediate our way of thinking!
\item Principal results. Some overlap with abstract ok.
\item Principal conclustions suggested by the results (don't keep the reader in suspense). Some overlap with abstract ok.
\end{enumerate}
\begin{itemize}
\item Funnel effect desirable: Importance of overall topic, summarize knowledge about an aspect of it, identify an uresolved question of that aspect, say how paper answers that question.
\end{itemize}

\section{Methods}

\begin{itemize}
\item Describe and defend method sufficently to allow experiment to be repeated.
\item Even though most readers will skip it since method is stated in introduction, results will not be taken seriously if method is insufficiently documented.
\item Use subheadings to group methods, try to make them match the ones used in Results.
\item Like a cookbook: ``How'', and ``how much'' should be precisely answered.
\item Statistical knowledge - of established knowledge - should not be included. If deemed important, cite relevant litterature.
\item When referencing, state the findings, but omit the reasons behind the findings.
\item Passive voice appropriate, of use of ``we''.
\item Past tense.
\end{itemize}


\section{Results}

\begin{itemize}
\item Digest results until only the vital pieces of information remains. Make what is important stand out.
\item Drop variables that are neglible, and be sure to mention it.
\item If negative aspects are meaningful, include them.
\item Strive for clarity! This is the new piece of information offered to the world. Make sure it sees it.
\item Results define everything. The introduction and methods explain why and how they come into existence, discusstion explains what they mean.
\item Avoid redundancy. State data in figures/tables, not in text.
\item Although dominated by passive voice, add some active voice to the mix too.
\item Past tense on own results.
\end{itemize}

\begin{enumerate}
\item Overall description of the experiment, providing ``big picture'' without repeating technicalities of methods section.
\item Results, with crystal clarity.
\end{enumerate}

\section{Discussion}

\begin{enumerate}
\item Present principles, relationships, and generalizations shown by the Results. Discuss, not recapitulate.
\item Point out exceptions, lack of correlation, or unsettled points.
\item Show how results and interpretations agree/contrast with previously published work, and with own expectations.
\item Discuss theoretical implications, as well as possible practical applications. This will largely affect how significant the article will be percieved.
\item State conclusions are clearly as possible.
\item Summarize evidence for each conclusion. Beware of causality leaps!
\item Discussion pairs Introduction. Be sure to answer any question asked there. Don't recapitulate, but extend from where introduction left off.
\item Be sure to emphasize that this is a ``small truth'', not the ``whole truth''.
\item Active voice.
\end{enumerate}


\section{Conclusion}

\begin{enumerate}
\item Take a step back. Summarize experiment, relating it to the big picture.
\item Discuss further work
\item Did the results support / reject the hypothesis stated in the introduction?
\item If the used should remember just one sentence from your article, state it here!
\end{enumerate}

\section{Revision}

\begin{itemize}
\item Information selection wise?
\item Information accurate?
\item Consistency?
\item Organization logical?
\item Clear wording?
\item Points stated brielfy, simply, directly?
\item Grammar, spelling, word use, correct?
\item Check figures and tables for errors/consistency.
\item IEEE compliant?
\end{itemize}

%%%%%%%%%%%%%%%%%%                              ~~~~~~~~~~~~~~~~~~~~~~~~~~~~~~~~~~~~~~~~~~~~~~~~~~
% DOCUMENT APPENDICES %
%%%%%%%%%%%%%%%%%%                              ~~~~~~~~~~~~~~~~~~~~~~~~~~~~~~~~~~~~~~~~~~~~~~~~~~

\appendices



% use section* for acknowledgement
\ifCLASSOPTIONcompsoc
  \section*{Acknowledgments}
\else
  \section*{Acknowledgment}
\fi


The authors would like to thank...


% Can use something like this to put references on a page
% by themselves when using endfloat and the captionsoff option.
\ifCLASSOPTIONcaptionsoff
  \newpage
\fi


% \bibliographystyle{IEEEtran}
% \bibliography{../../Library/library}


\end{document}


